	\newpage
\section{Założenia projektowe – wymagania}		%1

%ctrl+alt+j - przeniesienie z kodu do pdf


Niniejszy projekt ma na celu stworzenie sieci komputerowej dla przedsiębiorstwa, które będzie składać się z następujących elementów:
\begin{enumerate}
    \item Utworzenie własnej domeny AD według formatu \texttt{firma.ad}, gdzie \texttt{firma} to nazwa firmy, dla której przygotowujemy projekt.
    \item Autoryzacja pracownika przy użyciu imiennego konta działającego na wszystkich komputerach w sieci firmowej. Login zgodny ze schematem: \texttt{imie.nazwisko}. Utworzyć minimum po trzy konta dla każdego wydziału. Tworzenie grup oraz kont w domenie wykonywać przez skrypt, który odczyta dane grupy i kont z pliku i utworzy je w domenie AD.
    \item Pracownicy powinni należeć do grupy globalnej odpowiedniej dla wydziału, w którym pracują (przewidzieć 5 przykładowych wydziałów, np. kadry, płace, gospodarczy, marketing, itp. wg własnego uznania). W przedsiębiorstwie przewidziano wydział informatyczny, którego pracownicy mają w pełni administrować domeną przedsiębiorstwa. Tworzenie grup oraz kont w domenie wykonywać przez skrypt, który odczyta dane grupy i kont z pliku i utworzy je w domenie AD.
    \item Pracownicy powinni korzystać z zasobów sieciowych o nazwach: \texttt{wspolny} oraz zasób wydziałowy (oddzielny zasób dla każdego wydziału). Zasoby powinny być udostępnione poprzez klaster pracy awaryjnej, który powinien korzystać z przestrzeni dyskowej (macierzy RAID-1) udostępnionej poprzez iSCSI o przestrzeni wypadkowej 30 GB. Jeśli pozwalają zasoby sprzętowe, to programową macierz RAID-1 oraz iSCSI Target Serwer można zainstalować na oddzielnym serwerze o nazwie \texttt{SMPXX.firma.ad}.
    \item Zasób \texttt{wspolny} ma być mapowany użytkownikowi jako dysk (patrz: założenia projektowe), natomiast zasób wydziałowy jako jeden z dysków (patrz: założenia projektowe) adekwatnie do grupy wydziałowej, w której znajduje się pracownik. Mechanizm mapowania automatyczny przy użyciu polis GPO.
    \item System ma umożliwić instalację stacji klienckich z obrazu udostępnionego na serwerze.
    \item Pracownicy powinni mieć dostęp do drukarek sieciowych udostępnianych poprzez serwer wydruków. Serwer wydruków można zainstalować na oddzielnym serwerze, np. o nazwie \texttt{SPRXX.firma.ad}, lub w przypadku małych zasobów RAM, na serwerze \texttt{SDC}. Dostęp do drukarki sieciowej: \texttt{\textbackslash\textbackslash SPRXX-firma.ad\textbackslash nazwa-drukarki}.
    \item Konfiguracja stacji klienckich w sposób automatyczny.
    \item Na klastrze pracy awaryjnej należy wdrożyć serwer DHCP.
    \item Po udanym wdrożeniu serwera DHCP należy wyłączyć serwer DHCP pracujący na \texttt{SDC}.
    \item Wdrożyć serwer WWW wraz z firmową stroną (nie domyślną) dostępną pod adresem \texttt{www.firma.ad}.
    \item Wdrożyć WordPress współpracujący z usługą IIS Windows Serwera 2022.
\end{enumerate}

