\newpage
\section{Opis użytych technologii}		%2
%(W podpunktach dokonać krótkiej charakterystyki użytych technologii ) 

\subsection{Active Directory} 

Active Directory\footnote{Active Directory\cite{ActiveDirectoryWiki}} (AD) to usługa katalogowa stworzona przez firmę Microsoft. Jest to usługa, która umożliwia zarządzanie zasobami sieciowymi, takimi jak użytkownicy, grupy, komputery, drukarki, serwery, itp. Wszystkie te zasoby są przechowywane w jednym centralnym repozytorium, które jest dostępne dla wszystkich komputerów w sieci. Active Directory jest zintegrowany z systemem operacyjnym Windows Server, co pozwala na łatwe zarządzanie zasobami sieciowymi.

\subsection{DNS}

DNS\footnote{Domain Name System\cite{DNSWiki}} (Domain Name System) to system, który przyporządkowuje nazwy domenowe do adresów IP. Dzięki DNS, użytkownicy mogą korzystać z łatwych do zapamiętania nazw domenowych, zamiast pamiętać adresy IP. DNS jest niezbędny do poprawnego funkcjonowania sieci Internet, ponieważ umożliwia przekierowanie ruchu między różnymi serwerami.

\subsection{DHCP}

DHCP\footnote{Dynamic Host Configuration Protocol\cite{DHCPwiki}} (Dynamic Host Configuration Protocol) to protokół, który umożliwia automatyczne przydzielanie adresów IP komputerom w sieci. Dzięki DHCP, administrator sieci może skonfigurować serwer DHCP, który będzie przydzielał adresy IP komputerom w sieci. DHCP jest niezbędny do poprawnego funkcjonowania sieci, ponieważ ułatwia zarządzanie adresami IP.

\subsection{iSCSI}

iSCSI\footnote{Internet Small Computer System Interface\cite{ISCSIWiki}} (Internet Small Computer System Interface) to protokół, który umożliwia przesyłanie danych między serwerem a magazynem danych za pomocą sieci IP. Dzięki iSCSI, administrator sieci może skonfigurować serwer iSCSI, który będzie udostępniał magazyn danych za pomocą sieci IP. iSCSI jest niezbędny do poprawnego funkcjonowania sieci, ponieważ umożliwia przechowywanie danych na zewnętrznym magazynie danych.

\subsection{RAID}

RAID\footnote{Redundant Array of Independent Disks\cite{RAIDwiki}} (Redundant Array of Independent Disks) to technologia, która umożliwia łączenie wielu dysków twardych w jedną logiczną jednostkę. Dzięki RAID, administrator sieci może skonfigurować macierz RAID, która będzie zapewniała redundancję danych i zwiększała wydajność systemu. RAID jest niezbędny do poprawnego funkcjonowania sieci, ponieważ umożliwia przechowywanie danych na wielu dyskach twardych.\\
W projekcie został wykorzystany RAID1\footnote{Poziomy RAID\cite{RAIDLevelsWiki}} (mirror), który zapewnia redundantność danych poprzez zapisywanie tych samych danych na dwóch dyskach twardych.


\subsection{Group Policy}

Group Policy\footnote{Group Policy\cite{GroupPolicyWiki}} to usługa, która umożliwia zarządzanie ustawieniami komputerów w sieci. Dzięki Group Policy, administrator sieci może skonfigurować ustawienia komputerów w sieci, takie jak polityki bezpieczeństwa, ustawienia systemowe, itp. Group Policy jest zintegrowany z systemem operacyjnym Windows Server, co pozwala na łatwe zarządzanie ustawieniami komputerów w sieci.

\subsection{IIS}

IIS\footnote{Internet Information Services\cite{IISWiki}} (Internet Information Services) to serwer WWW stworzony przez firmę Microsoft. Jest to serwer, który umożliwia hostowanie stron internetowych i aplikacji internetowych. IIS jest zintegrowany z systemem operacyjnym Windows Server, co pozwala na łatwe hostowanie stron internetowych i aplikacji internetowych.

\subsection{WordPress}

WordPress\footnote{WordPress\cite{WordpressWiki}} to system zarządzania treścią (CMS), który umożliwia tworzenie i zarządzanie stronami internetowymi. WordPress jest jednym z najpopularniejszych systemów zarządzania treścią na świecie, ponieważ jest łatwy w użyciu i posiada wiele funkcji. WordPress jest oparty na języku PHP i bazie danych MySQL, co pozwala na łatwe tworzenie i zarządzanie stronami internetowymi.

\subsection{MySQL}

MySQL\footnote{MySQL\cite{MySQLWiki}} to system zarządzania bazą danych (DBMS), który umożliwia przechowywanie danych w bazie danych. MySQL jest jednym z najpopularniejszych systemów zarządzania bazą danych na świecie, ponieważ jest łatwy w użyciu i posiada wiele funkcji. MySQL jest oparty na języku SQL, co pozwala na łatwe tworzenie i zarządzanie bazami danych.

\subsection{WDS}

WDS\footnote{Windows Deployment Services\cite{WDSwiki}} (Windows Deployment Services) to usługa, która umożliwia zdalne instalowanie systemu operacyjnego Windows na komputerach w sieci. Dzięki WDS, administrator sieci może skonfigurować serwer WDS, który będzie udostępniał obrazy systemu operacyjnego Windows za pomocą sieci. WDS jest niezbędny do poprawnego funkcjonowania sieci, ponieważ umożliwia zdalne instalowanie systemu operacyjnego Windows na komputerach w sieci.



