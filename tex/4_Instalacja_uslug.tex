	\newpage
\section{Procedury instalacyjne poszczególnych usług}		%4
% Procedury instalacyjne poszczególnych usług.
% (W podpunktach zamieścić polecenia dotyczące instalacji wdrażanych usług) 

% *---------------------------------------------------------------------------------------------------------------------------%

\subsection{Instalacja maszyn wirtualnych oraz ich podstawowa konfiguracja}	

Aby rozpocząć instalację maszyn wirtualnych, należy uruchomić program Oracle VM VirtualBox\footnote{Program Oracle VM Virtual Box\cite{VirtualBox}}. Pozowli on nam na stworzenie maszyn wirtualnych, które będą wykorzystywane w projekcie. Następnie należy wykonać następujące kroki:\\

Nadanie odpowiednich parametrów(pamięci operacyjnej jak i trwałej) przedstawione na rysunkach od \OznaczZdjecie[Rys.]{OZNACZ-01-SDC} do \OznaczZdjecie[Rys.]{OZNACZ-03-SDC}. Wszystkie maszyny będą instalowane w ten sposób, więc nie będziemy powtarzać tych kroków dla każdej z nich.
\fg[\textwidth]{rys/01_VM/01-SDC.png}{Instalacja maszyny wirtualnej cz.1}{OZNACZ-01-SDC}
\clearpage

\fg[\textwidth]{rys/01_VM/02-SDC.png}{Instalacja maszyny wirtualnej cz.2}{OZNACZ-02-SDC}
\clearpage

\fg[\textwidth]{rys/01_VM/03-SDC.png}{Instalacja maszyny wirtualnej cz.3}{OZNACZ-03-SDC}
\clearpage

Przeprowadzenie podstawowej instalacji systemu operacyjnego Windows Server 2022 przedstawione na rysunkach od \OznaczZdjecie[Rys.]{OZNACZ-04-SDC} do \OznaczZdjecie[Rys.]{OZNACZ-07-SDC}

\fg[\textwidth]{rys/01_VM/04-SDC.png}{Instalacja Systemu Win Server 2022 cz.1}{OZNACZ-04-SDC}
\clearpage

\fg[\textwidth]{rys/01_VM/05-SDC.png}{Instalacja Systemu Win Server 2022 cz.2}{OZNACZ-05-SDC}
\clearpage

\fg[\textwidth]{rys/01_VM/06-SDC.png}{Instalacja Systemu Win Server 2022 cz.3}{OZNACZ-06-SDC}
\clearpage

\fg[\textwidth]{rys/01_VM/07-SDC.png}{Instalacja Systemu Win Server 2022 cz.4}{OZNACZ-07-SDC}
\clearpage
\subsubsection{Serwer SDC96}
Logowanie do systemu Windows Server 2022 przedstawione na rysunku \OznaczZdjecie[Rys.]{OZNACZ-08-SDC}
\fg[\textwidth]{rys/01_VM/08-SDC.png}{Pierwsze logowanie do systemu}{OZNACZ-08-SDC}
\clearpage
Opcjonalna zmiana języka systemu na polski oraz synchronizacja zegara systemowego przedstawione na rysunkach od \OznaczZdjecie[Rys.]{OZNACZ-10-SDC} do \OznaczZdjecie[Rys.]{OZNACZ-15-SDC}
\fg[\textwidth]{rys/01_VM/10-SDC.png}{Sprawdzanie stanu sieci}{OZNACZ-10-SDC}
\clearpage

\fg[\textwidth]{rys/01_VM/11-SDC.png}{Zmiana czasu i języka cz.1}{OZNACZ-11-SDC}
\clearpage

\fg[\textwidth]{rys/01_VM/12-SDC.png}{Zmiana czasu i języka cz.2}{OZNACZ-12-SDC}
\clearpage

\fg[\textwidth]{rys/01_VM/13-SDC.png}{Zmiana czasu i języka cz.3}{OZNACZ-13-SDC}
\clearpage

\fg[\textwidth]{rys/01_VM/13-2-SDC.png}{Zmiana czasu i języka cz.4}{OZNACZ-13-2-SDC}
\clearpage


\fg[\textwidth]{rys/01_VM/14-SDC.png}{Zmiana czasu i języka cz.5}{OZNACZ-14-SDC}
\clearpage

\fg[\textwidth]{rys/01_VM/15-SDC.png}{Zmiana czasu i języka cz.6}{OZNACZ-15-SDC}
\clearpage
Zmiana nazwy serwera na SDC96 przedstawiona na rysunkach od \OznaczZdjecie[Rys.]{OZNACZ-16-SDC} do \OznaczZdjecie[Rys.]{OZNACZ-20-SDC}
\fg[\textwidth]{rys/01_VM/16-SDC.png}{Zmiana nazwy systemu cz.1}{OZNACZ-16-SDC}
\clearpage

\fg[\textwidth]{rys/01_VM/17-SDC.png}{Zmiana nazwy systemu cz.2}{OZNACZ-17-SDC}
\clearpage

\fg[\textwidth]{rys/01_VM/18-SDC.png}{Zmiana nazwy systemu cz.3}{OZNACZ-18-SDC}
\clearpage

\fg[\textwidth]{rys/01_VM/19-SDC.png}{Zmiana nazwy systemu cz.4}{OZNACZ-19-SDC}
\clearpage

\fg[\textwidth]{rys/01_VM/20-SDC.png}{Zmiana nazwy systemu cz.5}{OZNACZ-20-SDC}
\clearpage
\subsubsection{Serwer SMP96}
Aby zainstalować serwer SMP96 należy wykonać te same kroki co na serwerze SDC96, więc nie będą one powtarzane, jednakże nie zmieniamy języka, więć nie będzie potrzebna nam karta sieciowa NAT więc można od razu ustawić karte sieciową w trybie sieci wewnętrznej(każdy serwer musi mieć swoją kartę sieciową ustawioną w ten sposób). Aby to zrobić należy wykonać kroki przedstawione na rysunkach od \OznaczZdjecie[Rys.]{OZNACZ-25-SMP} do \OznaczZdjecie[Rys.]{OZNACZ-26-SMP}

\fg[\textwidth]{rys/01_VM/25-SMP.png}{Konfigracja maszyny SMP96.aronX.ad}{OZNACZ-25-SMP}

\clearpage

\fg[\textwidth]{rys/01_VM/26-SMP.png}{Zmiana sieci na sieć wewnętrzną}{OZNACZ-26-SMP}

\clearpage

Aby wykonać kolejne kroki (tj. \OznaczZdjecie[Rys.]{OZNACZ-27-SMP} - \OznaczZdjecie[Rys.]{OZNACZ-30-SMP})konieczna jest konfiguracja AD(\ref{sec:AD})

\fg[\textwidth]{rys/01_VM/27-SMP.png}{Dołączanie do domeny aronX.ad cz.1}{OZNACZ-27-SMP}
\clearpage

\fg[\textwidth]{rys/01_VM/28-SMP.png}{Dołączanie do domeny aronX.ad cz.2}{OZNACZ-28-SMP}
\clearpage

\fg[\textwidth]{rys/01_VM/29-SMP.png}{Dołączanie do domeny aronX.ad cz.3}{OZNACZ-29-SMP}
\clearpage

\fg[\textwidth]{rys/01_VM/30-SMP.png}{Dołączanie do domeny aronX.ad cz.4}{OZNACZ-30-SMP}
\clearpage

\subsubsection{Dodwanie serwerów do menadżera serwerów w SDC96}
Aby dodać serwer SMP96(bądź jakikolwiek inny) do menadżera serwerów w SDC96 należy wykonać kroki przedstawione na rysunkach od \OznaczZdjecie[Rys.]{OZNACZ-33-SDC} do \OznaczZdjecie[Rys.]{OZNACZ-34-SDC}. Pozwala to na zarządzanie serwerami z jednego miejsca(np. dodwanie funkcji bądź roli).

\fg[\textwidth]{rys/01_VM/33-SDC.png}{Dodawanie serwerów w menadżerze serwerów cz.1}{OZNACZ-33-SDC}
\clearpage

\fg[\textwidth]{rys/01_VM/34-SDC.png}{Dodawanie serwerów w menadżerze serwerów cz.2}{OZNACZ-34-SDC}
\clearpage

\subsubsection{Serwery SN1-96 i SN2-96}

Ponownie, aby dołączyć serwery SN1-96 oraz SN2-96 do domeny należy wykonać kroki \OznaczZdjecie[Rys.]{OZNACZ-38-SN1} - \OznaczZdjecie[Rys.]{OZNACZ-40-SN1} lecz konieczna jest wcześniejsza konfiguracja AD(sekcja \ref{sec:AD})

\fg[\textwidth]{rys/01_VM/38-SN1.png}{Dołączanie SN1 i SN2 do domeny cz.1}{OZNACZ-38-SN1}

\clearpage

\fg[\textwidth]{rys/01_VM/39-SN1.png}{Dołączanie SN1 i SN2 do domeny cz.2}{OZNACZ-39-SN1}

\clearpage

\fg[\textwidth]{rys/01_VM/40-SN1.png}{Dołączanie SN1 i SN2 do domeny cz.3}{OZNACZ-40-SN1}

\clearpage

% * ---------------------------------------------------------------------------------------------------------------------------%

\subsection{Konfiguracja Active Directory}
\label{sec:AD}
Aby skonfigurować Active Directory należy wykonać kroki przedstawione na rysunkach od \OznaczZdjecie[Rys.]{OZNACZ-01-AD} do \OznaczZdjecie[Rys.]{OZNACZ-21-AD}. Wszystkie kroki dotyczą serwera SDC96.\\


Na rysunkach \OznaczZdjecie[Rys.]{OZNACZ-01-AD} - \OznaczZdjecie[Rys.]{OZNACZ-03-AD} przedstawiono proces dodawania kart sieciowych do maszyny wirtualnej. Karta sieciowa nr 1 to Karta NAT. Będzie ona potrzebna, gdy będziemy pobierali aktualizacje bądź instalowali oprogramowanie zewnętrzne(np. WordPress w sekcji nr.\ref{sec:WP}), natomiast karta sieciowa nr 2 to karta sieci wewnętrznej. Będzie ona potrzebna do komunikacji z innymi serwerami czy stacjami klienckimi w sieci. 

\fg[\textwidth]{rys/04_Procedury_Instalacyjne/01_AD_DNS/01-AD.png}{Dodawanie kart sieciowych do wirtualnej maszyny cz.1}{OZNACZ-01-AD}
\clearpage

\fg[\textwidth]{rys/04_Procedury_Instalacyjne/01_AD_DNS/02-AD.png}{Dodawanie kart sieciowych do wirtualnej maszyny cz.2}{OZNACZ-02-AD}
\clearpage

\fg[\textwidth]{rys/04_Procedury_Instalacyjne/01_AD_DNS/03-AD.png}{Dodawanie kart sieciowych do wirtualnej maszyny cz.3}{OZNACZ-03-AD}
\clearpage

Na rysunkach \OznaczZdjecie[Rys.]{OZNACZ-04-AD}  oraz \OznaczZdjecie[Rys.]{OZNACZ-05-AD} przedstawiono konfiguracje adresu IP serwera SDC96. 

\fg[\textwidth]{rys/04_Procedury_Instalacyjne/01_AD_DNS/04-AD.png}{Ustawianie adresu serwera SDC96 cz.1}{OZNACZ-04-AD}
\clearpage

\fg[\textwidth]{rys/04_Procedury_Instalacyjne/01_AD_DNS/05-AD.png}{Ustawianie adresu serwera SDC96 cz.2}{OZNACZ-05-AD}
\clearpage

Na rysunkach \OznaczZdjecie[Rys.]{OZNACZ-06-AD} - \OznaczZdjecie[Rys.]{OZNACZ-13-AD} przedstawiono proces instalacji roli Active Directory Domain Services oraz DNS(chociaż narazie go nie konfigurujemy). 

\fg[\textwidth]{rys/04_Procedury_Instalacyjne/01_AD_DNS/06-Ad.png}{Dodawanie funkcji serwera AD cz.1}{OZNACZ-06-Ad}

\clearpage

\fg[\textwidth]{rys/04_Procedury_Instalacyjne/01_AD_DNS/07-AD.png}{Dodawanie funkcji serwera AD cz.2}{OZNACZ-07-AD}

\clearpage

\fg[\textwidth]{rys/04_Procedury_Instalacyjne/01_AD_DNS/08-AD.png}{Dodawanie funkcji serwera AD cz.3}{OZNACZ-08-AD}

\clearpage

\fg[\textwidth]{rys/04_Procedury_Instalacyjne/01_AD_DNS/09-AD.png}{Dodawanie funkcji serwera AD cz.4}{OZNACZ-09-AD}

\clearpage

\fg[\textwidth]{rys/04_Procedury_Instalacyjne/01_AD_DNS/10-AD.png}{Dodawanie funkcji serwera AD cz.5}{OZNACZ-10-AD}

\clearpage

\fg[\textwidth]{rys/04_Procedury_Instalacyjne/01_AD_DNS/11-AD.png}{Dodawanie funkcji serwera AD cz.6}{OZNACZ-11-AD}

\clearpage

\fg[\textwidth]{rys/04_Procedury_Instalacyjne/01_AD_DNS/12-AD.png}{Dodawanie funkcji serwera AD cz.7}{OZNACZ-12-AD}

\clearpage

\fg[\textwidth]{rys/04_Procedury_Instalacyjne/01_AD_DNS/13-AD.png}{Dodawanie funkcji serwera AD cz.8}{OZNACZ-13-AD}

\clearpage

Po kliknięciu `\textit{Promote this server to a domain controller}` zostanie wyświetlone okno konfiguracji usługi Active Directory Domain Services. Na rysunkach \OznaczZdjecie[Rys.]{OZNACZ-14-AD} - \OznaczZdjecie[Rys.]{OZNACZ-20-AD} przedstawiono proces konfiguracji domeny.

\fg[\textwidth]{rys/04_Procedury_Instalacyjne/01_AD_DNS/14-AD.png}{Konfiguracja AD cz.1}{OZNACZ-14-AD}
\clearpage

\fg[\textwidth]{rys/04_Procedury_Instalacyjne/01_AD_DNS/15-AD.png}{Konfiguracja AD cz.2}{OZNACZ-15-AD}
\clearpage

\fg[\textwidth]{rys/04_Procedury_Instalacyjne/01_AD_DNS/16-AD.png}{Konfiguracja AD cz.3}{OZNACZ-16-AD}
\clearpage

\fg[\textwidth]{rys/04_Procedury_Instalacyjne/01_AD_DNS/18-AD.png}{Konfiguracja AD cz.4}{OZNACZ-18-AD}
\clearpage

\fg[\textwidth]{rys/04_Procedury_Instalacyjne/01_AD_DNS/19-AD.png}{Konfiguracja AD cz.5}{OZNACZ-19-AD}
\clearpage

\fg[\textwidth]{rys/04_Procedury_Instalacyjne/01_AD_DNS/20-AD.png}{Konfiguracja AD cz.6}{OZNACZ-20-AD}
\clearpage

\fg[\textwidth]{rys/04_Procedury_Instalacyjne/01_AD_DNS/21-AD.png}{Działający Active Directory}{OZNACZ-21-AD}
 %TODO PRZENIEŚĆ DO TESTÓW 
\clearpage

% *---------------------------------------------------------------------------------------------------------------------------%

\subsection{Zdjęcia dotyczące DNS}

Aby skonfigurować DNS należy wykonać kroki przedstawione na rysunkach od \OznaczZdjecie[Rys.]{OZNACZ-01-DNS} do \OznaczZdjecie[Rys.]{OZNACZ-15-DNS}. Jeżeli nie widzimy opcji `DNS` w menadżerze serwera, należy zainstalować rolę DNS na serwerze SDC96 co było przedstawione wcześniej w sekcji Active Directory(sekcja nr.\ref{sec:AD}).\\


\fg[\textwidth]{rys/04_Procedury_Instalacyjne/01_AD_DNS/DNS/01-DNS.png}{Konfiguracja DNS cz.1}{OZNACZ-01-DNS}
\clearpage

Na rysunkach \OznaczZdjecie[Rys.]{OZNACZ-02-DNS} - \OznaczZdjecie[Rys.]{OZNACZ-11-DNS} dodajemy hosta dns oraz strefy przeszukiwania w przód i wstecz. 

\fg[\textwidth]{rys/04_Procedury_Instalacyjne/01_AD_DNS/DNS/02-DNS.png}{Konfiguracja DNS cz.2}{OZNACZ-02-DNS}
\clearpage

\fg[\textwidth]{rys/04_Procedury_Instalacyjne/01_AD_DNS/DNS/03-DNS.png}{Konfiguracja DNS cz.3}{OZNACZ-03-DNS}
\clearpage

\fg[\textwidth]{rys/04_Procedury_Instalacyjne/01_AD_DNS/DNS/04-DNS.png}{Konfiguracja DNS cz.4}{OZNACZ-04-DNS}
\clearpage

\fg[\textwidth]{rys/04_Procedury_Instalacyjne/01_AD_DNS/DNS/05-DNS.png}{Konfiguracja DNS cz.5}{OZNACZ-05-DNS}

\clearpage

\fg[\textwidth]{rys/04_Procedury_Instalacyjne/01_AD_DNS/DNS/06-DNS.png}{Konfiguracja DNS cz.6}{OZNACZ-06-DNS}

\clearpage

\fg[\textwidth]{rys/04_Procedury_Instalacyjne/01_AD_DNS/DNS/07-DNS.png}{Konfiguracja DNS cz.7}{OZNACZ-07-DNS}

\clearpage

\fg[\textwidth]{rys/04_Procedury_Instalacyjne/01_AD_DNS/DNS/08-DNS.png}{Konfiguracja DNS cz.8}{OZNACZ-08-DNS}

\clearpage

\fg[\textwidth]{rys/04_Procedury_Instalacyjne/01_AD_DNS/DNS/09-DNS.png}{Konfiguracja DNS cz.9}{OZNACZ-09-DNS}

\clearpage

\fg[\textwidth]{rys/04_Procedury_Instalacyjne/01_AD_DNS/DNS/10-DNS.png}{Konfiguracja DNS cz.10}{OZNACZ-10-DNS}

\clearpage

\fg[\textwidth]{rys/04_Procedury_Instalacyjne/01_AD_DNS/DNS/11-DNS.png}{Konfiguracja DNS cz.11}{OZNACZ-11-DNS}
\clearpage

Następnie, na rysunkach \OznaczZdjecie[Rys.]{OZNACZ-13-DNS} - \OznaczZdjecie[Rys.]{OZNACZ-13-DNS} dodajemy wskaźnik strefy przeszukiwania wstecz.
\fg[\textwidth]{rys/04_Procedury_Instalacyjne/01_AD_DNS/DNS/12-DNS.png}{Konfiguracja DNS cz.12}{OZNACZ-12-DNS}
\clearpage

\fg[\textwidth]{rys/04_Procedury_Instalacyjne/01_AD_DNS/DNS/13-DNS.png}{Konfiguracja DNS cz.13}{OZNACZ-13-DNS}
\clearpage

\fg[\textwidth]{rys/04_Procedury_Instalacyjne/01_AD_DNS/DNS/14-DNS.png}{Konfiguracja DNS cz.14}{OZNACZ-14-DNS}
 %TODO PRZENIEŚĆ DO TESTÓW
\clearpage

\fg[\textwidth]{rys/04_Procedury_Instalacyjne/01_AD_DNS/DNS/15-DNS.png}{Konfiguracja DNS cz.15}{OZNACZ-15-DNS}
 %TODO PRZENIEŚĆ DO TESTÓW
\clearpage

% *---------------------------------------------------------------------------------------------------------------------------%

\subsection{Zdjęcia dotyczące Grup i Kont}

Dodawanie użytkowników oraz grup do domeny zostało zautomatyzowane przy pomocy skryptu Powershell\footnote{Skrypt utworzony z pomocą strony\cite{Powershell}} przedstawionego na \OznaczKod{skrypt}. Skrypt przyjmuje plik csv\OznaczKod{dane} z spreparowanymi danymi użytkowników, grup, haseł oraz przypisanych do nich grup.

\clearpage	
\ListingFile{Skrypt Powershell dodający użytkowników}{skrypt}

\ListingFile{Fragment pliku CSV z danymi użytkowników}{dane}

\clearpage

Aby uruchomić skrypt należy otowrzyć środowisko Powershell ISE(\OznaczZdjecie[Rys.]{OZNACZ-01-Grupy-K}), `przerzucić' do niego skrypt oraz go uruchomić, tak jak na \OznaczZdjecie[Rys.]{OZNACZ-02-Grupy-K}
\fg[\textwidth]{rys/04_Procedury_Instalacyjne/02_Grupy_Konta/01.png}{Wykonanie skryptu Powershell cz.1}{OZNACZ-01-Grupy-K}

\clearpage

\fg[\textwidth]{rys/04_Procedury_Instalacyjne/02_Grupy_Konta/02.png}{Wykonanie skryptu Powershell cz.2}{OZNACZ-02-Grupy-K}

\clearpage

Dalsze kroki przedstawione na rysunkach od \OznaczZdjecie[Rys.]{OZNACZ-03-Grupy-K} do \OznaczZdjecie[Rys.]{OZNACZ-06-Grupy-K} przedstawiają dane poszczególnych użytkowników na przykładzie 1 z nich.


\fg[\textwidth]{rys/04_Procedury_Instalacyjne/02_Grupy_Konta/03.png}{Grupy organizacyjne}{OZNACZ-03-Grupy-K}

\clearpage

\fg[\textwidth]{rys/04_Procedury_Instalacyjne/02_Grupy_Konta/04.png}{właściwości użytkownika stworzonego skryptem cz.1}{OZNACZ-04-Grupy-K}

\clearpage

\fg[\textwidth]{rys/04_Procedury_Instalacyjne/02_Grupy_Konta/05.png}{właściwości użytkownika stworzonego skryptem cz.2}{OZNACZ-05-Grupy-K}

\clearpage

\fg[\textwidth]{rys/04_Procedury_Instalacyjne/02_Grupy_Konta/06.png}{właściwości użytkownika stworzonego skryptem cz.3}{OZNACZ-06-Grupy-K}

\clearpage

Następnie konieczne jest dodanie grup organizacjnch do grup funkcyjnch(tj. Domain Users, Domain Admins...) zależnie od poziomu kontroli jaki mają one mieć. Na rysunkach od \OznaczZdjecie[Rys.]{OZNACZ-07-Grupy-K} do \OznaczZdjecie[Rys.]{OZNACZ-15-Grupy-K} przedstawiono proces dodawania grup organizacyjnych do grup funkcyjnych.

\fg[\textwidth]{rys/04_Procedury_Instalacyjne/02_Grupy_Konta/07.png}{Dodawanie grup cz.1}{OZNACZ-07-Grupy-K}

\clearpage

\fg[\textwidth]{rys/04_Procedury_Instalacyjne/02_Grupy_Konta/08.png}{Dodawanie grup cz.2}{OZNACZ-08-Grupy-K}

\clearpage

\fg[\textwidth]{rys/04_Procedury_Instalacyjne/02_Grupy_Konta/09.png}{Dodawanie grup cz.3}{OZNACZ-09-Grupy-K}

\clearpage

\fg[\textwidth]{rys/04_Procedury_Instalacyjne/02_Grupy_Konta/10.png}{Dodawanie grup cz.4}{OZNACZ-10-Grupy-K}

\clearpage

\fg[\textwidth]{rys/04_Procedury_Instalacyjne/02_Grupy_Konta/11.png}{Dodawanie grup cz.5}{OZNACZ-11-Grupy-K}

\clearpage

\fg[\textwidth]{rys/04_Procedury_Instalacyjne/02_Grupy_Konta/12.png}{Dodawanie grup cz.6}{OZNACZ-12-Grupy-K}

\clearpage

\fg[\textwidth]{rys/04_Procedury_Instalacyjne/02_Grupy_Konta/13.png}{Dodawanie grup cz.7}{OZNACZ-13-Grupy-K}

\clearpage

\fg[\textwidth]{rys/04_Procedury_Instalacyjne/02_Grupy_Konta/14.png}{Dodawanie grup cz.8}{OZNACZ-14-Grupy-K}

\clearpage

\fg[\textwidth]{rys/04_Procedury_Instalacyjne/02_Grupy_Konta/15.png}{Dodawanie grup cz.9}{OZNACZ-15-Grupy-K} %TODO PRZENIEŚĆ DO TESTÓW

\clearpage

% *---------------------------------------------------------------------------------------------------------------------------%

\subsection{Zdjęcia dotyczące RAID}

Aby rozpocząć konfigurację RAID należy dodać dyski do maszyny wirtualnej SMP96. Proces ten przedstawiony jest na rysunkach od \OznaczZdjecie[Rys.]{OZNACZ-01-RAID} do \OznaczZdjecie[Rys.]{OZNACZ-05-RAID}. 

\clearpage
\fg[\textwidth]{rys/04_Procedury_Instalacyjne/03_Klaster_iSCSI/RAID/01.png}{Dodawnaie dysków do SMP96 cz.1}{OZNACZ-01-RAID}

\clearpage

\fg[\textwidth]{rys/04_Procedury_Instalacyjne/03_Klaster_iSCSI/RAID/02.png}{Dodawnaie dysków do SMP96 cz.2}{OZNACZ-02-RAID}

\clearpage

\fg[\textwidth]{rys/04_Procedury_Instalacyjne/03_Klaster_iSCSI/RAID/03.png}{Dodawnaie dysków do SMP96 cz.3}{OZNACZ-03-RAID}

\clearpage

\fg[\textwidth]{rys/04_Procedury_Instalacyjne/03_Klaster_iSCSI/RAID/04.png}{Dodawnaie dysków do SMP96 cz.4}{OZNACZ-04-RAID}

W trakcie wykonywania powyższych kroków wkradł się błąd odnośnie rozmiaru dysku, został on później poprawiony.

\clearpage

\fg[\textwidth]{rys/04_Procedury_Instalacyjne/03_Klaster_iSCSI/RAID/05.png}{Dodawnaie dysków do SMP96 cz.5}{OZNACZ-05-RAID}


\clearpage

Po udanym przydzieleniu dysków do maszyny wirtualnej SMP96, należy uruchomić ją i uruchomić program `Disk Management'. Następnie należy wykonać kroki przedstawione na rysunkach od \OznaczZdjecie[Rys.]{OZNACZ-06-RAID} do \OznaczZdjecie[Rys.]{OZNACZ-12-RAID}.

\fg[\textwidth]{rys/04_Procedury_Instalacyjne/03_Klaster_iSCSI/RAID/06.png}{Inicjalizowanie RAID1 cz.1}{OZNACZ-06-RAID}

\clearpage

\fg[\textwidth]{rys/04_Procedury_Instalacyjne/03_Klaster_iSCSI/RAID/07.png}{Inicjalizowanie RAID1 cz.2}{OZNACZ-07-RAID}

\clearpage

\fg[\textwidth]{rys/04_Procedury_Instalacyjne/03_Klaster_iSCSI/RAID/08.png}{Inicjalizowanie RAID1 cz.3}{OZNACZ-08-RAID}

\clearpage

\fg[\textwidth]{rys/04_Procedury_Instalacyjne/03_Klaster_iSCSI/RAID/09.png}{Inicjalizowanie RAID1 cz.4}{OZNACZ-09-RAID}

\clearpage

\fg[\textwidth]{rys/04_Procedury_Instalacyjne/03_Klaster_iSCSI/RAID/10.png}{Inicjalizowanie RAID1 cz.5}{OZNACZ-10-RAID}

\clearpage

\fg[\textwidth]{rys/04_Procedury_Instalacyjne/03_Klaster_iSCSI/RAID/11.png}{Inicjalizowanie RAID1 cz.6}{OZNACZ-11-RAID}

\clearpage

\fg[\textwidth]{rys/04_Procedury_Instalacyjne/03_Klaster_iSCSI/RAID/12.png}{Inicjalizowanie RAID1 cz.7}{OZNACZ-12-RAID}

\clearpage

% *---------------------------------------------------------------------------------------------------------------------------%

\subsection{Zdjęcia dotyczące iSCSI}

Aby rozpocząć konfigurację iSCSI należy dodać rolę iSCSI Target Server na serwerze SMP96. Proces ten przedstawiony jest na rysunkach od \OznaczZdjecie[Rys.]{OZNACZ-01-iSCSI} do \OznaczZdjecie[Rys.]{OZNACZ-02-iSCSI}.

\clearpage

\fg[\textwidth]{rys/04_Procedury_Instalacyjne/03_Klaster_iSCSI/iSCSI/01.png}{Dodawanie usługi iSCSI cz.1}{OZNACZ-01-iSCSI}

\clearpage

\fg[\textwidth]{rys/04_Procedury_Instalacyjne/03_Klaster_iSCSI/iSCSI/02.png}{Dodawanie usługi iSCSI cz.2}{OZNACZ-02-iSCSI}

\clearpage

Po wykonaniu powyższych kroków, należy uruchomić program `Server Manager` i przejść do `File and Storage Services` -\textgreater `iSCSI`. Następnie należy wykonać kroki przedstawione na rysunkach od \OznaczZdjecie[Rys.]{OZNACZ-03-iSCSI} do \OznaczZdjecie[Rys.]{OZNACZ-08-iSCSI}.

\fg[\textwidth]{rys/04_Procedury_Instalacyjne/03_Klaster_iSCSI/iSCSI/03.png}{Dodawanie wirtualnego dysku do iSCSI cz.1}{OZNACZ-03-iSCSI}

\clearpage

\fg[\textwidth]{rys/04_Procedury_Instalacyjne/03_Klaster_iSCSI/iSCSI/04.png}{Dodawanie wirtualnego dysku do iSCSI cz.2}{OZNACZ-04-iSCSI}

\clearpage

\fg[\textwidth]{rys/04_Procedury_Instalacyjne/03_Klaster_iSCSI/iSCSI/05.png}{Dodawanie wirtualnego dysku do iSCSI cz.3}{OZNACZ-05-iSCSI}

\clearpage

\fg[\textwidth]{rys/04_Procedury_Instalacyjne/03_Klaster_iSCSI/iSCSI/06.png}{Dodawanie wirtualnego dysku do iSCSI cz.4}{OZNACZ-06-iSCSI}

\clearpage

\fg[\textwidth]{rys/04_Procedury_Instalacyjne/03_Klaster_iSCSI/iSCSI/07.png}{Dodawanie wirtualnego dysku do iSCSI cz.5}{OZNACZ-07-iSCSI}

\clearpage

\fg[\textwidth]{rys/04_Procedury_Instalacyjne/03_Klaster_iSCSI/iSCSI/08.png}{Dodawanie wirtualnego dysku do iSCSI cz.6}{OZNACZ-08-iSCSI}

\clearpage

W międzyczasie, na serwerach SN1-96 oraz SN2-96 należy dodać i skonfigurować rolę iSCSI Initiator. Proces ten przedstawiony jest na rysunkach od \OznaczZdjecie[Rys.]{OZNACZ-09-iSCSI} do \OznaczZdjecie[Rys.]{OZNACZ-10-iSCSI}.  

\fg[\textwidth]{rys/04_Procedury_Instalacyjne/03_Klaster_iSCSI/iSCSI/09.png}{Konfiguracja iSCSI Initiator cz.1}{OZNACZ-09-iSCSI}

\clearpage


\fg[\textwidth]{rys/04_Procedury_Instalacyjne/03_Klaster_iSCSI/iSCSI/10.png}{Konfiguracja iSCSI Initiator cz.2}{OZNACZ-10-iSCSI}

W trakcie konfiguracji iSCSI Initiator na serwerach SN1-96 oraz SN2-96, należy podać adres IP serwera SMP96. Niestety doszło do pomyłki i podano błędny adres IP. Poprawny adres IP serwera SMP96 to 192.168.96.41 a nie 192.168.96.199. Problem ten został w tamtym momencie nie zauważony, więc nie został poprawiony na zdjęciach(reszta w sekcji Wniosków(Sekcja nr.\ref{sec:wnioski})). Mimo błędu, kroki nadal są poprawne i można je wykonać z podanym adresem IP.

\clearpage

Następnie na serwerze SMP96 należy kontynuuwać konfigurację iSCSI. Proces ten przedstawiony jest na rysunkach od \OznaczZdjecie[Rys.]{OZNACZ-11-iSCSI} do \OznaczZdjecie[Rys.]{OZNACZ-14-iSCSI}.

\fg[\textwidth]{rys/04_Procedury_Instalacyjne/03_Klaster_iSCSI/iSCSI/11.png}{Dodawanie wirtualnego dysku do iSCSI cz.7}{OZNACZ-11-iSCSI}

\clearpage

\fg[\textwidth]{rys/04_Procedury_Instalacyjne/03_Klaster_iSCSI/iSCSI/12.png}{Dodawanie wirtualnego dysku do iSCSI cz.8}{OZNACZ-12-iSCSI}

\clearpage

\fg[\textwidth]{rys/04_Procedury_Instalacyjne/03_Klaster_iSCSI/iSCSI/13.png}{Dodawanie wirtualnego dysku do iSCSI cz.9}{OZNACZ-13-iSCSI}

\clearpage

\fg[\textwidth]{rys/04_Procedury_Instalacyjne/03_Klaster_iSCSI/iSCSI/14.png}{Dodawanie wirtualnego dysku do iSCSI cz.10}{OZNACZ-14-iSCSI}

\clearpage

Gdy skończymy konfigurację iSCSI na serwerze SMP96, należy wrócić do serwerów SN1-96 oraz SN2-96 i dokończyć konfigurację iSCSI Initiator. Proces ten przedstawiony jest na rysunkach od \OznaczZdjecie[Rys.]{OZNACZ-15-iSCSI} do \OznaczZdjecie[Rys.]{OZNACZ-16-iSCSI}.

\fg[\textwidth]{rys/04_Procedury_Instalacyjne/03_Klaster_iSCSI/iSCSI/15.png}{Łączenie SN1 i SN2 przes iSCSI Initiator cz.1}{OZNACZ-15-iSCSI}

\clearpage

\fg[\textwidth]{rys/04_Procedury_Instalacyjne/03_Klaster_iSCSI/iSCSI/16.png}{Łączenie SN1 i SN2 przes iSCSI Initiator cz.2}{OZNACZ-16-iSCSI}

\clearpage

Później, dla wygody zarządzania dodajemy serwery SN1-96 oraz SN2-96 do menadżera serwerów w SDC96. Proces ten przedstawiony jest na rysunku \OznaczZdjecie[Rys.]{OZNACZ-17-iSCSI}. Gdy już to zrobimy instalujemy funkcje `Failover Clustering` na serwerach SN1-96 oraz SN2-96. Proces ten przedstawiony jest na rysunkach od \OznaczZdjecie[Rys.]{OZNACZ-18-iSCSI} do \OznaczZdjecie[Rys.]{OZNACZ-20-iSCSI}.

\fg[\textwidth]{rys/04_Procedury_Instalacyjne/03_Klaster_iSCSI/iSCSI/17.png}{Dodawanie SN1 i SN2 do widoku serwerów w SDC96 cz.1}{OZNACZ-17-iSCSI}

\clearpage

\fg[\textwidth]{rys/04_Procedury_Instalacyjne/03_Klaster_iSCSI/iSCSI/18.png}{Instalacja roli Failover Cluster cz.1}{OZNACZ-18-iSCSI}

\clearpage

\fg[\textwidth]{rys/04_Procedury_Instalacyjne/03_Klaster_iSCSI/iSCSI/19.png}{Instalacja roli Failover Cluster cz.2}{OZNACZ-19-iSCSI}

\clearpage

\fg[\textwidth]{rys/04_Procedury_Instalacyjne/03_Klaster_iSCSI/iSCSI/20.png}{Instalacja roli Failover Cluster cz.3}{OZNACZ-20-iSCSI}

\clearpage

Następnie, przechodzimy do managera dysków na SN1 i wykonujemy kroki przedstawione na rysunkach od \OznaczZdjecie[Rys.]{OZNACZ-21-iSCSI} do \OznaczZdjecie[Rys.]{OZNACZ-26-iSCSI}.

\fg[\textwidth]{rys/04_Procedury_Instalacyjne/03_Klaster_iSCSI/iSCSI/21.png}{Montowanie dysku na SN1 cz.1}{OZNACZ-21-iSCSI}

\clearpage

\fg[\textwidth]{rys/04_Procedury_Instalacyjne/03_Klaster_iSCSI/iSCSI/22.png}{Montowanie dysku na SN1 cz.2}{OZNACZ-22-iSCSI}

\clearpage

\fg[\textwidth]{rys/04_Procedury_Instalacyjne/03_Klaster_iSCSI/iSCSI/23.png}{Montowanie dysku na SN1 cz.3}{OZNACZ-23-iSCSI}

\clearpage

\fg[\textwidth]{rys/04_Procedury_Instalacyjne/03_Klaster_iSCSI/iSCSI/24.png}{Montowanie dysku na SN1 cz.4}{OZNACZ-24-iSCSI}

\clearpage

\fg[\textwidth]{rys/04_Procedury_Instalacyjne/03_Klaster_iSCSI/iSCSI/25.png}{Montowanie dysku na SN1 cz.5}{OZNACZ-25-iSCSI}

\clearpage

\fg[\textwidth]{rys/04_Procedury_Instalacyjne/03_Klaster_iSCSI/iSCSI/26.png}{Montowanie dysku na SN1 cz.6}{OZNACZ-26-iSCSI}

\clearpage

Następnie, przechodzimy do programu `Failover Cluster Manager` na serwerze SN1-96 i wykonujemy kroki przedstawione na rysunkach od \OznaczZdjecie[Rys.]{OZNACZ-27-iSCSI} do \OznaczZdjecie[Rys.]{OZNACZ-36-iSCSI}.

\fg[\textwidth]{rys/04_Procedury_Instalacyjne/03_Klaster_iSCSI/iSCSI/27.png}{Walidacja konfiguracji cz.1}{OZNACZ-27-iSCSI}

\clearpage

\fg[\textwidth]{rys/04_Procedury_Instalacyjne/03_Klaster_iSCSI/iSCSI/28.png}{Walidacja konfiguracji cz.2}{OZNACZ-28-iSCSI}

\clearpage

\fg[\textwidth]{rys/04_Procedury_Instalacyjne/03_Klaster_iSCSI/iSCSI/29.png}{Walidacja konfiguracji cz.3}{OZNACZ-29-iSCSI}

\clearpage

\fg[\textwidth]{rys/04_Procedury_Instalacyjne/03_Klaster_iSCSI/iSCSI/30.png}{Walidacja konfiguracji cz.4}{OZNACZ-30-iSCSI}

\clearpage

\fg[\textwidth]{rys/04_Procedury_Instalacyjne/03_Klaster_iSCSI/iSCSI/31.png}{Walidacja konfiguracji cz.5}{OZNACZ-31-iSCSI}

\clearpage

\fg[\textwidth]{rys/04_Procedury_Instalacyjne/03_Klaster_iSCSI/iSCSI/32.png}{Walidacja konfiguracji cz.6}{OZNACZ-32-iSCSI}

\clearpage

\fg[\textwidth]{rys/04_Procedury_Instalacyjne/03_Klaster_iSCSI/iSCSI/33.png}{Tworzenie Klastra cz.1}{OZNACZ-33-iSCSI}

\clearpage

\fg[\textwidth]{rys/04_Procedury_Instalacyjne/03_Klaster_iSCSI/iSCSI/34.png}{Tworzenie Klastra cz.2}{OZNACZ-34-iSCSI}

\clearpage

\fg[\textwidth]{rys/04_Procedury_Instalacyjne/03_Klaster_iSCSI/iSCSI/35.png}{Tworzenie Klastra cz.3}{OZNACZ-35-iSCSI}

\clearpage

\fg[\textwidth]{rys/04_Procedury_Instalacyjne/03_Klaster_iSCSI/iSCSI/36.png}{Tworzenie Klastra cz.4}{OZNACZ-36-iSCSI}

% Widok potwierdzenia będzie wyglądał odrobinę inaczej, z powodu błędnego adresu IP serwera SMP96. Wszystkie kroki są jednak poprawne i można je wykonać z podanym adresem IP.

\clearpage

% *---------------------------------------------------------------------------------------------------------------------------%

\subsection{Zdjęcia dotyczące FS}

Aby rozpocząć konfigurację FS należy dodać rolę File Server na serwerach SN1-96 oraz SN2-96. Proces ten przedstawiony jest na rysunkach od \OznaczZdjecie[Rys.]{OZNACZ-01-FS} do \OznaczZdjecie[Rys.]{OZNACZ-03-FS}.

\fg[\textwidth]{rys/04_Procedury_Instalacyjne/03_Klaster_iSCSI/FS/01.png}{Instalacja roli File Serwer na SN1 i SN2 cz.1}{OZNACZ-01-FS}

\clearpage

\fg[\textwidth]{rys/04_Procedury_Instalacyjne/03_Klaster_iSCSI/FS/02.png}{Instalacja roli File Serwer na SN1 i SN2 cz.2}{OZNACZ-02-FS}

\clearpage

\fg[\textwidth]{rys/04_Procedury_Instalacyjne/03_Klaster_iSCSI/FS/03.png}{Instalacja roli File Serwer na SN1 i SN2 cz.3}{OZNACZ-03-FS}

\clearpage

Następnie, na serwerze SMP96 należy dodać nowe wirtualne dyski. Proces ten przedstawiony jest na rysunkach od \OznaczZdjecie[Rys.]{OZNACZ-04-FS} do \OznaczZdjecie[Rys.]{OZNACZ-10-FS}.

\fg[\textwidth]{rys/04_Procedury_Instalacyjne/03_Klaster_iSCSI/FS/04.png}{Dodawanie wirtualnego dysku dla FS cz.1}{OZNACZ-04-FS}

\clearpage

\fg[\textwidth]{rys/04_Procedury_Instalacyjne/03_Klaster_iSCSI/FS/05.png}{Dodawanie wirtualnego dysku dla FS cz.2}{OZNACZ-05-FS}

\clearpage

\fg[\textwidth]{rys/04_Procedury_Instalacyjne/03_Klaster_iSCSI/FS/06.png}{Dodawanie wirtualnego dysku dla FS cz.3}{OZNACZ-06-FS}

\clearpage

\fg[\textwidth]{rys/04_Procedury_Instalacyjne/03_Klaster_iSCSI/FS/07.png}{Dodawanie wirtualnego dysku dla FS cz.4}{OZNACZ-07-FS}

\clearpage

\fg[\textwidth]{rys/04_Procedury_Instalacyjne/03_Klaster_iSCSI/FS/08.png}{Dodawanie wirtualnego dysku dla FS cz.5}{OZNACZ-08-FS}

\clearpage

\fg[\textwidth]{rys/04_Procedury_Instalacyjne/03_Klaster_iSCSI/FS/09.png}{Dodawanie wirtualnego dysku dla FS cz.6}{OZNACZ-09-FS}

\clearpage

\fg[\textwidth]{rys/04_Procedury_Instalacyjne/03_Klaster_iSCSI/FS/10.png}{Dodawanie wirtualnego dysku dla FS cz.7}{OZNACZ-10-FS}

\clearpage

Kontynnując, na serwerze SN1-96 należy skonfigurować nową rolę File Server. Proces ten przedstawiony jest na rysunkach od \OznaczZdjecie[Rys.]{OZNACZ-11-FS} do \OznaczZdjecie[Rys.]{OZNACZ-14-FS}.

\fg[\textwidth]{rys/04_Procedury_Instalacyjne/03_Klaster_iSCSI/FS/11.png}{Konfiguracja File Server cz.1}{OZNACZ-11-FS}

\clearpage

\fg[\textwidth]{rys/04_Procedury_Instalacyjne/03_Klaster_iSCSI/FS/12.png}{Konfiguracja File Server cz.2}{OZNACZ-12-FS}

\clearpage

\fg[\textwidth]{rys/04_Procedury_Instalacyjne/03_Klaster_iSCSI/FS/13.png}{Konfiguracja File Server cz.3}{OZNACZ-13-FS}

\clearpage

\fg[\textwidth]{rys/04_Procedury_Instalacyjne/03_Klaster_iSCSI/FS/14.png}{Konfiguracja File Server cz.4}{OZNACZ-14-FS}

\clearpage

Jako kolejny krok należy na serwerach SN1-96 oraz SN2-96 uruchomić narzędzie `iSCSI Initiator` i połączyć się z serwerem SMP96. Proces ten przedstawiony jest na rysunkach od \OznaczZdjecie[Rys.]{OZNACZ-15-FS} do \OznaczZdjecie[Rys.]{OZNACZ-17-FS}.

\fg[\textwidth]{rys/04_Procedury_Instalacyjne/03_Klaster_iSCSI/FS/15.png}{Łączenie SN1 i SN2 z SMP96 cz.1}{OZNACZ-15-FS}

\clearpage

\fg[\textwidth]{rys/04_Procedury_Instalacyjne/03_Klaster_iSCSI/FS/16.png}{Łączenie SN1 i SN2 z SMP96 cz.2}{OZNACZ-16-FS}

\clearpage

\fg[\textwidth]{rys/04_Procedury_Instalacyjne/03_Klaster_iSCSI/FS/17.png}{Łączenie SN1 i SN2 z SMP96 cz.3}{OZNACZ-17-FS}

\clearpage

Następnie, na serwerze SN1-96 należy wejść w program `Disk Management`, powinniśmy tam zobaczyć nowy dysk. Proces jego konfiguracji przedstawiony jest na rysunkach od \OznaczZdjecie[Rys.]{OZNACZ-18-FS} do \OznaczZdjecie[Rys.]{OZNACZ-21-FS}.
\fg[\textwidth]{rys/04_Procedury_Instalacyjne/03_Klaster_iSCSI/FS/18.png}{Montowanie dysku FS cz.1}{OZNACZ-18-FS}

\clearpage

\fg[\textwidth]{rys/04_Procedury_Instalacyjne/03_Klaster_iSCSI/FS/19.png}{Montowanie dysku FS cz.2}{OZNACZ-19-FS}

\clearpage

\fg[\textwidth]{rys/04_Procedury_Instalacyjne/03_Klaster_iSCSI/FS/20.png}{Montowanie dysku FS cz.3}{OZNACZ-20-FS}

\clearpage

\fg[\textwidth]{rys/04_Procedury_Instalacyjne/03_Klaster_iSCSI/FS/21.png}{Montowanie dysku FS cz.4}{OZNACZ-21-FS}

\clearpage

Potem, nadal na serwerze SN1-96, należy uruchomić program `Failover Cluster Manager` i dodać nowy dysk do klastra. Proces ten przedstawiony jest na rysunku \OznaczZdjecie[Rys.]{OZNACZ-22-FS} 

\fg[\textwidth]{rys/04_Procedury_Instalacyjne/03_Klaster_iSCSI/FS/22.png}{Dodawanie dysku do usługi FS cz.1}{OZNACZ-22-FS}

\clearpage

Teraz możemy wznowić przerwaną konfiguracje z powodu braku dysku, co widać na zdjeciach od \OznaczZdjecie[Rys.]{OZNACZ-23-FS} do \OznaczZdjecie[Rys.]{OZNACZ-26-FS}.

\fg[\textwidth]{rys/04_Procedury_Instalacyjne/03_Klaster_iSCSI/FS/23.png}{Dodawanie dysku do usługi FS cz.}{OZNACZ-23-FS}

\clearpage


\fg[\textwidth]{rys/04_Procedury_Instalacyjne/03_Klaster_iSCSI/FS/25.png}{Potwierdzenie konfiguracji}{OZNACZ-25-FS}

\clearpage

\fg[\textwidth]{rys/04_Procedury_Instalacyjne/03_Klaster_iSCSI/FS/26.png}{Podsumowanie konfiguracji}{OZNACZ-26-FS}

\clearpage

Na rysunku \OznaczZdjecie[Rys.]{OZNACZ-27-FS} przedstawiono potwierdzenie dodania nowego dysku do klastra.

\fg[\textwidth]{rys/04_Procedury_Instalacyjne/03_Klaster_iSCSI/FS/27.png}{Nowy dysk w klastrze}{OZNACZ-27-FS}

\clearpage

Następnie na serwerze SN1-96 należy dodać współdzielenie plików na nowym serwerze plików (SFS96) i skonfigurować uprawnienia. Proces ten przedstawiony jest na rysunkach od \OznaczZdjecie[Rys.]{OZNACZ-28-FS} do \OznaczZdjecie[Rys.]{OZNACZ-35-FS}.

\fg[\textwidth]{rys/04_Procedury_Instalacyjne/03_Klaster_iSCSI/FS/28.png}{Dodawanie wspólnego udziału cz.1}{OZNACZ-28-FS}

\clearpage

\fg[\textwidth]{rys/04_Procedury_Instalacyjne/03_Klaster_iSCSI/FS/29.png}{Dodawanie wspólnego udziału cz.2}{OZNACZ-29-FS}

\clearpage

\fg[\textwidth]{rys/04_Procedury_Instalacyjne/03_Klaster_iSCSI/FS/30.png}{Dodawanie wspólnego udziału cz.3}{OZNACZ-30-FS}

\clearpage

\fg[\textwidth]{rys/04_Procedury_Instalacyjne/03_Klaster_iSCSI/FS/31.png}{Dodawanie wspólnego udziału cz.4}{OZNACZ-31-FS}

\clearpage

\fg[\textwidth]{rys/04_Procedury_Instalacyjne/03_Klaster_iSCSI/FS/32.png}{Dodawanie wspólnego udziału cz.5}{OZNACZ-32-FS}

\clearpage

\fg[\textwidth]{rys/04_Procedury_Instalacyjne/03_Klaster_iSCSI/FS/33.png}{Dodawanie wspólnego udziału cz.6}{OZNACZ-33-FS}

\clearpage



\fg[\textwidth]{rys/04_Procedury_Instalacyjne/03_Klaster_iSCSI/FS/34.png}{Dodawanie wspólnego udziału cz.7}{OZNACZ-34-FS}

\clearpage

\fg[\textwidth]{rys/04_Procedury_Instalacyjne/03_Klaster_iSCSI/FS/35.png}{Dodawanie wspólnego udziału cz.8}{OZNACZ-35-FS}

\clearpage


Następnie, konfigurujemy nowe współudziały dla wszystkich grup organizacyjnych na serwerze. Proces ten przedstawiony jest na rysunkach od \OznaczZdjecie[Rys.]{OZNACZ-36-FS} do \OznaczZdjecie[Rys.]{OZNACZ-48-FS}.

\fg[\textwidth]{rys/04_Procedury_Instalacyjne/03_Klaster_iSCSI/FS/36.png}{Udziały grup organizacyjnych cz.1}{OZNACZ-36-FS}

\clearpage

\fg[\textwidth]{rys/04_Procedury_Instalacyjne/03_Klaster_iSCSI/FS/37.png}{Udziały grup organizacyjnych cz.2}{OZNACZ-37-FS}

\clearpage

\fg[\textwidth]{rys/04_Procedury_Instalacyjne/03_Klaster_iSCSI/FS/38.png}{Udziały grup organizacyjnych cz.3}{OZNACZ-38-FS}

\clearpage

\fg[\textwidth]{rys/04_Procedury_Instalacyjne/03_Klaster_iSCSI/FS/39.png}{Udziały grup organizacyjnych cz.4}{OZNACZ-39-FS}

\clearpage

\fg[\textwidth]{rys/04_Procedury_Instalacyjne/03_Klaster_iSCSI/FS/40.png}{Udziały grup organizacyjnych cz.5}{OZNACZ-40-FS}

\clearpage

\fg[\textwidth]{rys/04_Procedury_Instalacyjne/03_Klaster_iSCSI/FS/41.png}{Udziały grup organizacyjnych cz.6}{OZNACZ-41-FS}

\clearpage

\fg[\textwidth]{rys/04_Procedury_Instalacyjne/03_Klaster_iSCSI/FS/42.png}{Udziały grup organizacyjnych cz.7}{OZNACZ-42-FS}

\clearpage

\fg[\textwidth]{rys/04_Procedury_Instalacyjne/03_Klaster_iSCSI/FS/43.png}{Udziały grup organizacyjnych cz.8}{OZNACZ-43-FS}

\clearpage

\fg[\textwidth]{rys/04_Procedury_Instalacyjne/03_Klaster_iSCSI/FS/44.png}{Udziały grup organizacyjnych cz.9}{OZNACZ-44-FS}

\clearpage

\fg[\textwidth]{rys/04_Procedury_Instalacyjne/03_Klaster_iSCSI/FS/45.png}{Udziały grup organizacyjnych cz.10}{OZNACZ-45-FS}

\clearpage

\fg[\textwidth]{rys/04_Procedury_Instalacyjne/03_Klaster_iSCSI/FS/46.png}{Udziały grup organizacyjnych cz.11}{OZNACZ-46-FS}

\clearpage

\fg[\textwidth]{rys/04_Procedury_Instalacyjne/03_Klaster_iSCSI/FS/47.png}{Udziały grup organizacyjnych cz.12}{OZNACZ-47-FS}

\clearpage

\fg[\textwidth]{rys/04_Procedury_Instalacyjne/03_Klaster_iSCSI/FS/48.png}{Udziały grup organizacyjnych cz.13}{OZNACZ-48-FS}
 % TODO DODAĆ DO TESTÓW POZNIEJ
\clearpage

% *---------------------------------------------------------------------------------------------------------------------------%

\subsection{Zdjęcia dotyczące GPO DYSKI}

Aby rozpocząć konfigurację GPO DYSKI należy uruchomić program `Group Policy Management` na serwerze SDC96 oraz dodać ścieżki dla każdej grupy organizacyjnej. Proces ten przedstawiony jest na rysunkach od \OznaczZdjecie[Rys.]{OZNACZ-01-GPO-Dyski} do \OznaczZdjecie[Rys.]{OZNACZ-14-GPO-Dyski}.

\fg[\textwidth]{rys/04_Procedury_Instalacyjne/04_GPO_Mapowanie/GPO_DYSKI/01.png}{GPO cz.1}{OZNACZ-01-GPO-Dyski} 

\clearpage

\fg[\textwidth]{rys/04_Procedury_Instalacyjne/04_GPO_Mapowanie/GPO_DYSKI/02.png}{GPO cz.2}{OZNACZ-02-GPO-Dyski}

\clearpage

\fg[\textwidth]{rys/04_Procedury_Instalacyjne/04_GPO_Mapowanie/GPO_DYSKI/03.png}{GPO cz.3}{OZNACZ-03-GPO-Dyski}

\clearpage

\fg[\textwidth]{rys/04_Procedury_Instalacyjne/04_GPO_Mapowanie/GPO_DYSKI/04.png}{GPO cz.4}{OZNACZ-04-GPO-Dyski}

\clearpage

\fg[\textwidth]{rys/04_Procedury_Instalacyjne/04_GPO_Mapowanie/GPO_DYSKI/05.png}{GPO cz.5}{OZNACZ-05-GPO-Dyski}

\clearpage

\fg[\textwidth]{rys/04_Procedury_Instalacyjne/04_GPO_Mapowanie/GPO_DYSKI/06.png}{GPO cz.6}{OZNACZ-06-GPO-Dyski}
 % TODO DODAĆ DO TESTÓW POZNIEJ
\clearpage

\fg[\textwidth]{rys/04_Procedury_Instalacyjne/04_GPO_Mapowanie/GPO_DYSKI/07.png}{GPO cz.7}{OZNACZ-07-GPO-Dyski}

\clearpage

\fg[\textwidth]{rys/04_Procedury_Instalacyjne/04_GPO_Mapowanie/GPO_DYSKI/08.png}{GPO cz.8}{OZNACZ-08-GPO-Dyski}

\clearpage

\fg[\textwidth]{rys/04_Procedury_Instalacyjne/04_GPO_Mapowanie/GPO_DYSKI/09.png}{GPO cz.9}{OZNACZ-09-GPO-Dyski}

\clearpage

\fg[\textwidth]{rys/04_Procedury_Instalacyjne/04_GPO_Mapowanie/GPO_DYSKI/10.png}{GPO cz.10}{OZNACZ-10-GPO-Dyski}

\clearpage

\fg[\textwidth]{rys/04_Procedury_Instalacyjne/04_GPO_Mapowanie/GPO_DYSKI/11.png}{GPO cz.11}{OZNACZ-11-GPO-Dyski}

\clearpage

\fg[\textwidth]{rys/04_Procedury_Instalacyjne/04_GPO_Mapowanie/GPO_DYSKI/12.png}{GPO cz.12}{OZNACZ-12-GPO-Dyski}

\clearpage

\fg[\textwidth]{rys/04_Procedury_Instalacyjne/04_GPO_Mapowanie/GPO_DYSKI/13.png}{GPO cz.13}{OZNACZ-13-GPO-Dyski}

\clearpage

\fg[\textwidth]{rys/04_Procedury_Instalacyjne/04_GPO_Mapowanie/GPO_DYSKI/14.png}{GPO cz.14}{OZNACZ-14-GPO-Dyski}

\clearpage

Następnie konfigurujemy Item-level targeting dla każdej grupy organizacyjnej w celu przypisania odpowiednich dysków dla każdej grupy. Proces ten przedstawiony jest na rysunkach od \OznaczZdjecie[Rys.]{OZNACZ-15-GPO-Dyski} do \OznaczZdjecie[Rys.]{OZNACZ-31-GPO-Dyski}.


\fg[\textwidth]{rys/04_Procedury_Instalacyjne/04_GPO_Mapowanie/GPO_DYSKI/15.png}{GPO cz.15}{OZNACZ-15-GPO-Dyski}

\clearpage

\fg[\textwidth]{rys/04_Procedury_Instalacyjne/04_GPO_Mapowanie/GPO_DYSKI/16.png}{GPO cz.16}{OZNACZ-16-GPO-Dyski}

\clearpage

\fg[\textwidth]{rys/04_Procedury_Instalacyjne/04_GPO_Mapowanie/GPO_DYSKI/17.png}{GPO cz.17}{OZNACZ-17-GPO-Dyski}

\clearpage

\fg[\textwidth]{rys/04_Procedury_Instalacyjne/04_GPO_Mapowanie/GPO_DYSKI/18.png}{GPO cz.18}{OZNACZ-18-GPO-Dyski}

\clearpage

\fg[\textwidth]{rys/04_Procedury_Instalacyjne/04_GPO_Mapowanie/GPO_DYSKI/19.png}{GPO cz.19}{OZNACZ-19-GPO-Dyski}

\clearpage

\fg[\textwidth]{rys/04_Procedury_Instalacyjne/04_GPO_Mapowanie/GPO_DYSKI/20.png}{GPO cz.20}{OZNACZ-20-GPO-Dyski}

\clearpage

\fg[\textwidth]{rys/04_Procedury_Instalacyjne/04_GPO_Mapowanie/GPO_DYSKI/21.png}{GPO cz.21}{OZNACZ-21-GPO-Dyski}

\clearpage

\fg[\textwidth]{rys/04_Procedury_Instalacyjne/04_GPO_Mapowanie/GPO_DYSKI/22.png}{GPO cz.22}{OZNACZ-22-GPO-Dyski}

\clearpage

\fg[\textwidth]{rys/04_Procedury_Instalacyjne/04_GPO_Mapowanie/GPO_DYSKI/23.png}{GPO cz.23}{OZNACZ-23-GPO-Dyski}

\clearpage

\fg[\textwidth]{rys/04_Procedury_Instalacyjne/04_GPO_Mapowanie/GPO_DYSKI/24.png}{GPO cz.24}{OZNACZ-24-GPO-Dyski}

\clearpage

\fg[\textwidth]{rys/04_Procedury_Instalacyjne/04_GPO_Mapowanie/GPO_DYSKI/25.png}{GPO cz.25}{OZNACZ-25-GPO-Dyski}

\clearpage

\fg[\textwidth]{rys/04_Procedury_Instalacyjne/04_GPO_Mapowanie/GPO_DYSKI/26.png}{GPO cz.26}{OZNACZ-26-GPO-Dyski}

\clearpage

\fg[\textwidth]{rys/04_Procedury_Instalacyjne/04_GPO_Mapowanie/GPO_DYSKI/27.png}{GPO cz.27}{OZNACZ-27-GPO-Dyski}

\clearpage

\fg[\textwidth]{rys/04_Procedury_Instalacyjne/04_GPO_Mapowanie/GPO_DYSKI/28.png}{GPO cz.28}{OZNACZ-28-GPO-Dyski}

\clearpage

\fg[\textwidth]{rys/04_Procedury_Instalacyjne/04_GPO_Mapowanie/GPO_DYSKI/29.png}{GPO cz.29}{OZNACZ-29-GPO-Dyski}

\clearpage

\fg[\textwidth]{rys/04_Procedury_Instalacyjne/04_GPO_Mapowanie/GPO_DYSKI/30.png}{GPO cz.30}{OZNACZ-30-GPO-Dyski}

\clearpage

\fg[\textwidth]{rys/04_Procedury_Instalacyjne/04_GPO_Mapowanie/GPO_DYSKI/31.png}{GPO cz.31}{OZNACZ-31-GPO-Dyski}

\clearpage

% *---------------------------------------------------------------------------------------------------------------------------%

\subsection{Zdjęcia dotyczące GPO RESZTA}

Po skonfigurowaniu udziałów GPO, należy skonfigurować pozostałe ustawienia GPO, czyli zmienną środowiskową i folder Projekty96. Proces ten przedstawiony jest na rysunkach od \OznaczZdjecie[Rys.]{oznacz-01-GPO-Reszta} do \OznaczZdjecie[Rys.]{oznacz-12-GPO-Reszta}.

\fg[\textwidth]{rys/04_Procedury_Instalacyjne/04_GPO_Mapowanie/GPO_RESZTA/01.png}{Tworzenie zmiennej środowiskowej cz.1}{oznacz-01-GPO-Reszta}

\clearpage

\fg[\textwidth]{rys/04_Procedury_Instalacyjne/04_GPO_Mapowanie/GPO_RESZTA/02.png}{Tworzenie zmiennej środowiskowej cz.2}{oznacz-02-GPO-Reszta}

\clearpage

\fg[\textwidth]{rys/04_Procedury_Instalacyjne/04_GPO_Mapowanie/GPO_RESZTA/03.png}{Tworzenie zmiennej środowiskowej cz.3}{oznacz-03-GPO-Reszta}

\clearpage

\fg[\textwidth]{rys/04_Procedury_Instalacyjne/04_GPO_Mapowanie/GPO_RESZTA/04.png}{Tworzenie zmiennej środowiskowej cz.4}{oznacz-04-GPO-Reszta}

\clearpage

\fg[\textwidth]{rys/04_Procedury_Instalacyjne/04_GPO_Mapowanie/GPO_RESZTA/05.png}{Tworzenie zmiennej środowiskowej cz.5}{oznacz-05-GPO-Reszta}

\clearpage

\fg[\textwidth]{rys/04_Procedury_Instalacyjne/04_GPO_Mapowanie/GPO_RESZTA/06.png}{Tworzenie zmiennej środowiskowej cz.6}{oznacz-06-GPO-Reszta}

\clearpage

\fg[\textwidth]{rys/04_Procedury_Instalacyjne/04_GPO_Mapowanie/GPO_RESZTA/07.png}{Dodawanie folderu na projekty cz.1}{oznacz-07-GPO-Reszta}

\clearpage

\fg[\textwidth]{rys/04_Procedury_Instalacyjne/04_GPO_Mapowanie/GPO_RESZTA/08.png}{Dodawanie folderu na projekty cz.2}{oznacz-08-GPO-Reszta}

\clearpage

\fg[\textwidth]{rys/04_Procedury_Instalacyjne/04_GPO_Mapowanie/GPO_RESZTA/09.png}{Dodawanie folderu na projekty cz.3}{oznacz-09-GPO-Reszta}

\clearpage

\fg[\textwidth]{rys/04_Procedury_Instalacyjne/04_GPO_Mapowanie/GPO_RESZTA/10.png}{1Dodawanie folderu na projekty cz.4}{oznacz-10-GPO-Reszta}

\clearpage

\fg[\textwidth]{rys/04_Procedury_Instalacyjne/04_GPO_Mapowanie/GPO_RESZTA/11.png}{Dodawanie folderu na projekty cz.5}{oznacz-11-GPO-Reszta}

\clearpage

\fg[\textwidth]{rys/04_Procedury_Instalacyjne/04_GPO_Mapowanie/GPO_RESZTA/12.png}{Dodawanie folderu na projekty cz.6}{oznacz-12-GPO-Reszta}

\clearpage

% *---------------------------------------------------------------------------------------------------------------------------%

\subsection{Zdjęcia dotyczące DHCP}

Aby rozpocząć konfigurację DHCP należy uruchomić program `Server Manager` na serwerze SDC96 oraz dodać rolę DHCP do SN1-96 oraz SN2-96. Proces ten przedstawiony jest na rysunkach od \OznaczZdjecie[Rys.]{oznacz-01-DHCP} do \OznaczZdjecie[Rys.]{oznacz-04-DHCP}.



\fg[\textwidth]{rys/04_Procedury_Instalacyjne/05_DHCP/01.png}{Dodawnie roli DHCP na serwerach SN1 i SN2 cz.1}{oznacz-01-DHCP}

\clearpage

\fg[\textwidth]{rys/04_Procedury_Instalacyjne/05_DHCP/02.png}{Dodawnie roli DHCP na serwerach SN1 i SN2 cz.2}{oznacz-02-DHCP}

\clearpage

\fg[\textwidth]{rys/04_Procedury_Instalacyjne/05_DHCP/03.png}{Dodawnie roli DHCP na serwerach SN1 i SN2 cz.3}{oznacz-03-DHCP}

\clearpage

\fg[\textwidth]{rys/04_Procedury_Instalacyjne/05_DHCP/04.png}{Dodawnie roli DHCP na serwerach SN1 i SN2 cz.4}{oznacz-04-DHCP}

\clearpage

Po zakończeniu instalacji roli DHCP, należy dodać dysk do serwera SMP96. Proces ten przedstawiony jest na rysunkach od \OznaczZdjecie[Rys.]{oznacz-08-DHCP} do \OznaczZdjecie[Rys.]{oznacz-14-DHCP}.

\fg[\textwidth]{rys/04_Procedury_Instalacyjne/05_DHCP/08.png}{Dodawanie dysku DHCP cz.1}{oznacz-08-DHCP}

\clearpage

\fg[\textwidth]{rys/04_Procedury_Instalacyjne/05_DHCP/09.png}{Dodawanie dysku DHCP cz.2}{oznacz-09-DHCP}

\clearpage

\fg[6cm]{rys/04_Procedury_Instalacyjne/05_DHCP/10.png}{Dodawanie dysku DHCP cz.3}{oznacz-10-DHCP}

\clearpage

\fg[\textwidth]{rys/04_Procedury_Instalacyjne/05_DHCP/11.png}{Dodawanie dysku DHCP cz.4}{oznacz-11-DHCP}

\clearpage

\fg[\textwidth]{rys/04_Procedury_Instalacyjne/05_DHCP/12.png}{Dodawanie dysku DHCP cz.5}{oznacz-12-DHCP}

\clearpage

\fg[\textwidth]{rys/04_Procedury_Instalacyjne/05_DHCP/13.png}{Dodawanie dysku DHCP cz.6}{oznacz-13-DHCP}

\clearpage

\fg[\textwidth]{rys/04_Procedury_Instalacyjne/05_DHCP/14.png}{Dodawanie dysku DHCP cz.7}{oznacz-14-DHCP}

\clearpage

Nastepnie należy dodać ten dysk do klastra na serwerze SMP96. Proces ten przedstawiony jest na rysunkach od \OznaczZdjecie[Rys.]{oznacz-15-DHCP} do \OznaczZdjecie[Rys.]{oznacz-27-DHCP}.

\fg[\textwidth]{rys/04_Procedury_Instalacyjne/05_DHCP/15.png}{Dodawanie wirtualnego dysku DHCP do iSCSI cz.1}{oznacz-15-DHCP}

\clearpage


\fg[\textwidth]{rys/04_Procedury_Instalacyjne/05_DHCP/19.png}{Dodawanie wirtualnego dysku DHCP do iSCSI cz.2}{oznacz-19-DHCP}

\clearpage

\fg[\textwidth]{rys/04_Procedury_Instalacyjne/05_DHCP/20.png}{Dodawanie wirtualnego dysku DHCP do iSCSI cz.3}{oznacz-20-DHCP}

\clearpage

\fg[\textwidth]{rys/04_Procedury_Instalacyjne/05_DHCP/21.png}{Dodawanie wirtualnego dysku DHCP do iSCSI cz.4}{oznacz-21-DHCP}

\clearpage

\fg[\textwidth]{rys/04_Procedury_Instalacyjne/05_DHCP/22.png}{Dodawanie wirtualnego dysku DHCP do iSCSI cz.5}{oznacz-22-DHCP}

\clearpage

\fg[\textwidth]{rys/04_Procedury_Instalacyjne/05_DHCP/23.png}{Dodawanie wirtualnego dysku DHCP do iSCSI cz.6}{oznacz-23-DHCP}

\clearpage


\fg[\textwidth]{rys/04_Procedury_Instalacyjne/05_DHCP/25.png}{Dodawanie wirtualnego dysku DHCP do iSCSI cz.7}{oznacz-25-DHCP}

\clearpage

\fg[\textwidth]{rys/04_Procedury_Instalacyjne/05_DHCP/26.png}{Dodawanie wirtualnego dysku DHCP do iSCSI cz.8}{oznacz-26-DHCP}

\clearpage

\fg[\textwidth]{rys/04_Procedury_Instalacyjne/05_DHCP/27.png}{Dodawanie wirtualnego dysku DHCP do iSCSI cz.9}{oznacz-27-DHCP}

\clearpage

Teraz na serwerach SN1-96 oraz SN2-96 należy uruchomić narzędzie `iSCSI Initiator` i połączyć się z serwerem SMP96. Proces ten przedstawiony jest na rysunkach od \OznaczZdjecie[Rys.]{oznacz-28-DHCP} do \OznaczZdjecie[Rys.]{oznacz-31-DHCP}.

\fg[\textwidth]{rys/04_Procedury_Instalacyjne/05_DHCP/28.png}{Łączenie z iSCSI cz.1}{oznacz-28-DHCP}

\clearpage

\fg[\textwidth]{rys/04_Procedury_Instalacyjne/05_DHCP/29.png}{Łączenie z iSCSI cz.2}{oznacz-29-DHCP}

\clearpage

\fg[\textwidth]{rys/04_Procedury_Instalacyjne/05_DHCP/30.png}{Łączenie z iSCSI cz.3}{oznacz-30-DHCP}

\clearpage

\fg[\textwidth]{rys/04_Procedury_Instalacyjne/05_DHCP/31.png}{Łączenie z iSCSI cz.4}{oznacz-31-DHCP}

\clearpage

Potem na serwerze SN1-96 należy wejść w program `Disk Management`, powinniśmy tam zobaczyć nowy dysk. Proces jego konfiguracji przedstawiony jest na rysunkach od \OznaczZdjecie[Rys.]{oznacz-32-DHCP} do \OznaczZdjecie[Rys.]{oznacz-36-DHCP}.

\fg[\textwidth]{rys/04_Procedury_Instalacyjne/05_DHCP/32.png}{Inicjowanie dysku na SN1 cz.1}{oznacz-32-DHCP}

\clearpage

\fg[\textwidth]{rys/04_Procedury_Instalacyjne/05_DHCP/33.png}{Inicjowanie dysku na SN1 cz.2}{oznacz-33-DHCP}

\clearpage

\fg[\textwidth]{rys/04_Procedury_Instalacyjne/05_DHCP/34.png}{Inicjowanie dysku na SN1 cz.3}{oznacz-34-DHCP}

\clearpage

\fg[\textwidth]{rys/04_Procedury_Instalacyjne/05_DHCP/35.png}{Inicjowanie dysku na SN1 cz.4}{oznacz-35-DHCP}

\clearpage

\fg[\textwidth]{rys/04_Procedury_Instalacyjne/05_DHCP/36.png}{Inicjowanie dysku na SN1 cz.5}{oznacz-36-DHCP}

\clearpage

Następnie dodajemy nowy dysk do klastra na serwerze SN1-96. Proces ten przedstawiony jest na rysunkach od \OznaczZdjecie[Rys.]{oznacz-37-DHCP} do \OznaczZdjecie[Rys.]{oznacz-38-DHCP}.

\fg[\textwidth]{rys/04_Procedury_Instalacyjne/05_DHCP/37.png}{Dodawanie dysku DHCP cz.1}{oznacz-37-DHCP}

\clearpage

\fg[\textwidth]{rys/04_Procedury_Instalacyjne/05_DHCP/38.png}{Dodawanie dysku DHCP cz.2}{oznacz-38-DHCP}

\clearpage

Gdy ukończymy powyższe kroki, możemy zainstalować rolę DHCP na klastrze. Proces ten przedstawiony jest na rysunkach od \OznaczZdjecie[Rys.]{oznacz-39-DHCP} do \OznaczZdjecie[Rys.]{oznacz-43-DHCP}.

\fg[\textwidth]{rys/04_Procedury_Instalacyjne/05_DHCP/39.png}{Konfiguracja Roli DHCP na klastrze cz.1}{oznacz-39-DHCP}

\clearpage

\fg[\textwidth]{rys/04_Procedury_Instalacyjne/05_DHCP/40.png}{Konfiguracja Roli DHCP na klastrze cz.2}{oznacz-40-DHCP}

\clearpage

\fg[\textwidth]{rys/04_Procedury_Instalacyjne/05_DHCP/41.png}{Konfiguracja Roli DHCP na klastrze cz.3}{oznacz-41-DHCP}

\clearpage

\fg[\textwidth]{rys/04_Procedury_Instalacyjne/05_DHCP/42.png}{Konfiguracja Roli DHCP na klastrze cz.3}{oznacz-42-DHCP}

\clearpage

\fg[\textwidth]{rys/04_Procedury_Instalacyjne/05_DHCP/43.png}{Konfiguracja Roli DHCP na klastrze cz.4}{oznacz-43-DHCP}

\clearpage

Teraz, gdy już mamy zainstalowaną rolę DHCP na klastrze, możemy skonfigurować serwer DHCP i jego zakres(192.168.96.100-200). Proces ten przedstawiony jest na rysunkach od \OznaczZdjecie[Rys.]{oznacz-44-DHCP} do \OznaczZdjecie[Rys.]{oznacz-56-DHCP}.

\fg[\textwidth]{rys/04_Procedury_Instalacyjne/05_DHCP/44.png}{Dodawanie puli adresów cz.1}{oznacz-44-DHCP}

\clearpage

\fg[\textwidth]{rys/04_Procedury_Instalacyjne/05_DHCP/45.png}{Dodawanie puli adresów cz.2}{oznacz-45-DHCP}

\clearpage

\fg[\textwidth]{rys/04_Procedury_Instalacyjne/05_DHCP/46.png}{Dodawanie puli adresów cz.3}{oznacz-46-DHCP}

\clearpage

\fg[\textwidth]{rys/04_Procedury_Instalacyjne/05_DHCP/47.png}{Dodawanie puli adresów cz.4}{oznacz-47-DHCP}

\clearpage

\fg[\textwidth]{rys/04_Procedury_Instalacyjne/05_DHCP/48.png}{Dodawanie puli adresów cz.5}{oznacz-48-DHCP}

\clearpage

\fg[\textwidth]{rys/04_Procedury_Instalacyjne/05_DHCP/49.png}{Dodawanie puli adresów cz.6}{oznacz-49-DHCP}

\clearpage

\fg[\textwidth]{rys/04_Procedury_Instalacyjne/05_DHCP/50.png}{Dodawanie puli adresów cz.7}{oznacz-50-DHCP}

\clearpage

\fg[\textwidth]{rys/04_Procedury_Instalacyjne/05_DHCP/51.png}{Dodawanie puli adresów cz.8}{oznacz-51-DHCP}
Jako że nasza struktura nie posiada routera, adres ten można zignorować/ ustawić na jakikolwiek dostepny.
\clearpage

\fg[\textwidth]{rys/04_Procedury_Instalacyjne/05_DHCP/52.png}{Dodawanie puli adresów cz.9}{oznacz-52-DHCP}

\clearpage

\fg[\textwidth]{rys/04_Procedury_Instalacyjne/05_DHCP/53.png}{Dodawanie puli adresów cz.10}{oznacz-53-DHCP}

\clearpage

\fg[\textwidth]{rys/04_Procedury_Instalacyjne/05_DHCP/54.png}{Dodawanie puli adresów cz.Dodawanie puli adresów cz.11}{oznacz-54-DHCP}

\clearpage

%TODO DODAĆ DO TESTÓW POZNIEJ te 2 zdjecia

\fg[\textwidth]{rys/04_Procedury_Instalacyjne/05_DHCP/55.png}{Działający adres DHCP}{oznacz-55-DHCP}

\clearpage

\fg[\textwidth]{rys/04_Procedury_Instalacyjne/05_DHCP/56.png}{Testowanie komendami}{oznacz-56-DHCP}

\clearpage

% *---------------------------------------------------------------------------------------------------------------------------%

\subsection{Zdjęcia dotyczące Serwera Wydruków}

Aby rozpocząć konfigurację serwera wydruków, należy dodać wirtualną drukarke do serwera SDC96 oraz skonfigurować ją. Proces ten przedstawiony jest na rysunkach od \OznaczZdjecie[Rys.]{oznacz-01-Print} do \OznaczZdjecie[Rys.]{oznacz-07-Print}.

\clearpage

\fg[\textwidth]{rys/04_Procedury_Instalacyjne/07_Serwer_Wydrukow/01.png}{Dodawanie drukarki cz.1}{oznacz-01-Print}

\clearpage

\fg[\textwidth]{rys/04_Procedury_Instalacyjne/07_Serwer_Wydrukow/02.png}{Dodawanie drukarki cz.2}{oznacz-02-Print}

\clearpage

\fg[\textwidth]{rys/04_Procedury_Instalacyjne/07_Serwer_Wydrukow/03.png}{Dodawanie drukarki cz.3}{oznacz-03-Print}

\clearpage

\fg[\textwidth]{rys/04_Procedury_Instalacyjne/07_Serwer_Wydrukow/04.png}{Dodawanie drukarki cz.4}{oznacz-04-Print}

\clearpage

\fg[\textwidth]{rys/04_Procedury_Instalacyjne/07_Serwer_Wydrukow/05.png}{Dodawanie drukarki cz.5}{oznacz-05-Print}

\clearpage

\fg[\textwidth]{rys/04_Procedury_Instalacyjne/07_Serwer_Wydrukow/06.png}{Dodawanie drukarki cz.6}{oznacz-06-Print}

\clearpage

\fg[\textwidth]{rys/04_Procedury_Instalacyjne/07_Serwer_Wydrukow/07.png}{Dodawanie drukarki cz.7}{oznacz-07-Print}

\clearpage

Następnie należy dodać role `Print and Document Services' na serwerze SDC96. Proces ten przedstawiony jest na rysunkach od \OznaczZdjecie[Rys.]{oznacz-08-Print} do \OznaczZdjecie[Rys.]{oznacz-11-Print}. 

\fg[\textwidth]{rys/04_Procedury_Instalacyjne/07_Serwer_Wydrukow/08.png}{Instalacja usługi wydruku cz.1}{oznacz-08-Print}

\clearpage

\fg[\textwidth]{rys/04_Procedury_Instalacyjne/07_Serwer_Wydrukow/09.png}{Instalacja usługi wydruku cz.2}{oznacz-09-Print}

\clearpage

\fg[\textwidth]{rys/04_Procedury_Instalacyjne/07_Serwer_Wydrukow/10.png}{Instalacja usługi wydruku cz.3}{oznacz-10-Print}

\clearpage

\fg[\textwidth]{rys/04_Procedury_Instalacyjne/07_Serwer_Wydrukow/11.png}{Instalacja usługi wydruku cz.4}{oznacz-11-Print}

\clearpage
Teraz dodajemy drukarkę sieciową na maszynie klienckiej. Proces ten przedstawiony jest na rysunkach od \OznaczZdjecie[Rys.]{oznacz-12-Print} do \OznaczZdjecie[Rys.]{oznacz-15-Print}.	
\fg[\textwidth]{rys/04_Procedury_Instalacyjne/07_Serwer_Wydrukow/12.png}{Dodawanie drukarki na kliencie cz.1}{oznacz-12-Print}

\clearpage

Następnie możemy się połączyć z drukarką na maszynie klienckiej co widać na rysunku \OznaczZdjecie[Rys.]{oznacz-13-Print}. 

\fg[\textwidth]{rys/04_Procedury_Instalacyjne/07_Serwer_Wydrukow/13.png}{Dodawanie drukarki na kliencie cz.2}{oznacz-13-Print}

\clearpage

%TODO DODAĆ DO TESTÓW POZNIEJ 

\fg[\textwidth]{rys/04_Procedury_Instalacyjne/07_Serwer_Wydrukow/14.png}{Dodawanie drukarki na kliencie cz.3}{oznacz-14-Print}

\clearpage

\fg[\textwidth]{rys/04_Procedury_Instalacyjne/07_Serwer_Wydrukow/15.png}{Dodawanie drukarki na kliencie cz.4}{oznacz-15-Print}

\clearpage

% *---------------------------------------------------------------------------------------------------------------------------%

\subsection{Zdjęcia dotyczące Instalacji Stacji Klienckich}

Proces instalacji stacji klienckich rozpoczynamy od uruchomienia komputera. Następnie należy dodać go do domeny i aktywować DHCP. Proces ten przedstawiony jest na rysunkach od \OznaczZdjecie[Rys.]{oznacz-01-client} do \OznaczZdjecie[Rys.]{oznacz-04-client}.


\fg[\textwidth]{rys/04_Procedury_Instalacyjne/08_Instalacja_Stacji_Klienckich/01.png}{instacja stacji klienckich cz.1}{oznacz-01-client}

\clearpage

\fg[\textwidth]{rys/04_Procedury_Instalacyjne/08_Instalacja_Stacji_Klienckich/02-DNS.png}{instacja stacji klienckich cz.2}{oznacz-02-client}

\clearpage

\fg[\textwidth]{rys/04_Procedury_Instalacyjne/08_Instalacja_Stacji_Klienckich/02.png}{instacja stacji klienckich cz.3}{oznacz-02-client}

\clearpage

\fg[\textwidth]{rys/04_Procedury_Instalacyjne/08_Instalacja_Stacji_Klienckich/03-DNS.png}{instacja stacji klienckich cz.4}{oznacz-03-client}

\clearpage

\fg[\textwidth]{rys/04_Procedury_Instalacyjne/08_Instalacja_Stacji_Klienckich/04-DNS.png}{instacja stacji klienckich cz.5}{oznacz-04-client}

\clearpage

% *---------------------------------------------------------------------------------------------------------------------------%

\subsection{Zdjęcia dotyczące WDS}

Konfiguracja WDS rozpoczyna się od dodania roli WDS na serwerze SDC96. Proces ten przedstawiony jest na rysunkach od \OznaczZdjecie[Rys.]{OZNACZ-WDS-01} do \OznaczZdjecie[Rys.]{OZNACZ-WDS-04}.

\fg[\textwidth]{rys/04_Procedury_Instalacyjne/08_Instalacja_Stacji_Klienckich/WDS/01.png}{Instalacja usługi WSD cz.1}{OZNACZ-WDS-01}

\clearpage

\fg[\textwidth]{rys/04_Procedury_Instalacyjne/08_Instalacja_Stacji_Klienckich/WDS/02.png}{instacja stacji klienckich cz.2}{OZNACZ-WDS-02}

\clearpage

\fg[\textwidth]{rys/04_Procedury_Instalacyjne/08_Instalacja_Stacji_Klienckich/WDS/03.png}{instacja stacji klienckich cz.3}{OZNACZ-WDS-03}

\clearpage

\fg[\textwidth]{rys/04_Procedury_Instalacyjne/08_Instalacja_Stacji_Klienckich/WDS/04.png}{instacja stacji klienckich cz.4}{OZNACZ-WDS-04}

\clearpage

Nastepnie możemy skonfigurować WDS. Proces ten przedstawiony jest na rysunkach od \OznaczZdjecie[Rys.]{OZNACZ-WDS-05} do \OznaczZdjecie[Rys.]{OZNACZ-WDS-24}.

\fg[\textwidth]{rys/04_Procedury_Instalacyjne/08_Instalacja_Stacji_Klienckich/WDS/05.png}{Konfiguracja WDS cz.1}{OZNACZ-WDS-05}

\clearpage

\fg[\textwidth]{rys/04_Procedury_Instalacyjne/08_Instalacja_Stacji_Klienckich/WDS/06.png}{Konfiguracja WDS cz.2}{OZNACZ-WDS-06}

\clearpage

\fg[\textwidth]{rys/04_Procedury_Instalacyjne/08_Instalacja_Stacji_Klienckich/WDS/07.png}{Konfiguracja WDS cz.3}{OZNACZ-WDS-07}

\clearpage

\fg[\textwidth]{rys/04_Procedury_Instalacyjne/08_Instalacja_Stacji_Klienckich/WDS/08.png}{Konfiguracja WDS cz.4}{OZNACZ-WDS-08}

\clearpage

\fg[\textwidth]{rys/04_Procedury_Instalacyjne/08_Instalacja_Stacji_Klienckich/WDS/09.png}{Konfiguracja WDS cz.5}{OZNACZ-WDS-09}

\clearpage

\fg[\textwidth]{rys/04_Procedury_Instalacyjne/08_Instalacja_Stacji_Klienckich/WDS/10.png}{Konfiguracja WDS cz.6}{OZNACZ-WDS-10}

\clearpage



\fg[\textwidth]{rys/04_Procedury_Instalacyjne/08_Instalacja_Stacji_Klienckich/WDS/12.png}{Konfiguracja WDS cz.7}{OZNACZ-WDS-12}

\clearpage

\fg[\textwidth]{rys/04_Procedury_Instalacyjne/08_Instalacja_Stacji_Klienckich/WDS/13.png}{Konfiguracja WDS cz.8}{OZNACZ-WDS-13}

\clearpage

\fg[\textwidth]{rys/04_Procedury_Instalacyjne/08_Instalacja_Stacji_Klienckich/WDS/14.png}{Konfiguracja WDS cz.9}{OZNACZ-WDS-14}

\clearpage

\fg[\textwidth]{rys/04_Procedury_Instalacyjne/08_Instalacja_Stacji_Klienckich/WDS/11.png}{Konfiguracja WDS cz.10}{OZNACZ-WDS-11}

\clearpage

\fg[\textwidth]{rys/04_Procedury_Instalacyjne/08_Instalacja_Stacji_Klienckich/WDS/15.png}{Konfiguracja WDS cz.11}{OZNACZ-WDS-15}

\clearpage

\fg[\textwidth]{rys/04_Procedury_Instalacyjne/08_Instalacja_Stacji_Klienckich/WDS/16.png}{Konfiguracja WDS cz.12}{OZNACZ-WDS-16}

\clearpage

\fg[\textwidth]{rys/04_Procedury_Instalacyjne/08_Instalacja_Stacji_Klienckich/WDS/17.png}{Konfiguracja WDS cz.13}{OZNACZ-WDS-17}

\clearpage

\fg[\textwidth]{rys/04_Procedury_Instalacyjne/08_Instalacja_Stacji_Klienckich/WDS/18.png}{Konfiguracja WDS cz.14}{OZNACZ-WDS-18}

\clearpage

\fg[\textwidth]{rys/04_Procedury_Instalacyjne/08_Instalacja_Stacji_Klienckich/WDS/19.png}{Konfiguracja WDS cz.15}{OZNACZ-WDS-19}

\clearpage

\fg[\textwidth]{rys/04_Procedury_Instalacyjne/08_Instalacja_Stacji_Klienckich/WDS/20.png}{Konfiguracja WDS cz.16}{OZNACZ-WDS-20}

\clearpage

\fg[\textwidth]{rys/04_Procedury_Instalacyjne/08_Instalacja_Stacji_Klienckich/WDS/21.png}{Konfiguracja WDS cz.17}{OZNACZ-WDS-21}

\clearpage

\fg[\textwidth]{rys/04_Procedury_Instalacyjne/08_Instalacja_Stacji_Klienckich/WDS/22.png}{Konfiguracja WDS cz.18}{OZNACZ-WDS-22}

\clearpage

\fg[\textwidth]{rys/04_Procedury_Instalacyjne/08_Instalacja_Stacji_Klienckich/WDS/23.png}{Konfiguracja WDS cz.19}{OZNACZ-WDS-23}

\clearpage

\fg[\textwidth]{rys/04_Procedury_Instalacyjne/08_Instalacja_Stacji_Klienckich/WDS/24.png}{Konfiguracja WDS cz.20}{OZNACZ-WDS-24}

\clearpage

Nastepnie dodajemy maszyne wirtualną na której zainstalujemy system operacyjny za pomocą WDS. Proces ten przedstawiony jest na rysunkach od \OznaczZdjecie[Rys.]{OZNACZ-WDS-25} do \OznaczZdjecie[Rys.]{OZNACZ-WDS-28}.

\fg[\textwidth]{rys/04_Procedury_Instalacyjne/08_Instalacja_Stacji_Klienckich/WDS/25.png}{WDS-25}{OZNACZ-WDS-25}

\clearpage

\fg[\textwidth]{rys/04_Procedury_Instalacyjne/08_Instalacja_Stacji_Klienckich/WDS/26.png}{Dodawanie maszyny wirtualnej cz.1}{OZNACZ-WDS-26}

\clearpage

\fg[\textwidth]{rys/04_Procedury_Instalacyjne/08_Instalacja_Stacji_Klienckich/WDS/27.png}{Konfiguracja WDS cz.2}{OZNACZ-WDS-27}

\clearpage

\fg[\textwidth]{rys/04_Procedury_Instalacyjne/08_Instalacja_Stacji_Klienckich/WDS/28.png}{Konfiguracja WDS cz.3}{OZNACZ-WDS-28}

\clearpage

\fg[\textwidth]{rys/04_Procedury_Instalacyjne/08_Instalacja_Stacji_Klienckich/WDS/29.png}{test WDS cz.1}{OZNACZ-WDS-29}

\clearpage

% TODO DODAĆ DO TESTÓW POZNIEJ

\fg[\textwidth]{rys/04_Procedury_Instalacyjne/08_Instalacja_Stacji_Klienckich/WDS/30.png}{test WDS cz.2}{OZNACZ-WDS-30}

\clearpage

% *---------------------------------------------------------------------------------------------------------------------------%



\subsection{Zdjęcia dotyczące IIS}

Instalacje usługi IIS rozpoczynamy od dodania roli IIS na serwerze SDC96. Proces ten przedstawiony jest na rysunkach od \OznaczZdjecie[Rys.]{oznacz-IIS-01} do \OznaczZdjecie[Rys.]{oznacz-IIS-04}.

\fg[\textwidth]{rys/04_Procedury_Instalacyjne/06_WWW_WordPress/IIS/01.png}{Instalacja usługi IIS cz.1}{oznacz-01-IIS}

\clearpage

\fg[\textwidth]{rys/04_Procedury_Instalacyjne/06_WWW_WordPress/IIS/02.png}{Instalacja usługi IIS cz.2}{oznacz-02-IIS}

\clearpage

\fg[\textwidth]{rys/04_Procedury_Instalacyjne/06_WWW_WordPress/IIS/03.png}{Instalacja usługi IIS cz.3}{oznacz-03-IIS}
\clearpage

\fg[\textwidth]{rys/04_Procedury_Instalacyjne/06_WWW_WordPress/IIS/04.png}{Instalacja usługi IIS cz.3}{oznacz-04-IIS}
\clearpage

Tworzymy folder WWW, tak jak na rysunku \OznaczZdjecie[Rys.]{oznacz-05-IIS}.

\fg[\textwidth]{rys/04_Procedury_Instalacyjne/06_WWW_WordPress/IIS/05.png}{Tworzenie folder WWW}{oznacz-05-IIS}

\clearpage
Nastepnie przechodzimy do Tools i wybieramy `Internet Information Services (IIS) Manager` po czym konfigurujemy tą usługę. Proces ten przedstawiony jest na rysunkach od \OznaczZdjecie[Rys.]{oznacz-06-IIS} do \OznaczZdjecie[Rys.]{oznacz-11-IIS}.
\fg[5cm]{rys/04_Procedury_Instalacyjne/06_WWW_WordPress/IIS/06.png}{Konfiguracja usługi IIS cz.1}{oznacz-06-IIS}

\clearpage

\fg[\textwidth]{rys/04_Procedury_Instalacyjne/06_WWW_WordPress/IIS/07.png}{Konfiguracja usługi IIS cz.2}{oznacz-07-IIS}

\clearpage

\fg[\textwidth]{rys/04_Procedury_Instalacyjne/06_WWW_WordPress/IIS/08.png}{Konfiguracja usługi IIS cz.3}{oznacz-08-IIS}

\clearpage

\fg[\textwidth]{rys/04_Procedury_Instalacyjne/06_WWW_WordPress/IIS/09.png}{Konfiguracja usługi IIS cz.3}{oznacz-09-IIS}

\clearpage

Aby sprawdzić czy IIS działa poprawnie, należy wpisać w przeglądarce adres `www.aronx.ad`. Proces ten przedstawiony jest na rysunkach od \OznaczZdjecie[Rys.]{oznacz-10-IIS} do \OznaczZdjecie[Rys.]{oznacz-11-IIS}.
\fg[\textwidth]{rys/04_Procedury_Instalacyjne/06_WWW_WordPress/IIS/10.png}{Strona testowa}{oznacz-10-IIS}

\clearpage

\fg[\textwidth]{rys/04_Procedury_Instalacyjne/06_WWW_WordPress/IIS/11.png}{Strona firmy aronX.ad}{oznacz-11-IIS}

\clearpage

% *---------------------------------------------------------------------------------------------------------------------------%

\subsection{Zdjęcia dotyczące WordPress}
\label{sec:WP}

Aby Wordpress mogł działać poprawnie konieczny jest MySQL. Instalacje MySQL rozpoczynamy od pobrania instalki MySQL ze strony \href{https://www.mysql.com/downloads/} . Proces ten przedstawiony jest na rysunku \OznaczZdjecie[Rys.]{oznacz-01-WP}.

\fg[\textwidth]{rys/04_Procedury_Instalacyjne/06_WWW_WordPress/WordPress/01.png}{Pobieranie MySQL}{oznacz-01-WP}

\clearpage

Nastepnie upewniamy się że nasz serwer ma karte sieciową ustawioną na NAT, ponieważ konieczne będzie zainstalowanie aktualizacji. Proces ten przedstawiony jest na rysunku \OznaczZdjecie[Rys.]{oznacz-02-WP}.

\fg[\textwidth]{rys/04_Procedury_Instalacyjne/06_WWW_WordPress/WordPress/02.png}{Ustawienia karty sieciowej}{oznacz-02-WP}

\clearpage

Po zaktualizowaniu można przejść do instalacji MySQL. Proces ten przedstawiony jest na rysunkach od \OznaczZdjecie[Rys.]{oznacz-03-WP} do \OznaczZdjecie[Rys.]{oznacz-11-WP}.

\fg[\textwidth]{rys/04_Procedury_Instalacyjne/06_WWW_WordPress/WordPress/03.png}{instalacja MySQL cz.1}{oznacz-03-WP}

\clearpage

\fg[\textwidth]{rys/04_Procedury_Instalacyjne/06_WWW_WordPress/WordPress/04.png}{instalacja MySQL cz.2}{oznacz-04-WP}

\clearpage

\fg[\textwidth]{rys/04_Procedury_Instalacyjne/06_WWW_WordPress/WordPress/05.png}{instalacja MySQL cz.3}{oznacz-05-WP}

\clearpage

\fg[\textwidth]{rys/04_Procedury_Instalacyjne/06_WWW_WordPress/WordPress/06.png}{instalacja MySQL cz.4}{oznacz-06-WP}

\clearpage

\fg[\textwidth]{rys/04_Procedury_Instalacyjne/06_WWW_WordPress/WordPress/07.png}{instalacja MySQL cz.4}{oznacz-07-WP}

\clearpage

\fg[\textwidth]{rys/04_Procedury_Instalacyjne/06_WWW_WordPress/WordPress/08.png}{instalacja MySQL cz.5}{oznacz-08-WP}

\clearpage

\fg[\textwidth]{rys/04_Procedury_Instalacyjne/06_WWW_WordPress/WordPress/09.png}{instalacja MySQL cz.6}{oznacz-09-WP}

\clearpage

\fg[\textwidth]{rys/04_Procedury_Instalacyjne/06_WWW_WordPress/WordPress/10.png}{instalacja MySQL cz.7}{oznacz-10-WP}

\clearpage

\fg[\textwidth]{rys/04_Procedury_Instalacyjne/06_WWW_WordPress/WordPress/11.png}{instalacja MySQL cz.8}{oznacz-11-WP}

\clearpage


\fg[\textwidth]{rys/04_Procedury_Instalacyjne/06_WWW_WordPress/WordPress/12.png}{instalacja MySQL cz.9}{oznacz-12-WP}

\clearpage
Kolejny krok to stworzenie użytkownika i bazy danych z której będzie korzystał nasz Wordpress. Proces ten przedstawiony jest na rysunku \OznaczZdjecie[Rys.]{oznacz-13-WP}

\fg[\textwidth]{rys/04_Procedury_Instalacyjne/06_WWW_WordPress/WordPress/13.png}{Dodawanie użytkownika i bazy danyc w MySQL}{oznacz-13-WP}

\clearpage

Nastepnie pobieramy najnowszą wersje Wordpressa ze strony \href{https://pl.wordpress.org/} i przenosimy go do folderu C na serwerze. Proces ten przedstawiony jest na rysunkach od \OznaczZdjecie[Rys.]{oznacz-14-WP} do \OznaczZdjecie[Rys.]{oznacz-15-WP}.

\fg[\textwidth]{rys/04_Procedury_Instalacyjne/06_WWW_WordPress/WordPress/14.png}{Pobieranie Wordpressa}{oznacz-14-WP}

\clearpage

\fg[\textwidth]{rys/04_Procedury_Instalacyjne/06_WWW_WordPress/WordPress/15.png}{Eksportowanie Wordpressa}{oznacz-15-WP}

\clearpage

Gdy już to zrobimy dodajemy nową stronę w IIS. Proces ten przedstawiony jest na rysunku \OznaczZdjecie[Rys.]{oznacz-16-WP}.
\fg[\textwidth]{rys/04_Procedury_Instalacyjne/06_WWW_WordPress/WordPress/16.png}{Dodwanie nowej stony}{oznacz-16-WP}

\clearpage

Potem edytujemy plik konfiguracyjny wp-config.php. Proces ten przedstawiony jest na rysunku \OznaczZdjecie[Rys.]{oznacz-17-WP}.

\fg[\textwidth]{rys/04_Procedury_Instalacyjne/06_WWW_WordPress/WordPress/17.png}{edytowanie wp-config.php}{oznacz-17-WP}

\clearpage

Nastepnie musimy pobrać nową role serwera, CGI. Proces ten przedstawiony jest na rysunkach od \OznaczZdjecie[Rys.]{oznacz-18-WP} do \OznaczZdjecie[Rys.]{oznacz-21-WP}.

\fg[\textwidth]{rys/04_Procedury_Instalacyjne/06_WWW_WordPress/WordPress/18.png}{instalacja usługi CGi cz.1}{oznacz-18-WP}

\clearpage

\fg[\textwidth]{rys/04_Procedury_Instalacyjne/06_WWW_WordPress/WordPress/19.png}{instalacja usługi CGi cz.2}{oznacz-19-WP}

\clearpage

\fg[\textwidth]{rys/04_Procedury_Instalacyjne/06_WWW_WordPress/WordPress/20.png}{instalacja usługi CGi cz.3}{oznacz-20-WP}

\clearpage

\fg[\textwidth]{rys/04_Procedury_Instalacyjne/06_WWW_WordPress/WordPress/21.png}{instalacja usługi CGi cz.4}{oznacz-21-WP}

\clearpage

Pobieramy PHP ze strony \href{https://windows.php.net/download/} i przenosimy go do folderu C:\textbackslash{}php. Proces ten przedstawiony jest na rysunkach od \OznaczZdjecie[Rys.]{oznacz-22-WP} do \OznaczZdjecie[Rys.]{oznacz-23-WP}.

\fg[\textwidth]{rys/04_Procedury_Instalacyjne/06_WWW_WordPress/WordPress/22.png}{22}{oznacz-22-WP}

\clearpage

\fg[\textwidth]{rys/04_Procedury_Instalacyjne/06_WWW_WordPress/WordPress/23.png}{23}{oznacz-23-WP}

\clearpage

Nastepnie konieczne jest to aby dodać nowy wpis do zmiennej środowiskowej. Proces ten przedstawiony jest na rysunkach od \OznaczZdjecie[Rys.]{oznacz-24-WP} do \OznaczZdjecie[Rys.]{oznacz-26-WP}.

\fg[\textwidth]{rys/04_Procedury_Instalacyjne/06_WWW_WordPress/WordPress/24.png}{Nowa zmienna środowiskowa cz.1}{oznacz-24-WP}

\clearpage

\fg[\textwidth]{rys/04_Procedury_Instalacyjne/06_WWW_WordPress/WordPress/25.png}{Nowa zmienna środowiskowa cz.2}{oznacz-25-WP}

\clearpage

\fg[\textwidth]{rys/04_Procedury_Instalacyjne/06_WWW_WordPress/WordPress/26.png}{Nowa zmienna środowiskowa cz.3}{oznacz-26-WP}


\clearpage

Następnie w folderze php kopiujemy plik `php.ini-production' i zmieniamy jego nazwę na `php.ini'. Proces ten przedstawiony jest na rysunkach od \OznaczZdjecie[Rys.]{oznacz-27-WP} do \OznaczZdjecie[Rys.]{oznacz-31-WP}.

\fg[\textwidth]{rys/04_Procedury_Instalacyjne/06_WWW_WordPress/WordPress/27.png}{Zmiana nazwy pliku na php.ini}{oznacz-27-WP}

\clearpage

\fg[\textwidth]{rys/04_Procedury_Instalacyjne/06_WWW_WordPress/WordPress/28.png}{Konfiguracja php.ini cz.1}{oznacz-28-WP}

\clearpage

\fg[\textwidth]{rys/04_Procedury_Instalacyjne/06_WWW_WordPress/WordPress/29.png}{Konfiguracja php.ini cz.2}{oznacz-29-WP}

\clearpage

\fg[\textwidth]{rys/04_Procedury_Instalacyjne/06_WWW_WordPress/WordPress/30.png}{Konfiguracja php.ini cz.3}{oznacz-30-WP}

\clearpage

\fg[\textwidth]{rys/04_Procedury_Instalacyjne/06_WWW_WordPress/WordPress/31.png}{Konfiguracja php.ini cz.3}{oznacz-31-WP}
\clearpage

Potem kontynuujemy konfiguracje. Proces ten przedstawiony jest na rysunkach od \OznaczZdjecie[Rys.]{oznacz-32-WP} do \OznaczZdjecie[Rys.]{oznacz-42-WP}.

\fg[\textwidth]{rys/04_Procedury_Instalacyjne/06_WWW_WordPress/WordPress/32.png}{Dodwanie strony Wordpress cz.1}{oznacz-32-WP}
\clearpage

\fg[\textwidth]{rys/04_Procedury_Instalacyjne/06_WWW_WordPress/WordPress/33.png}{Dodwanie strony Wordpress cz.2}{oznacz-33-WP}
\clearpage

\fg[\textwidth]{rys/04_Procedury_Instalacyjne/06_WWW_WordPress/WordPress/34.png}{Dodwanie strony Wordpress cz.3}{oznacz-34-WP}
\clearpage

\fg[\textwidth]{rys/04_Procedury_Instalacyjne/06_WWW_WordPress/WordPress/35.png}{Dodwanie strony Wordpress cz.4}{oznacz-35-WP}
\clearpage

\fg[\textwidth]{rys/04_Procedury_Instalacyjne/06_WWW_WordPress/WordPress/36.png}{Dodwanie strony Wordpress cz.5}{oznacz-36-WP}
\clearpage

\fg[\textwidth]{rys/04_Procedury_Instalacyjne/06_WWW_WordPress/WordPress/37.png}{Dodwanie strony Wordpress cz.5}{oznacz-37-WP}
\clearpage

\fg[\textwidth]{rys/04_Procedury_Instalacyjne/06_WWW_WordPress/WordPress/38.png}{Zmiany w DNS}{oznacz-38-WP}
\clearpage

\fg[\textwidth]{rys/04_Procedury_Instalacyjne/06_WWW_WordPress/WordPress/39.png}{Nowy host}{oznacz-39-WP}
\clearpage

\fg[\textwidth]{rys/04_Procedury_Instalacyjne/06_WWW_WordPress/WordPress/40.png}{Nowy pointer(wordpress)}{oznacz-40-WP}
\clearpage

\fg[\textwidth]{rys/04_Procedury_Instalacyjne/06_WWW_WordPress/WordPress/41.png}{Potwierdzenie}{oznacz-41-WP}
\clearpage

Aby wordpress działał poprawnie konieczne jest zainstalowanie kolejnej aktualizacji, mianowicie Web Deploy 4.0 ze strony \href{https://www.microsoft.com/en-us/download/details.aspx?id=106070} po czym odpalamy instalke. Proces ten przedstawiony jest na rysunkach od \OznaczZdjecie[Rys.]{oznacz-42-WP} do \OznaczZdjecie[Rys.]{oznacz-47-WP}.

\fg[\textwidth]{rys/04_Procedury_Instalacyjne/06_WWW_WordPress/WordPress/42.png}{Web Deploy 4.0}{oznacz-42-WP}
\clearpage


\fg[\textwidth]{rys/04_Procedury_Instalacyjne/06_WWW_WordPress/WordPress/43.png}{Pobieranie Web Deploy 4.0}{oznacz-43-WP}
\clearpage

\fg[\textwidth]{rys/04_Procedury_Instalacyjne/06_WWW_WordPress/WordPress/44.png}{Instalacja Web Deploy 4.0 cz.1}{oznacz-44-WP}
\clearpage

\fg[\textwidth]{rys/04_Procedury_Instalacyjne/06_WWW_WordPress/WordPress/45.png}{Instalacja Web Deploy 4.0 cz.2}{oznacz-45-WP}
\clearpage

\fg[\textwidth]{rys/04_Procedury_Instalacyjne/06_WWW_WordPress/WordPress/46.png}{Instalacja Web Deploy 4.0 cz.2}{oznacz-46-WP}
\clearpage

% TODO przeniesc to testow

\fg[\textwidth]{rys/04_Procedury_Instalacyjne/06_WWW_WordPress/WordPress/47.png}{Działający WordPress}{oznacz-47-WP}
\clearpage