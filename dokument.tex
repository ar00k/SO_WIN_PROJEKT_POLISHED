%╔════════════════════════════╗
%║	  Szablon dostosował	  ║
%║	mgr inż. Dawid Kotlarski  ║
%║		  06.10.2024		  ║
%╚════════════════════════════╝
\documentclass[12pt,twoside,a4paper,openany]{article}

    % ------------------------------------------------------------------------
% PAKIETY
% ------------------------------------------------------------------------

%różne pakiety matematyczne, warto przejrzeć dokumentację, muszą być powyżej ustawień językowych.
\usepackage{mathrsfs}   %Różne symbole matematyczne opisane w katalogu ~\doc\latex\comprehensive. Zamienia \mathcal{L} ze zwykłego L na L-transformatę.
\usepackage{eucal}      %Różne symbole matematyczne.
\usepackage{amssymb}    %Różne symbole matematyczne.
\usepackage{amsmath}    %Dodatkowe funkcje matematyczne, np. polecenie \dfac{}{} skladajace ulamek w trybie wystawionym (porównaj $\dfrac{1}{2}$, a $\frac{1}{2}$).

%język polski i klawiatura
\usepackage[polish]{babel}
%\usepackage{qtimes} % czcionka Times new Roman
\usepackage[OT4]{polski}
%\usepackage[cp1250]{inputenc}                       %Strona kodowa polskich znaków.

%obsługa pdf'a
\usepackage[pdftex,usenames,dvipsnames]{color}      %Obsługa kolorów. Opcje usenames i dvipsnames wprowadzają dodatkowe nazwy kolorow.
\usepackage[pdftex,pagebackref=false,draft=false,pdfpagelabels=false,colorlinks=true,urlcolor=blue,linkcolor=black,citecolor=green,pdfstartview=FitH,pdfstartpage=1,pdfpagemode=UseOutlines,bookmarks=true,bookmarksopen=true,bookmarksopenlevel=2,bookmarksnumbered=true,pdfauthor={Dawid Kotlarski},pdftitle={Dokumentacja Projektowa},pdfsubject={},pdfkeywords={transient recovery voltage trv},unicode=true]{hyperref}   %Opcja pagebackref=true dotyczy bibliografii: pokazuje w spisie literatury numery stron, na których odwołano się do danej pozycji.

%bibliografia
%\usepackage[numbers,sort&compress]{natbib}  %Porządkuje zawartość odnośników do literatury, np. [2-4,6]. Musi być pod pdf'em, a styl bibliogfafii musi mieć nazwę z dodatkiem 'nat', np. \bibliographystyle{unsrtnat} (w kolejności cytowania).
\usepackage[
backend=biber,
style=numeric,
sorting=none
]{biblatex}
\addbibresource{bibliografia.bib}
\usepackage{hypernat}                       %Potrzebna pakietowi natbib do wspolpracy z pakietem hyperref (wazna kolejnosc: 1. hyperref, 2. natbib, 3. hypernat).

%grafika i geometria strony
\usepackage{extsizes}           %Dostepne inne rozmiary czcionek, np. 14 w poleceniu: \documentclass[14pt]{article}.
\usepackage[final]{graphicx}
\usepackage[a4paper,left=3.5cm,right=2.5cm,top=2.5cm,bottom=2.5cm]{geometry}

%strona tytułowa
\usepackage{strona_tytulowa}

%inne
\usepackage[hide]{todo}                     %Wprowadza polecenie \todo{treść}. Opcje pakietu: hide/show. Polecenie \todos ma byc na koncu dokumentu, wszystkie \todo{} po \todos sa ignorowane.
\usepackage[basic,physics]{circ}            %Wprowadza środowisko circuit do rysowania obwodów elektrycznych. Musi byc poniżej pakietow językowych.
\usepackage[sf,bf,outermarks]{titlesec}     %Troszczy się o wygląd tytułów rozdziałów (section, subsection, ...). sf oznacza czcionkę sans serif (typu arial), bf -- bold. U mnie: oddzielna linia dla naglowku paragraph. Patrz tez: tocloft -- lepiej robi format spisu tresci.
\usepackage{tocloft}                        %Troszczy się o format spisu trsci.
\usepackage{expdlist}    %Zmienia definicję środowiska description, daje większe możliwości wpływu na wygląd listy.
\usepackage{flafter}     %Wprowadza parametr [tb] do polecenia \suppressfloats[t] (polecenie to powoduje nie umieszczanie rysunkow, tabel itp. na stronach, na ktorych jest to polecenie (np. moze byc to stroma z tytulem rozdzialu, ktory chcemy zeby byl u samej gory, a nie np. pod rysunkiem)).
\usepackage{array}       %Ładniej drukuje tabelki (np. daje wiecej miejsca w komorkach -- nie są tak ścieśnione, jak bez tego pakietu).
\usepackage{listings}    %Listingi programow.
\usepackage[format=hang,labelsep=period,labelfont={bf,small},textfont=small]{caption}   %Formatuje podpisy pod rysunkami i tabelami. Parametr 'hang' powoduje wcięcie kolejnych linii podpisu na szerokosc nazwy podpisu, np. 'Rysunek 1.'.
\usepackage{appendix}    %Troszczy się o załączniki.
\usepackage{floatflt}    %Troszczy się o oblewanie rysunkow tekstem.
\usepackage{here}        %Wprowadza dodtkowy parametr umiejscowienia rysunków, tabel, itp.: H (duże). Umiejscawia obiekty ruchome dokladnie tam gdzie są w kodzie źródłowym dokumentu.
\usepackage{makeidx}     %Troszczy się o indeks (skorowidz).

%nieużywane, ale potencjalnie przydatne
\usepackage{sectsty}           %Formatuje nagłówki, np. żeby były kolorowe -- polecenie: \allsectionsfont{\color{Blue}}.
%\usepackage{version}           %Wersje dokumentu.

%============
\usepackage{longtable}			%tabelka
%============

%============
% Ustawienia listingów do kodu
%============

\usepackage{listings}
\usepackage{xcolor}

\definecolor{codegreen}{rgb}{0,0.6,0}
\definecolor{codegray}{rgb}{0.5,0.5,0.5}
\definecolor{codepurple}{rgb}{0.58,0,0.82}
\definecolor{backcolour}{rgb}{0.95,0.95,0.92}

% Definicja stylu "mystyle"
\lstdefinestyle{mystyle}{
	backgroundcolor=\color{backcolour},   
	commentstyle=\color{codegreen},
	keywordstyle=\color{blue},	%magenta
	numberstyle=\tiny\color{codegray},
	stringstyle=\color{codepurple},
	basicstyle=\ttfamily\footnotesize,
	breakatwhitespace=false,         
	breaklines=true,                 
	captionpos=b,                    
	keepspaces=true,                 
	numbers=left,                    
	numbersep=5pt,                  
	showspaces=false,                
	showstringspaces=false,
	showtabs=false,                  
	tabsize=2,
	literate=
  {á}{{\'a}}1 {é}{{\'e}}1 {í}{{\'i}}1 {ó}{{\'o}}1 {ú}{{\'u}}1
  {Á}{{\'A}}1 {É}{{\'E}}1 {Í}{{\'I}}1 {Ó}{{\'O}}1 {Ú}{{\'U}}1
  {à}{{\`a}}1 {è}{{\`e}}1 {ì}{{\`i}}1 {ò}{{\`o}}1 {ù}{{\`u}}1
  {À}{{\`A}}1 {È}{{\`E}}1 {Ì}{{\`I}}1 {Ò}{{\`O}}1 {Ù}{{\`U}}1
  {ä}{{\"a}}1 {ë}{{\"e}}1 {ï}{{\"i}}1 {ö}{{\"o}}1 {ü}{{\"u}}1
  {Ä}{{\"A}}1 {Ë}{{\"E}}1 {Ï}{{\"I}}1 {Ö}{{\"O}}1 {Ü}{{\"U}}1
  {â}{{\^a}}1 {ê}{{\^e}}1 {î}{{\^i}}1 {ô}{{\^o}}1 {û}{{\^u}}1
  {Â}{{\^A}}1 {Ê}{{\^E}}1 {Î}{{\^I}}1 {Ô}{{\^O}}1 {Û}{{\^U}}1
  {ã}{{\~a}}1 {ẽ}{{\~e}}1 {ĩ}{{\~i}}1 {õ}{{\~o}}1 {ũ}{{\~u}}1
  {Ã}{{\~A}}1 {Ẽ}{{\~E}}1 {Ĩ}{{\~I}}1 {Õ}{{\~O}}1 {Ũ}{{\~U}}1
  {œ}{{\oe}}1 {Œ}{{\OE}}1 {æ}{{\ae}}1 {Æ}{{\AE}}1 {ß}{{\ss}}1
  {ű}{{\H{u}}}1 {Ű}{{\H{U}}}1 {ő}{{\H{o}}}1 {Ő}{{\H{O}}}1
  {ç}{{\c c}}1 {Ç}{{\c C}}1 {ø}{{\o}}1 {Ø}{{\O}}1 {å}{{\r a}}1 {Å}{{\r A}}1
  {€}{{\euro}}1 {£}{{\pounds}}1 {«}{{\guillemotleft}}1
  {»}{{\guillemotright}}1 {ñ}{{\~n}}1 {Ñ}{{\~N}}1 {¿}{{?`}}1 {¡}{{!`}}1 
  {ą}{{\k{a}}}1 {ć}{{\'{c}}}1 {ę}{{\k{e}}}1 {ł}{{\l}}1 {ń}{{\'n}}1 
  {ó}{{\'o}}1 {ś}{{\'s}}1 {ź}{{\'z}}1 {ż}{{\.{z}}}1 
  {Ą}{{\k{A}}}1 {Ć}{{\'{C}}}1 {Ę}{{\k{E}}}1 {Ł}{{\L}}1 {Ń}{{\'N}}1
  {Ó}{{\'O}}1 {Ś}{{\'S}}1 {Ź}{{\'Z}}1 {Ż}{{\.{Z}}}1
}

\lstset{style=mystyle} % Deklaracja aktywnego stylu
%===========

%PAGINA GÓRNA I DOLNA
\usepackage{fancyhdr}          %Dodaje naglowki jakie się chce.
\pagestyle{fancy}
\fancyhf{}
% numery stron w paginie dolnej na srodku
\fancyfoot[C]{\footnotesize DOKUMENTACJA PROJEKTU – SYSTEMY OPERACYJNE  \\ 
\normalsize\sffamily  \thepage\ z~\pageref{LastPage}}


%\fancyhead[L]{\small\sffamily \nouppercase{\leftmark}}
\fancyhead[C]{\footnotesize \textit{AKADEMIA NAUK STOSOWANYCH W NOWYM SĄCZU}\\}

\renewcommand{\headrulewidth}{0.4pt}
\renewcommand{\footrulewidth}{0.4pt}

    % ------------------------------------------------------------------------
% USTAWIENIA
% ------------------------------------------------------------------------

% ------------------------------------------------------------------------
%   Kropki po numerach sekcji, podsekcji, itd.
%   Np. 1.2. Tytuł podrozdziału
% ------------------------------------------------------------------------
\makeatletter
    \def\numberline#1{\hb@xt@\@tempdima{#1.\hfil}}                      %kropki w spisie treści
    \renewcommand*\@seccntformat[1]{\csname the#1\endcsname.\enspace}   %kropki w treści dokumentu
\makeatother

% ------------------------------------------------------------------------
%   Numeracja równań, rysunków i tabel
%   Np.: (1.2), gdzie:
%   1 - numer sekcji, 2 - numer równania, rysunku, tabeli
%   Uwaga ogólna: o otoczeniu figure ma być najpierw \caption{}, potem \label{}, inaczej odnośnik nie działa!
% ------------------------------------------------------------------------
\makeatletter
    \@addtoreset{equation}{section} %resetuje licznik po rozpoczęciu nowej sekcji
    \renewcommand{\theequation}{{\thesection}.\@arabic\c@equation} %dodaje kropki

    \@addtoreset{figure}{section}
    \renewcommand{\thefigure}{{\thesection}.\@arabic\c@figure}

    \@addtoreset{table}{section}
    \renewcommand{\thetable}{{\thesection}.\@arabic\c@table}
\makeatother

% ------------------------------------------------------------------------
% Tablica
% ------------------------------------------------------------------------
\newenvironment{tabela}[3]
{
    \begin{table}[!htb]
    \centering
    \caption[#1]{#2}
    \vskip 9pt
    #3
}{
    \end{table}
}

% ------------------------------------------------------------------------
% Dostosowanie wyglądu pozycji listy \todos, np. zamiast 'p.' jest 'str.'
% ------------------------------------------------------------------------
\renewcommand{\todoitem}[2]{%
    \item \label{todo:\thetodo}%
    \ifx#1\todomark%
        \else\textbf{#1 }%
    \fi%
    (str.~\pageref{todopage:\thetodo})\ #2}
\renewcommand{\todoname}{Do zrobienia...}
\renewcommand{\todomark}{~uzupełnić}

% ------------------------------------------------------------------------
% Definicje
% ------------------------------------------------------------------------
\def\nonumsection#1{%
    \section*{#1}%
    \addcontentsline{toc}{section}{#1}%
    }
\def\nonumsubsection#1{%
    \subsection*{#1}%
    \addcontentsline{toc}{subsection}{#1}%
    }
\reversemarginpar %umieszcza notki po lewej stronie, czyli tam gdzie jest więcej miejsca
\def\notka#1{%
    \marginpar{\footnotesize{#1}}%
    }
\def\mathcal#1{%
    \mathscr{#1}%
    }
\newcommand{\atp}{ATP/EMTP} % Inaczej: \def\atp{ATP/EMTP}

% ------------------------------------------------------------------------
% Inne
% ------------------------------------------------------------------------
\frenchspacing                      
\hyphenation{ATP/-EMTP}             %dzielenie wyrazu w danym miejscu
\setlength{\parskip}{3pt}           %odstęp pomiędzy akapitami
\linespread{1.3}                    %odstęp pomiędzy liniami (interlinia)
\setcounter{tocdepth}{4}            %uwzględnianie w spisie treści czterech poziomów sekcji
\setcounter{secnumdepth}{4}         %numerowanie do czwartego poziomu sekcji 
\titleformat{\paragraph}[hang]      %wygląd nagłówków
{\normalfont\sffamily\bfseries}{\theparagraph}{1em}{}

%komenda do łatwiejszego wstawiania zdjęć
\newcommand*{\fg}[4][\textwidth]{
    \begin{figure}[!htb]
        \begin{center}
            \includegraphics[width=#1]{#2}
            \caption{#3}
            \label{rys:#4}
        \end{center}
    \end{figure}
}

\newcommand*{\Oznacz}[2]{
\ref{#1:#2} (s. \pageref{#1:#2})
}

\newcommand*{\OznaczZdjecie}[2][Rysunek]{
#1 \Oznacz{rys}{#2}
}
    
\newcommand*{\OznaczKod}[1]{
\Oznacz{lst}{#1}
}

\newcommand*{\ListingFile}[2]{
    \lstinputlisting[caption=#1, label={lst:#2}, language=C++]{kod/#2.txt}
}


    %polecenia zdefiniowane w pakiecie strona_tytulowa.sty
    \title{Zaprojektować i wdrożyć system informatyczny na
    potrzeby przedsiębiorstwa}		%...Wpisać nazwę projektu...
    \author{Arkadiusz Ryczek 33096}
    \authorI{}
    \authorII{}		%jeśli są dwie osoby w projekcie to zostawiamy:    \authorII{}
		
	\uczelnia{AKADEMIA NAUK STOSOWANYCH \\W NOWYM SĄCZU}
    \instytut{Wydział Nauk Inżynieryjnych}
    \kierunek{Katedra Informatyki}
    \praca{DOKUMENTACJA PROJEKTOWA}
    \przedmiot{SYSTEMY OPERACYJNE}
    \prowadzacy{mgr inż. Jan Kozieński}
    \rok{2025}


%definicja składni mikrotik
\usepackage{fancyvrb}
\DefineVerbatimEnvironment{MT}{Verbatim}%
{commandchars=\+\[\],fontsize=\small,formatcom=\color{red},frame=lines,baselinestretch=1,} 
\let\mt\verb 
%zakonczenie definicji składni mikrotik

\usepackage{fancyhdr}    %biblioteka do nagłówka i stopki

			
\begin{document}
   
    \renewcommand{\figurename}{Rys.}    %musi byc pod \begin{document}, bo w~tym miejscu pakiet 'babel' narzuca swoje ustawienia
    \renewcommand{\tablename}{Tab.}     %j.w.
    \thispagestyle{empty}               %na tej stronie: brak numeru
    \stronatytulowa                     %strona tytułowa tworzona przez pakiet strona_tytulowa.tex
 
 \pagestyle{fancy}

    \newpage

    %formatowanie spisu treści i~nagłówków
    \renewcommand{\cftbeforesecskip}{8pt}
    \renewcommand{\cftsecafterpnum}{\vskip 8pt}
    \renewcommand{\cftparskip}{3pt}
    \renewcommand{\cfttoctitlefont}{\Large\bfseries\sffamily}
    \renewcommand{\cftsecfont}{\bfseries\sffamily}
    \renewcommand{\cftsubsecfont}{\sffamily}
    \renewcommand{\cftsubsubsecfont}{\sffamily}
    \renewcommand{\cftparafont}{\sffamily}
    %koniec formatowania spisu treści i nagłówków
     
    \tableofcontents    %spis treści
    \thispagestyle{fancy}
    \newpage

    
    \newpage

    
%%%%%%%%%%%%%%%%%%% treść główna dokumentu %%%%%%%%%%%%%%%%%%%%%%%%%

   	\newpage
\section{Założenia projektowe – wymagania}		%1

%ctrl+alt+j - przeniesienie z kodu do pdf


Niniejszy projekt ma na celu stworzenie sieci komputerowej dla przedsiębiorstwa, które będzie składać się z następujących elementów:
\begin{enumerate}
    \item Utworzenie własnej domeny AD według formatu \texttt{firma.ad}, gdzie \texttt{firma} to nazwa firmy, dla której przygotowujemy projekt.
    \item Autoryzacja pracownika przy użyciu imiennego konta działającego na wszystkich komputerach w sieci firmowej. Login zgodny ze schematem: \texttt{imie.nazwisko}. Utworzyć minimum po trzy konta dla każdego wydziału. Tworzenie grup oraz kont w domenie wykonywać przez skrypt, który odczyta dane grupy i kont z pliku i utworzy je w domenie AD.
    \item Pracownicy powinni należeć do grupy globalnej odpowiedniej dla wydziału, w którym pracują (przewidzieć 5 przykładowych wydziałów, np. kadry, płace, gospodarczy, marketing, itp. wg własnego uznania). W przedsiębiorstwie przewidziano wydział informatyczny, którego pracownicy mają w pełni administrować domeną przedsiębiorstwa. Tworzenie grup oraz kont w domenie wykonywać przez skrypt, który odczyta dane grupy i kont z pliku i utworzy je w domenie AD.
    \item Pracownicy powinni korzystać z zasobów sieciowych o nazwach: \texttt{wspolny} oraz zasób wydziałowy (oddzielny zasób dla każdego wydziału). Zasoby powinny być udostępnione poprzez klaster pracy awaryjnej, który powinien korzystać z przestrzeni dyskowej (macierzy RAID-1) udostępnionej poprzez iSCSI o przestrzeni wypadkowej 30 GB. Jeśli pozwalają zasoby sprzętowe, to programową macierz RAID-1 oraz iSCSI Target Serwer można zainstalować na oddzielnym serwerze o nazwie \texttt{SMPXX.firma.ad}.
    \item Zasób \texttt{wspolny} ma być mapowany użytkownikowi jako dysk (patrz: założenia projektowe), natomiast zasób wydziałowy jako jeden z dysków (patrz: założenia projektowe) adekwatnie do grupy wydziałowej, w której znajduje się pracownik. Mechanizm mapowania automatyczny przy użyciu polis GPO.
    \item System ma umożliwić instalację stacji klienckich z obrazu udostępnionego na serwerze.
    \item Pracownicy powinni mieć dostęp do drukarek sieciowych udostępnianych poprzez serwer wydruków. Serwer wydruków można zainstalować na oddzielnym serwerze, np. o nazwie \texttt{SPRXX.firma.ad}, lub w przypadku małych zasobów RAM, na serwerze \texttt{SDC}. Dostęp do drukarki sieciowej: \texttt{\textbackslash\textbackslash SPRXX-firma.ad\textbackslash nazwa-drukarki}.
    \item Konfiguracja stacji klienckich w sposób automatyczny.
    \item Na klastrze pracy awaryjnej należy wdrożyć serwer DHCP.
    \item Po udanym wdrożeniu serwera DHCP należy wyłączyć serwer DHCP pracujący na \texttt{SDC}.
    \item Wdrożyć serwer WWW wraz z firmową stroną (nie domyślną) dostępną pod adresem \texttt{www.firma.ad}.
    \item Wdrożyć WordPress współpracujący z usługą IIS Windows Serwera 2022.
\end{enumerate}


   \newpage
\section{Opis użytych technologii}		%2
%(W podpunktach dokonać krótkiej charakterystyki użytych technologii ) 

\subsection{Active Directory} 

Active Directory\footnote{Active Directory\cite{ActiveDirectoryWiki}} (AD) to usługa katalogowa stworzona przez firmę Microsoft. Jest to usługa, która umożliwia zarządzanie zasobami sieciowymi, takimi jak użytkownicy, grupy, komputery, drukarki, serwery, itp. Wszystkie te zasoby są przechowywane w jednym centralnym repozytorium, które jest dostępne dla wszystkich komputerów w sieci. Active Directory jest zintegrowany z systemem operacyjnym Windows Server, co pozwala na łatwe zarządzanie zasobami sieciowymi.

\subsection{DNS}

DNS\footnote{Domain Name System\cite{DNSWiki}} (Domain Name System) to system, który przyporządkowuje nazwy domenowe do adresów IP. Dzięki DNS, użytkownicy mogą korzystać z łatwych do zapamiętania nazw domenowych, zamiast pamiętać adresy IP. DNS jest niezbędny do poprawnego funkcjonowania sieci Internet, ponieważ umożliwia przekierowanie ruchu między różnymi serwerami.

\subsection{DHCP}

DHCP\footnote{Dynamic Host Configuration Protocol\cite{DHCPwiki}} (Dynamic Host Configuration Protocol) to protokół, który umożliwia automatyczne przydzielanie adresów IP komputerom w sieci. Dzięki DHCP, administrator sieci może skonfigurować serwer DHCP, który będzie przydzielał adresy IP komputerom w sieci. DHCP jest niezbędny do poprawnego funkcjonowania sieci, ponieważ ułatwia zarządzanie adresami IP.

\subsection{iSCSI}

iSCSI\footnote{Internet Small Computer System Interface\cite{ISCSIWiki}} (Internet Small Computer System Interface) to protokół, który umożliwia przesyłanie danych między serwerem a magazynem danych za pomocą sieci IP. Dzięki iSCSI, administrator sieci może skonfigurować serwer iSCSI, który będzie udostępniał magazyn danych za pomocą sieci IP. iSCSI jest niezbędny do poprawnego funkcjonowania sieci, ponieważ umożliwia przechowywanie danych na zewnętrznym magazynie danych.

\subsection{RAID}

RAID\footnote{Redundant Array of Independent Disks\cite{RAIDwiki}} (Redundant Array of Independent Disks) to technologia, która umożliwia łączenie wielu dysków twardych w jedną logiczną jednostkę. Dzięki RAID, administrator sieci może skonfigurować macierz RAID, która będzie zapewniała redundancję danych i zwiększała wydajność systemu. RAID jest niezbędny do poprawnego funkcjonowania sieci, ponieważ umożliwia przechowywanie danych na wielu dyskach twardych.\\
W projekcie został wykorzystany RAID1\footnote{Poziomy RAID\cite{RAIDLevelsWiki}} (mirror), który zapewnia redundantność danych poprzez zapisywanie tych samych danych na dwóch dyskach twardych.


\subsection{Group Policy}

Group Policy\footnote{Group Policy\cite{GroupPolicyWiki}} to usługa, która umożliwia zarządzanie ustawieniami komputerów w sieci. Dzięki Group Policy, administrator sieci może skonfigurować ustawienia komputerów w sieci, takie jak polityki bezpieczeństwa, ustawienia systemowe, itp. Group Policy jest zintegrowany z systemem operacyjnym Windows Server, co pozwala na łatwe zarządzanie ustawieniami komputerów w sieci.

\subsection{IIS}

IIS\footnote{Internet Information Services\cite{IISWiki}} (Internet Information Services) to serwer WWW stworzony przez firmę Microsoft. Jest to serwer, który umożliwia hostowanie stron internetowych i aplikacji internetowych. IIS jest zintegrowany z systemem operacyjnym Windows Server, co pozwala na łatwe hostowanie stron internetowych i aplikacji internetowych.

\subsection{WordPress}

WordPress\footnote{WordPress\cite{WordpressWiki}} to system zarządzania treścią (CMS), który umożliwia tworzenie i zarządzanie stronami internetowymi. WordPress jest jednym z najpopularniejszych systemów zarządzania treścią na świecie, ponieważ jest łatwy w użyciu i posiada wiele funkcji. WordPress jest oparty na języku PHP i bazie danych MySQL, co pozwala na łatwe tworzenie i zarządzanie stronami internetowymi.

\subsection{MySQL}

MySQL\footnote{MySQL\cite{MySQLWiki}} to system zarządzania bazą danych (DBMS), który umożliwia przechowywanie danych w bazie danych. MySQL jest jednym z najpopularniejszych systemów zarządzania bazą danych na świecie, ponieważ jest łatwy w użyciu i posiada wiele funkcji. MySQL jest oparty na języku SQL, co pozwala na łatwe tworzenie i zarządzanie bazami danych.

\subsection{WDS}

WDS\footnote{Windows Deployment Services\cite{WDSwiki}} (Windows Deployment Services) to usługa, która umożliwia zdalne instalowanie systemu operacyjnego Windows na komputerach w sieci. Dzięki WDS, administrator sieci może skonfigurować serwer WDS, który będzie udostępniał obrazy systemu operacyjnego Windows za pomocą sieci. WDS jest niezbędny do poprawnego funkcjonowania sieci, ponieważ umożliwia zdalne instalowanie systemu operacyjnego Windows na komputerach w sieci.




   	\newpage
\section{Schemat logiczny}		%3
% (Ma zawierać aktualne nazewnictwo i adresację IP.)
   	\newpage
\section{Procedury instalacyjne poszczególnych usług}		%4
% Procedury instalacyjne poszczególnych usług.
% (W podpunktach zamieścić polecenia dotyczące instalacji wdrażanych usług) 

% *---------------------------------------------------------------------------------------------------------------------------%

\subsection{Instalacja maszyn wirtualnych oraz ich podstawowa konfiguracja}	

Aby rozpocząć instalację maszyn wirtualnych, należy uruchomić program Oracle VM VirtualBox\footnote{Program Oracle VM Virtual Box\cite{VirtualBox}}. Pozowli on nam na stworzenie maszyn wirtualnych, które będą wykorzystywane w projekcie. Następnie należy wykonać następujące kroki:\\

Nadanie odpowiednich parametrów(pamięci operacyjnej jak i trwałej) przedstawione na rysunkach od \OznaczZdjecie[Rys.]{OZNACZ-01-SDC} do \OznaczZdjecie[Rys.]{OZNACZ-03-SDC}. Wszystkie maszyny będą instalowane w ten sposób, więc nie będziemy powtarzać tych kroków dla każdej z nich.
\fg[\textwidth]{rys/01_VM/01-SDC.png}{01-SDC}{OZNACZ-01-SDC}
\clearpage

\fg[\textwidth]{rys/01_VM/02-SDC.png}{02-SDC}{OZNACZ-02-SDC}
\clearpage

\fg[\textwidth]{rys/01_VM/03-SDC.png}{03-SDC}{OZNACZ-03-SDC}
\clearpage

Przeprowadzenie podstawowej instalacji systemu operacyjnego Windows Server 2022 przedstawione na rysunkach od \OznaczZdjecie[Rys.]{OZNACZ-04-SDC} do \OznaczZdjecie[Rys.]{OZNACZ-07-SDC}

\fg[\textwidth]{rys/01_VM/04-SDC.png}{04-SDC}{OZNACZ-04-SDC}
\clearpage

\fg[\textwidth]{rys/01_VM/05-SDC.png}{05-SDC}{OZNACZ-05-SDC}
\clearpage

\fg[\textwidth]{rys/01_VM/06-SDC.png}{06-SDC}{OZNACZ-06-SDC}
\clearpage

\fg[\textwidth]{rys/01_VM/07-SDC.png}{07-SDC}{OZNACZ-07-SDC}
\clearpage
\subsubsection{Serwer SDC96}
Logowanie do systemu Windows Server 2022 przedstawione na rysunku \OznaczZdjecie[Rys.]{OZNACZ-08-SDC}
\fg[\textwidth]{rys/01_VM/08-SDC.png}{08-SDC}{OZNACZ-08-SDC}
\clearpage
Opcjonalna zmiana języka systemu na polski oraz synchronizacja zegara systemowego przedstawione na rysunkach od \OznaczZdjecie[Rys.]{OZNACZ-10-SDC} do \OznaczZdjecie[Rys.]{OZNACZ-15-SDC}
\fg[\textwidth]{rys/01_VM/10-SDC.png}{10-SDC}{OZNACZ-10-SDC}
\clearpage

\fg[\textwidth]{rys/01_VM/11-SDC.png}{11-SDC}{OZNACZ-11-SDC}
\clearpage

\fg[\textwidth]{rys/01_VM/12-SDC.png}{12-SDC}{OZNACZ-12-SDC}
\clearpage

\fg[\textwidth]{rys/01_VM/13-SDC.png}{13-SDC}{OZNACZ-13-SDC}
\clearpage

\fg[\textwidth]{rys/01_VM/13-2-SDC.png}{13-2-SDC}{OZNACZ-13-2-SDC}
\clearpage


\fg[\textwidth]{rys/01_VM/14-SDC.png}{14-SDC}{OZNACZ-14-SDC}
\clearpage

\fg[\textwidth]{rys/01_VM/15-SDC.png}{15-SDC}{OZNACZ-15-SDC}
\clearpage
Zmiana nazwy serwera na SDC96 przedstawiona na rysunkach od \OznaczZdjecie[Rys.]{OZNACZ-16-SDC} do \OznaczZdjecie[Rys.]{OZNACZ-20-SDC}
\fg[\textwidth]{rys/01_VM/16-SDC.png}{16-SDC}{OZNACZ-16-SDC}
\clearpage

\fg[\textwidth]{rys/01_VM/17-SDC.png}{17-SDC}{OZNACZ-17-SDC}
\clearpage

\fg[\textwidth]{rys/01_VM/18-SDC.png}{18-SDC}{OZNACZ-18-SDC}
\clearpage

\fg[\textwidth]{rys/01_VM/19-SDC.png}{19-SDC}{OZNACZ-19-SDC}
\clearpage

\fg[\textwidth]{rys/01_VM/20-SDC.png}{20-SDC}{OZNACZ-20-SDC}
\clearpage
\subsubsection{Serwer SMP96}
Aby zainstalować serwer SMP96 należy wykonać te same kroki co na serwerze SDC96, więc nie będą one powtarzane, jednakże nie zmieniamy języka, więć nie będzie potrzebna nam karta sieciowa NAT więc można od razu ustawić karte sieciową w trybie sieci wewnętrznej(każdy serwer musi mieć swoją kartę sieciową ustawioną w ten sposób). Aby to zrobić należy wykonać kroki przedstawione na rysunkach od \OznaczZdjecie[Rys.]{OZNACZ-25-SMP} do \OznaczZdjecie[Rys.]{OZNACZ-26-SMP}

\fg[\textwidth]{rys/01_VM/25-SMP.png}{25-SMP}{OZNACZ-25-SMP}

\clearpage

\fg[\textwidth]{rys/01_VM/26-SMP.png}{26-SMP}{OZNACZ-26-SMP}

\clearpage

Aby wykonać kolejne kroki (tj. \OznaczZdjecie[Rys.]{OZNACZ-27-SMP} - OZNACZ-30-SMP) konieczna jest konfiguracja AD(\ref{sec:AD})

\fg[\textwidth]{rys/01_VM/27-SMP.png}{27-SMP}{OZNACZ-27-SMP}
\clearpage

\fg[\textwidth]{rys/01_VM/28-SMP.png}{28-SMP}{OZNACZ-28-SMP}
\clearpage

\fg[\textwidth]{rys/01_VM/29-SMP.png}{29-SMP}{OZNACZ-29-SMP}
\clearpage

\fg[\textwidth]{rys/01_VM/30-SMP.png}{30-SMP}{OZNACZ-30-SMP}
\clearpage

\subsubsection{Dodwanie serwerów do menadżera serwerów w SDC96}
Aby dodać serwer SMP96(bądź jakikolwiek inny) do menadżera serwerów w SDC96 należy wykonać kroki przedstawione na rysunkach od \OznaczZdjecie[Rys.]{OZNACZ-33-SDC} do \OznaczZdjecie[Rys.]{OZNACZ-34-SDC}. Pozwala to na zarządzanie serwerami z jednego miejsca(np. dodwanie funkcji bądź roli).

\fg[\textwidth]{rys/01_VM/33-SDC.png}{33-SDC}{OZNACZ-33-SDC}
\clearpage

\fg[\textwidth]{rys/01_VM/34-SDC.png}{34-SDC}{OZNACZ-34-SDC}
\clearpage

\subsubsection{Serwery SN1-96 i SN2-96}

Ponownie, aby dołączyć serwery SN1-96 oraz SN2-96 do domeny należy wykonać kroki \OznaczZdjecie[Rys.]{OZNACZ-38-SN1} - \OznaczZdjecie[Rys.]{OZNACZ-40-SN1} lecz konieczna jest wcześniejsza konfiguracja AD(sekcja \ref{sec:AD})

\fg[\textwidth]{rys/01_VM/38-SN1.png}{38-SN1}{OZNACZ-38-SN1}

\clearpage

\fg[\textwidth]{rys/01_VM/39-SN1.png}{39-SN1}{OZNACZ-39-SN1}

\clearpage

\fg[\textwidth]{rys/01_VM/40-SN1.png}{40-SN1}{OZNACZ-40-SN1}

\clearpage

% * ---------------------------------------------------------------------------------------------------------------------------%

\subsection{Konfiguracja Active Directory}
\label{sec:AD}
Aby skonfigurować Active Directory należy wykonać kroki przedstawione na rysunkach od \OznaczZdjecie[Rys.]{OZNACZ-01-AD} do \OznaczZdjecie[Rys.]{OZNACZ-21-AD}. Wszystkie kroki dotyczą serwera SDC96.\\


Na rysunkach \OznaczZdjecie[Rys.]{OZNACZ-01-AD} - \OznaczZdjecie[Rys.]{OZNACZ-03-AD} przedstawiono proces dodawania kart sieciowych do maszyny wirtualnej. Karta sieciowa nr 1 to Karta NAT. Będzie ona potrzebna, gdy będziemy pobierali aktualizacje bądź instalowali oprogramowanie zewnętrzne(np. WordPress w sekcji nr.\ref{sec:WP}), natomiast karta sieciowa nr 2 to karta sieci wewnętrznej. Będzie ona potrzebna do komunikacji z innymi serwerami czy stacjami klienckimi w sieci. 

\fg[\textwidth]{rys/04_Procedury_Instalacyjne/01_AD_DNS/01-AD.png}{01-AD}{OZNACZ-01-AD}
\clearpage

\fg[\textwidth]{rys/04_Procedury_Instalacyjne/01_AD_DNS/02-AD.png}{02-AD}{OZNACZ-02-AD}
\clearpage

\fg[\textwidth]{rys/04_Procedury_Instalacyjne/01_AD_DNS/03-AD.png}{03-AD}{OZNACZ-03-AD}
\clearpage

Na rysunkach \OznaczZdjecie[Rys.]{OZNACZ-04-AD}  oraz \OznaczZdjecie[Rys.]{OZNACZ-05-AD} przedstawiono konfiguracje adresu IP serwera SDC96. 

\fg[\textwidth]{rys/04_Procedury_Instalacyjne/01_AD_DNS/04-AD.png}{04-AD}{OZNACZ-04-AD}
\clearpage

\fg[\textwidth]{rys/04_Procedury_Instalacyjne/01_AD_DNS/05-AD.png}{05-AD}{OZNACZ-05-AD}
\clearpage

Na rysunkach \OznaczZdjecie[Rys.]{OZNACZ-06-AD} - \OznaczZdjecie[Rys.]{OZNACZ-13-AD} przedstawiono proces instalacji roli Active Directory Domain Services oraz DNS(chociaż narazie go nie konfigurujemy). 

\fg[\textwidth]{rys/04_Procedury_Instalacyjne/01_AD_DNS/06-Ad.png}{06-Ad}{OZNACZ-06-Ad}
\OznaczZdjecie[]{OZNACZ-06-Ad}
\clearpage

\fg[\textwidth]{rys/04_Procedury_Instalacyjne/01_AD_DNS/07-AD.png}{07-AD}{OZNACZ-07-AD}
\OznaczZdjecie[]{OZNACZ-07-AD}
\clearpage

\fg[\textwidth]{rys/04_Procedury_Instalacyjne/01_AD_DNS/08-AD.png}{08-AD}{OZNACZ-08-AD}
\OznaczZdjecie[]{OZNACZ-08-AD}
\clearpage

\fg[\textwidth]{rys/04_Procedury_Instalacyjne/01_AD_DNS/09-AD.png}{09-AD}{OZNACZ-09-AD}
\OznaczZdjecie[]{OZNACZ-09-AD}
\clearpage

\fg[\textwidth]{rys/04_Procedury_Instalacyjne/01_AD_DNS/10-AD.png}{10-AD}{OZNACZ-10-AD}
\OznaczZdjecie[]{OZNACZ-10-AD}
\clearpage

\fg[\textwidth]{rys/04_Procedury_Instalacyjne/01_AD_DNS/11-AD.png}{11-AD}{OZNACZ-11-AD}
\OznaczZdjecie[]{OZNACZ-11-AD}
\clearpage

\fg[\textwidth]{rys/04_Procedury_Instalacyjne/01_AD_DNS/12-AD.png}{12-AD}{OZNACZ-12-AD}
\OznaczZdjecie[]{OZNACZ-12-AD}
\clearpage

\fg[\textwidth]{rys/04_Procedury_Instalacyjne/01_AD_DNS/13-AD.png}{13-AD}{OZNACZ-13-AD}
\OznaczZdjecie[]{OZNACZ-13-AD}
\clearpage

Po kliknięciu `\textit{Promote this server to a domain controller}` zostanie wyświetlone okno konfiguracji usługi Active Directory Domain Services. Na rysunkach \OznaczZdjecie[Rys.]{OZNACZ-14-AD} - \OznaczZdjecie[Rys.]{OZNACZ-20-AD} przedstawiono proces konfiguracji domeny.

\fg[\textwidth]{rys/04_Procedury_Instalacyjne/01_AD_DNS/14-AD.png}{14-AD}{OZNACZ-14-AD}
\clearpage

\fg[\textwidth]{rys/04_Procedury_Instalacyjne/01_AD_DNS/15-AD.png}{15-AD}{OZNACZ-15-AD}
\clearpage

\fg[\textwidth]{rys/04_Procedury_Instalacyjne/01_AD_DNS/16-AD.png}{16-AD}{OZNACZ-16-AD}
\clearpage

\fg[\textwidth]{rys/04_Procedury_Instalacyjne/01_AD_DNS/18-AD.png}{18-AD}{OZNACZ-18-AD}
\clearpage

\fg[\textwidth]{rys/04_Procedury_Instalacyjne/01_AD_DNS/19-AD.png}{19-AD}{OZNACZ-19-AD}
\clearpage

\fg[\textwidth]{rys/04_Procedury_Instalacyjne/01_AD_DNS/20-AD.png}{20-AD}{OZNACZ-20-AD}
\clearpage

\fg[\textwidth]{rys/04_Procedury_Instalacyjne/01_AD_DNS/21-AD.png}{21-AD}{OZNACZ-21-AD}
 %TODO PRZENIEŚĆ DO TESTÓW 
\clearpage

% *---------------------------------------------------------------------------------------------------------------------------%

\subsection{Zdjęcia dotyczące DNS}

Aby skonfigurować DNS należy wykonać kroki przedstawione na rysunkach od \OznaczZdjecie[Rys.]{OZNACZ-01-DNS} do \OznaczZdjecie[Rys.]{OZNACZ-15-DNS}. Jeżeli nie widzimy opcji `DNS` w menadżerze serwera, należy zainstalować rolę DNS na serwerze SDC96 co było przedstawione wcześniej w sekcji Active Directory(sekcja nr.\ref{sec:AD}).\\


\fg[\textwidth]{rys/04_Procedury_Instalacyjne/01_AD_DNS/DNS/01-DNS.png}{01-DNS}{OZNACZ-01-DNS}
\clearpage

Na rysunkach \OznaczZdjecie[Rys.]{OZNACZ-02-DNS} - \OznaczZdjecie[Rys.]{OZNACZ-11-DNS} dodajemy hosta dns oraz strefy przeszukiwania w przód i wstecz. 

\fg[\textwidth]{rys/04_Procedury_Instalacyjne/01_AD_DNS/DNS/02-DNS.png}{02-DNS}{OZNACZ-02-DNS}
\clearpage

\fg[\textwidth]{rys/04_Procedury_Instalacyjne/01_AD_DNS/DNS/03-DNS.png}{03-DNS}{OZNACZ-03-DNS}
\clearpage

\fg[\textwidth]{rys/04_Procedury_Instalacyjne/01_AD_DNS/DNS/04-DNS.png}{04-DNS}{OZNACZ-04-DNS}
\clearpage

\fg[\textwidth]{rys/04_Procedury_Instalacyjne/01_AD_DNS/DNS/05-DNS.png}{05-DNS}{OZNACZ-05-DNS}

\clearpage

\fg[\textwidth]{rys/04_Procedury_Instalacyjne/01_AD_DNS/DNS/06-DNS.png}{06-DNS}{OZNACZ-06-DNS}

\clearpage

\fg[\textwidth]{rys/04_Procedury_Instalacyjne/01_AD_DNS/DNS/07-DNS.png}{07-DNS}{OZNACZ-07-DNS}

\clearpage

\fg[\textwidth]{rys/04_Procedury_Instalacyjne/01_AD_DNS/DNS/08-DNS.png}{08-DNS}{OZNACZ-08-DNS}

\clearpage

\fg[\textwidth]{rys/04_Procedury_Instalacyjne/01_AD_DNS/DNS/09-DNS.png}{09-DNS}{OZNACZ-09-DNS}

\clearpage

\fg[\textwidth]{rys/04_Procedury_Instalacyjne/01_AD_DNS/DNS/10-DNS.png}{10-DNS}{OZNACZ-10-DNS}

\clearpage

\fg[\textwidth]{rys/04_Procedury_Instalacyjne/01_AD_DNS/DNS/11-DNS.png}{11-DNS}{OZNACZ-11-DNS}
\clearpage

Następnie, na rysunkach \OznaczZdjecie[Rys.]{OZNACZ-13-DNS} - \OznaczZdjecie[Rys.]{OZNACZ-13-DNS} dodajemy wskaźnik strefy przeszukiwania wstecz.
\fg[\textwidth]{rys/04_Procedury_Instalacyjne/01_AD_DNS/DNS/12-DNS.png}{12-DNS}{OZNACZ-12-DNS}
\clearpage

\fg[\textwidth]{rys/04_Procedury_Instalacyjne/01_AD_DNS/DNS/13-DNS.png}{13-DNS}{OZNACZ-13-DNS}
\clearpage

\fg[\textwidth]{rys/04_Procedury_Instalacyjne/01_AD_DNS/DNS/14-DNS.png}{14-DNS}{OZNACZ-14-DNS}
 %TODO PRZENIEŚĆ DO TESTÓW
\clearpage

\fg[\textwidth]{rys/04_Procedury_Instalacyjne/01_AD_DNS/DNS/15-DNS.png}{15-DNS}{OZNACZ-15-DNS}
 %TODO PRZENIEŚĆ DO TESTÓW
\clearpage

% *---------------------------------------------------------------------------------------------------------------------------%

\subsection{Zdjęcia dotyczące Grup i Kont}

Dodawanie użytkowników oraz grup do domeny zostało zautomatyzowane przy pomocy skryptu Powershell\footnote{Skrypt utworzony za pomocą strony\cite{Powershell}} przedstawionego na \OznaczKod{skrypt}. Skrypt przyjmuje plik csv\OznaczKod{dane} z spreparowanymi danymi użytkowników, grup, haseł oraz przypisanych do nich grup.

\clearpage	
\ListingFile{Skrypt Powershell dodający użytkowników}{skrypt}

\ListingFile{Fragment pliku CSV z danymi użytkowników}{dane}

\clearpage

Aby uruchomić skrypt należy otowrzyć środowisko Powershell ISE(\OznaczZdjecie[Rys.]{OZNACZ-01-Grupy-K}), `przerzucić' do niego skrypt oraz go uruchomić, tak jak na \OznaczZdjecie[Rys.]{OZNACZ-02-Grupy-K}
\fg[\textwidth]{rys/04_Procedury_Instalacyjne/02_Grupy_Konta/01.png}{01}{OZNACZ-01-Grupy-K}

\clearpage

\fg[\textwidth]{rys/04_Procedury_Instalacyjne/02_Grupy_Konta/02.png}{02}{OZNACZ-02-Grupy-K}

\clearpage

Dalsze kroki przedstawione na rysunkach od \OznaczZdjecie[Rys.]{OZNACZ-03-Grupy-K} do \OznaczZdjecie[Rys.]{OZNACZ-06-Grupy-K} przedstawiają dane poszczególnych użytkowników na przykładzie 1 z nich.


\fg[\textwidth]{rys/04_Procedury_Instalacyjne/02_Grupy_Konta/03.png}{03}{OZNACZ-03-Grupy-K}

\clearpage

\fg[\textwidth]{rys/04_Procedury_Instalacyjne/02_Grupy_Konta/04.png}{04}{OZNACZ-04-Grupy-K}

\clearpage

\fg[\textwidth]{rys/04_Procedury_Instalacyjne/02_Grupy_Konta/05.png}{05}{OZNACZ-05-Grupy-K}

\clearpage

\fg[\textwidth]{rys/04_Procedury_Instalacyjne/02_Grupy_Konta/06.png}{06}{OZNACZ-06-Grupy-K}

\clearpage

Następnie konieczne jest dodanie grup organizacjnch do grup funkcyjnch(tj.Domain Users, Domain Admins...) zależnie od poziomu kontroli jaki mają one mieć. Na rysunkach od \OznaczZdjecie[Rys.]{OZNACZ-07-Grupy-K} do \OznaczZdjecie[Rys.]{OZNACZ-15-Grupy-K} przedstawiono proces dodawania grup organizacyjnych do grup funkcyjnych.

\fg[\textwidth]{rys/04_Procedury_Instalacyjne/02_Grupy_Konta/07.png}{07}{OZNACZ-07-Grupy-K}

\clearpage

\fg[\textwidth]{rys/04_Procedury_Instalacyjne/02_Grupy_Konta/08.png}{08}{OZNACZ-08-Grupy-K}

\clearpage

\fg[\textwidth]{rys/04_Procedury_Instalacyjne/02_Grupy_Konta/09.png}{09}{OZNACZ-09-Grupy-K}

\clearpage

\fg[\textwidth]{rys/04_Procedury_Instalacyjne/02_Grupy_Konta/10.png}{10}{OZNACZ-10-Grupy-K}

\clearpage

\fg[\textwidth]{rys/04_Procedury_Instalacyjne/02_Grupy_Konta/11.png}{11}{OZNACZ-11-Grupy-K}

\clearpage

\fg[\textwidth]{rys/04_Procedury_Instalacyjne/02_Grupy_Konta/12.png}{12}{OZNACZ-12-Grupy-K}

\clearpage

\fg[\textwidth]{rys/04_Procedury_Instalacyjne/02_Grupy_Konta/13.png}{13}{OZNACZ-13-Grupy-K}

\clearpage

\fg[\textwidth]{rys/04_Procedury_Instalacyjne/02_Grupy_Konta/14.png}{14}{OZNACZ-14-Grupy-K}

\clearpage

\fg[\textwidth]{rys/04_Procedury_Instalacyjne/02_Grupy_Konta/15.png}{15}{OZNACZ-15-Grupy-K} %TODO PRZENIEŚĆ DO TESTÓW

\clearpage

% *---------------------------------------------------------------------------------------------------------------------------%

\subsection{Zdjęcia dotyczące RAID}

Aby rozpocząć konfigurację RAID należy dodać dyski do maszyny wirtualnej SMP96. Proces ten przedstawiony jest na rysunkach od \OznaczZdjecie[Rys.]{OZNACZ-01-RAID} do \OznaczZdjecie[Rys.]{OZNACZ-05-RAID}. 

\clearpage
\fg[\textwidth]{rys/04_Procedury_Instalacyjne/03_Klaster_iSCSI/RAID/01.png}{01}{OZNACZ-01-RAID}

\clearpage

\fg[\textwidth]{rys/04_Procedury_Instalacyjne/03_Klaster_iSCSI/RAID/02.png}{02}{OZNACZ-02-RAID}

\clearpage

\fg[\textwidth]{rys/04_Procedury_Instalacyjne/03_Klaster_iSCSI/RAID/03.png}{03}{OZNACZ-03-RAID}

\clearpage

\fg[\textwidth]{rys/04_Procedury_Instalacyjne/03_Klaster_iSCSI/RAID/04.png}{04}{OZNACZ-04-RAID}

\clearpage

\fg[\textwidth]{rys/04_Procedury_Instalacyjne/03_Klaster_iSCSI/RAID/05.png}{05}{OZNACZ-05-RAID}

\clearpage

Po udanym przydzieleniu dysków do maszyny wirtualnej SMP96, należy uruchomić ją i uruchomić program `Disk Management`. Następnie należy wykonać kroki przedstawione na rysunkach od \OznaczZdjecie[Rys.]{OZNACZ-06-RAID} do \OznaczZdjecie[Rys.]{OZNACZ-12-RAID}.

\fg[\textwidth]{rys/04_Procedury_Instalacyjne/03_Klaster_iSCSI/RAID/06.png}{06}{OZNACZ-06-RAID}

\clearpage

\fg[\textwidth]{rys/04_Procedury_Instalacyjne/03_Klaster_iSCSI/RAID/07.png}{07}{OZNACZ-07-RAID}

\clearpage

\fg[\textwidth]{rys/04_Procedury_Instalacyjne/03_Klaster_iSCSI/RAID/08.png}{08}{OZNACZ-08-RAID}

\clearpage

\fg[\textwidth]{rys/04_Procedury_Instalacyjne/03_Klaster_iSCSI/RAID/09.png}{09}{OZNACZ-09-RAID}

\clearpage

\fg[\textwidth]{rys/04_Procedury_Instalacyjne/03_Klaster_iSCSI/RAID/10.png}{10}{OZNACZ-10-RAID}

\clearpage

\fg[\textwidth]{rys/04_Procedury_Instalacyjne/03_Klaster_iSCSI/RAID/11.png}{11}{OZNACZ-11-RAID}

\clearpage

\fg[\textwidth]{rys/04_Procedury_Instalacyjne/03_Klaster_iSCSI/RAID/12.png}{12}{OZNACZ-12-RAID}

\clearpage

% *---------------------------------------------------------------------------------------------------------------------------%

\subsection{Zdjęcia dotyczące iSCSI}

Aby rozpocząć konfigurację iSCSI należy dodać rolę iSCSI Target Server na serwerze SMP96. Proces ten przedstawiony jest na rysunkach od \OznaczZdjecie[Rys.]{OZNACZ-01-iSCSI} do \OznaczZdjecie[Rys.]{OZNACZ-02-iSCSI}.

\clearpage

\fg[\textwidth]{rys/04_Procedury_Instalacyjne/03_Klaster_iSCSI/iSCSI/01.png}{01}{OZNACZ-01-iSCSI}

\clearpage

\fg[\textwidth]{rys/04_Procedury_Instalacyjne/03_Klaster_iSCSI/iSCSI/02.png}{02}{OZNACZ-02-iSCSI}

\clearpage

Po wykonaniu powyższych kroków, należy uruchomić program `Server Manager` i przejść do `File and Storage Services` -> `iSCSI`. Następnie należy wykonać kroki przedstawione na rysunkach od \OznaczZdjecie[Rys.]{OZNACZ-03-iSCSI} do \OznaczZdjecie[Rys.]{OZNACZ-08-iSCSI}.

\fg[\textwidth]{rys/04_Procedury_Instalacyjne/03_Klaster_iSCSI/iSCSI/03.png}{03}{OZNACZ-03-iSCSI}

\clearpage

\fg[\textwidth]{rys/04_Procedury_Instalacyjne/03_Klaster_iSCSI/iSCSI/04.png}{04}{OZNACZ-04-iSCSI}

\clearpage

\fg[\textwidth]{rys/04_Procedury_Instalacyjne/03_Klaster_iSCSI/iSCSI/05.png}{05}{OZNACZ-05-iSCSI}

\clearpage

\fg[\textwidth]{rys/04_Procedury_Instalacyjne/03_Klaster_iSCSI/iSCSI/06.png}{06}{OZNACZ-06-iSCSI}

\clearpage

\fg[\textwidth]{rys/04_Procedury_Instalacyjne/03_Klaster_iSCSI/iSCSI/07.png}{07}{OZNACZ-07-iSCSI}

\clearpage

\fg[\textwidth]{rys/04_Procedury_Instalacyjne/03_Klaster_iSCSI/iSCSI/08.png}{08}{OZNACZ-08-iSCSI}

\clearpage

W międzyczasie, na serwerach SN1-96 oraz SN2-96 należy dodać i skonfigurować rolę iSCSI Initiator. Proces ten przedstawiony jest na rysunkach od \OznaczZdjecie[Rys.]{OZNACZ-09-iSCSI} do \OznaczZdjecie[Rys.]{OZNACZ-10-iSCSI}.  

\fg[\textwidth]{rys/04_Procedury_Instalacyjne/03_Klaster_iSCSI/iSCSI/09.png}{09}{OZNACZ-09-iSCSI}

\clearpage


\fg[\textwidth]{rys/04_Procedury_Instalacyjne/03_Klaster_iSCSI/iSCSI/10.png}{10}{OZNACZ-10-iSCSI}

W trakcie konfiguracji iSCSI Initiator na serwerach SN1-96 oraz SN2-96, należy podać adres IP serwera SMP96. Niestety doszło do pomyłki i podano błędny adres IP. Poprawny adres IP serwera SMP96 to 192.168.96.41 a nie 192.168.96.199. Problem ten został w tamtym momencie nie zauważony, więc nie został poprawiony na zdjęciach(reszta w sekcji Wniosków(Sekcja nr.\ref{sec:wnioski})). Mimo błędu, kroki nadal są poprawne i można je wykonać z podanym adresem IP.

\clearpage

Następnie na serwerze SMP96 należy kontynuuwać konfigurację iSCSI. Proces ten przedstawiony jest na rysunkach od \OznaczZdjecie[Rys.]{OZNACZ-11-iSCSI} do \OznaczZdjecie[Rys.]{OZNACZ-14-iSCSI}.

\fg[\textwidth]{rys/04_Procedury_Instalacyjne/03_Klaster_iSCSI/iSCSI/11.png}{11}{OZNACZ-11-iSCSI}

\clearpage

\fg[\textwidth]{rys/04_Procedury_Instalacyjne/03_Klaster_iSCSI/iSCSI/12.png}{12}{OZNACZ-12-iSCSI}

\clearpage

\fg[\textwidth]{rys/04_Procedury_Instalacyjne/03_Klaster_iSCSI/iSCSI/13.png}{13}{OZNACZ-13-iSCSI}

\clearpage

\fg[\textwidth]{rys/04_Procedury_Instalacyjne/03_Klaster_iSCSI/iSCSI/14.png}{14}{OZNACZ-14-iSCSI}

\clearpage

Gdy skończymy konfigurację iSCSI na serwerze SMP96, należy wrócić do serwerów SN1-96 oraz SN2-96 i dokończyć konfigurację iSCSI Initiator. Proces ten przedstawiony jest na rysunkach od \OznaczZdjecie[Rys.]{OZNACZ-15-iSCSI} do \OznaczZdjecie[Rys.]{OZNACZ-16-iSCSI}.

\fg[\textwidth]{rys/04_Procedury_Instalacyjne/03_Klaster_iSCSI/iSCSI/15.png}{15}{OZNACZ-15-iSCSI}

\clearpage

\fg[\textwidth]{rys/04_Procedury_Instalacyjne/03_Klaster_iSCSI/iSCSI/16.png}{16}{OZNACZ-16-iSCSI}

\clearpage

Później, dla wygody zarządzania dodajemy serwery SN1-96 oraz SN2-96 do menadżera serwerów w SDC96. Proces ten przedstawiony jest na rysunku \OznaczZdjecie[Rys.]{OZNACZ-17-iSCSI}. Gdy już to zrobimy instalujemy funkcje `Failover Clustering` na serwerach SN1-96 oraz SN2-96. Proces ten przedstawiony jest na rysunkach od \OznaczZdjecie[Rys.]{OZNACZ-18-iSCSI} do \OznaczZdjecie[Rys.]{OZNACZ-20-iSCSI}.

\fg[\textwidth]{rys/04_Procedury_Instalacyjne/03_Klaster_iSCSI/iSCSI/17.png}{17}{OZNACZ-17-iSCSI}

\clearpage

\fg[\textwidth]{rys/04_Procedury_Instalacyjne/03_Klaster_iSCSI/iSCSI/18.png}{18}{OZNACZ-18-iSCSI}

\clearpage

\fg[\textwidth]{rys/04_Procedury_Instalacyjne/03_Klaster_iSCSI/iSCSI/19.png}{19}{OZNACZ-19-iSCSI}

\clearpage

\fg[\textwidth]{rys/04_Procedury_Instalacyjne/03_Klaster_iSCSI/iSCSI/20.png}{20}{OZNACZ-20-iSCSI}

\clearpage

Następnie, przechodzimy do managera dysków na SN1 i wykonujemy kroki przedstawione na rysunkach od \OznaczZdjecie[Rys.]{OZNACZ-21-iSCSI} do \OznaczZdjecie[Rys.]{OZNACZ-26-iSCSI}.

\fg[\textwidth]{rys/04_Procedury_Instalacyjne/03_Klaster_iSCSI/iSCSI/21.png}{21}{OZNACZ-21-iSCSI}

\clearpage

\fg[\textwidth]{rys/04_Procedury_Instalacyjne/03_Klaster_iSCSI/iSCSI/22.png}{22}{OZNACZ-22-iSCSI}

\clearpage

\fg[\textwidth]{rys/04_Procedury_Instalacyjne/03_Klaster_iSCSI/iSCSI/23.png}{23}{OZNACZ-23-iSCSI}

\clearpage

\fg[\textwidth]{rys/04_Procedury_Instalacyjne/03_Klaster_iSCSI/iSCSI/24.png}{24}{OZNACZ-24-iSCSI}
\OznaczZdjecie[]{OZNACZ-24-iSCSI}
\clearpage

\fg[\textwidth]{rys/04_Procedury_Instalacyjne/03_Klaster_iSCSI/iSCSI/25.png}{25}{OZNACZ-25-iSCSI}

\clearpage

\fg[\textwidth]{rys/04_Procedury_Instalacyjne/03_Klaster_iSCSI/iSCSI/26.png}{26}{OZNACZ-26-iSCSI}

\clearpage

Następnie, przechodzimy do programu `Failover Cluster Manager` na serwerze SN1-96 i wykonujemy kroki przedstawione na rysunkach od \OznaczZdjecie[Rys.]{OZNACZ-27-iSCSI} do \OznaczZdjecie[Rys.]{OZNACZ-36-iSCSI}.

\fg[\textwidth]{rys/04_Procedury_Instalacyjne/03_Klaster_iSCSI/iSCSI/27.png}{27}{OZNACZ-27-iSCSI}

\clearpage

\fg[\textwidth]{rys/04_Procedury_Instalacyjne/03_Klaster_iSCSI/iSCSI/28.png}{28}{OZNACZ-28-iSCSI}

\clearpage

\fg[\textwidth]{rys/04_Procedury_Instalacyjne/03_Klaster_iSCSI/iSCSI/29.png}{29}{OZNACZ-29-iSCSI}

\clearpage

\fg[\textwidth]{rys/04_Procedury_Instalacyjne/03_Klaster_iSCSI/iSCSI/30.png}{30}{OZNACZ-30-iSCSI}

\clearpage

\fg[\textwidth]{rys/04_Procedury_Instalacyjne/03_Klaster_iSCSI/iSCSI/31.png}{31}{OZNACZ-31-iSCSI}

\clearpage

\fg[\textwidth]{rys/04_Procedury_Instalacyjne/03_Klaster_iSCSI/iSCSI/32.png}{32}{OZNACZ-32-iSCSI}

\clearpage

\fg[\textwidth]{rys/04_Procedury_Instalacyjne/03_Klaster_iSCSI/iSCSI/33.png}{33}{OZNACZ-33-iSCSI}

\clearpage

\fg[\textwidth]{rys/04_Procedury_Instalacyjne/03_Klaster_iSCSI/iSCSI/34.png}{34}{OZNACZ-34-iSCSI}

\clearpage

\fg[\textwidth]{rys/04_Procedury_Instalacyjne/03_Klaster_iSCSI/iSCSI/35.png}{35}{OZNACZ-35-iSCSI}

\clearpage

\fg[\textwidth]{rys/04_Procedury_Instalacyjne/03_Klaster_iSCSI/iSCSI/36.png}{36}{OZNACZ-36-iSCSI}

% Widok potwierdzenia będzie wyglądał odrobinę inaczej, z powodu błędnego adresu IP serwera SMP96. Wszystkie kroki są jednak poprawne i można je wykonać z podanym adresem IP.

\clearpage

% *---------------------------------------------------------------------------------------------------------------------------%

\subsection{Zdjęcia dotyczące FS}

Aby rozpocząć konfigurację FS należy dodać rolę File Server na serwerach SN1-96 oraz SN2-96. Proces ten przedstawiony jest na rysunkach od \OznaczZdjecie[Rys.]{OZNACZ-01-FS} do \OznaczZdjecie[Rys.]{OZNACZ-03-FS}.

\fg[\textwidth]{rys/04_Procedury_Instalacyjne/03_Klaster_iSCSI/FS/01.png}{01}{OZNACZ-01-FS}

\clearpage

\fg[\textwidth]{rys/04_Procedury_Instalacyjne/03_Klaster_iSCSI/FS/02.png}{02}{OZNACZ-02-FS}

\clearpage

\fg[\textwidth]{rys/04_Procedury_Instalacyjne/03_Klaster_iSCSI/FS/03.png}{03}{OZNACZ-03-FS}

\clearpage

Następnie, na serwerze SMP96 należy dodać nowe wirtualne dyski. Proces ten przedstawiony jest na rysunkach od \OznaczZdjecie[Rys.]{OZNACZ-04-FS} do \OznaczZdjecie[Rys.]{OZNACZ-10-FS}.

\fg[\textwidth]{rys/04_Procedury_Instalacyjne/03_Klaster_iSCSI/FS/04.png}{04}{OZNACZ-04-FS}

\clearpage

\fg[\textwidth]{rys/04_Procedury_Instalacyjne/03_Klaster_iSCSI/FS/05.png}{05}{OZNACZ-05-FS}

\clearpage

\fg[\textwidth]{rys/04_Procedury_Instalacyjne/03_Klaster_iSCSI/FS/06.png}{06}{OZNACZ-06-FS}

\clearpage

\fg[\textwidth]{rys/04_Procedury_Instalacyjne/03_Klaster_iSCSI/FS/07.png}{07}{OZNACZ-07-FS}

\clearpage

\fg[\textwidth]{rys/04_Procedury_Instalacyjne/03_Klaster_iSCSI/FS/08.png}{08}{OZNACZ-08-FS}

\clearpage

\fg[\textwidth]{rys/04_Procedury_Instalacyjne/03_Klaster_iSCSI/FS/09.png}{09}{OZNACZ-09-FS}

\clearpage

\fg[\textwidth]{rys/04_Procedury_Instalacyjne/03_Klaster_iSCSI/FS/10.png}{10}{OZNACZ-10-FS}

\clearpage

Kontynnując, na serwerze SN1-96 należy dodać skonfigurować nową rolę File Server. Proces ten przedstawiony jest na rysunkach od \OznaczZdjecie[Rys.]{OZNACZ-11-FS} do \OznaczZdjecie[Rys.]{OZNACZ-14-FS}.

\fg[\textwidth]{rys/04_Procedury_Instalacyjne/03_Klaster_iSCSI/FS/11.png}{11}{OZNACZ-11-FS}

\clearpage

\fg[\textwidth]{rys/04_Procedury_Instalacyjne/03_Klaster_iSCSI/FS/12.png}{12}{OZNACZ-12-FS}

\clearpage

\fg[\textwidth]{rys/04_Procedury_Instalacyjne/03_Klaster_iSCSI/FS/13.png}{13}{OZNACZ-13-FS}

\clearpage

\fg[\textwidth]{rys/04_Procedury_Instalacyjne/03_Klaster_iSCSI/FS/14.png}{14}{OZNACZ-14-FS}

\clearpage

Jako kolejny krok należy na serwerach SN1-96 oraz SN2-96 uruchomić narzędzie `iSCSI Initiator` i połączyć się z serwerem SMP96. Proces ten przedstawiony jest na rysunkach od \OznaczZdjecie[Rys.]{OZNACZ-15-FS} do \OznaczZdjecie[Rys.]{OZNACZ-17-FS}.

\fg[\textwidth]{rys/04_Procedury_Instalacyjne/03_Klaster_iSCSI/FS/15.png}{15}{OZNACZ-15-FS}

\clearpage

\fg[\textwidth]{rys/04_Procedury_Instalacyjne/03_Klaster_iSCSI/FS/16.png}{16}{OZNACZ-16-FS}

\clearpage

\fg[\textwidth]{rys/04_Procedury_Instalacyjne/03_Klaster_iSCSI/FS/17.png}{17}{OZNACZ-17-FS}

\clearpage

Następnie, na serwerze SN1-96 należy wejść w program `Disk Management`, powinniśmy tam zobaczyć nowy dysk. Proces jego konfiguracji przedstawiony jest na rysunkach od \OznaczZdjecie[Rys.]{OZNACZ-18-FS} do \OznaczZdjecie[Rys.]{OZNACZ-21-FS}.
\fg[\textwidth]{rys/04_Procedury_Instalacyjne/03_Klaster_iSCSI/FS/18.png}{18}{OZNACZ-18-FS}

\clearpage

\fg[\textwidth]{rys/04_Procedury_Instalacyjne/03_Klaster_iSCSI/FS/19.png}{19}{OZNACZ-19-FS}

\clearpage

\fg[\textwidth]{rys/04_Procedury_Instalacyjne/03_Klaster_iSCSI/FS/20.png}{20}{OZNACZ-20-FS}

\clearpage

\fg[\textwidth]{rys/04_Procedury_Instalacyjne/03_Klaster_iSCSI/FS/21.png}{21}{OZNACZ-21-FS}

\clearpage

Potem, nadal na serwerze SN1-96, należy uruchomić program `Failover Cluster Manager` i dodać nowy dysk do klastra. Proces ten przedstawiony jest na rysunku \OznaczZdjecie[Rys.]{OZNACZ-22-FS} 

\fg[\textwidth]{rys/04_Procedury_Instalacyjne/03_Klaster_iSCSI/FS/22.png}{22}{OZNACZ-22-FS}

\clearpage

Teraz możemy wznowić przerwaną konfiguracje z powodu braku dysku, co widać na zdjeciach od \OznaczZdjecie[Rys.]{OZNACZ-23-FS} do \OznaczZdjecie[Rys.]{OZNACZ-26-FS}.

\fg[\textwidth]{rys/04_Procedury_Instalacyjne/03_Klaster_iSCSI/FS/23.png}{23}{OZNACZ-23-FS}

\clearpage

\fg[\textwidth]{rys/04_Procedury_Instalacyjne/03_Klaster_iSCSI/FS/24.png}{24}{OZNACZ-24-FS}

\clearpage

\fg[\textwidth]{rys/04_Procedury_Instalacyjne/03_Klaster_iSCSI/FS/25.png}{25}{OZNACZ-25-FS}

\clearpage

\fg[\textwidth]{rys/04_Procedury_Instalacyjne/03_Klaster_iSCSI/FS/26.png}{26}{OZNACZ-26-FS}

\clearpage

Na rysunku \OznaczZdjecie[Rys.]{OZNACZ-27-FS} przedstawiono potwierdzenie dodania nowego dysku do klastra.

\fg[\textwidth]{rys/04_Procedury_Instalacyjne/03_Klaster_iSCSI/FS/27.png}{27}{OZNACZ-27-FS}

\clearpage

Następnie na serwerze SN1-96 należy dodać współdzielenie plików na nowym serwerze plików (SFS96) i skonfigurować uprawnienia. Proces ten przedstawiony jest na rysunkach od \OznaczZdjecie[Rys.]{OZNACZ-28-FS} do \OznaczZdjecie[Rys.]{OZNACZ-35-FS}.

\fg[\textwidth]{rys/04_Procedury_Instalacyjne/03_Klaster_iSCSI/FS/28.png}{28}{OZNACZ-28-FS}

\clearpage

\fg[\textwidth]{rys/04_Procedury_Instalacyjne/03_Klaster_iSCSI/FS/29.png}{29}{OZNACZ-29-FS}

\clearpage

\fg[\textwidth]{rys/04_Procedury_Instalacyjne/03_Klaster_iSCSI/FS/30.png}{30}{OZNACZ-30-FS}

\clearpage

\fg[\textwidth]{rys/04_Procedury_Instalacyjne/03_Klaster_iSCSI/FS/31.png}{31}{OZNACZ-31-FS}

\clearpage

\fg[\textwidth]{rys/04_Procedury_Instalacyjne/03_Klaster_iSCSI/FS/32.png}{32}{OZNACZ-32-FS}

\clearpage

\fg[\textwidth]{rys/04_Procedury_Instalacyjne/03_Klaster_iSCSI/FS/33.png}{33}{OZNACZ-33-FS}

\clearpage



\fg[\textwidth]{rys/04_Procedury_Instalacyjne/03_Klaster_iSCSI/FS/34.png}{34}{OZNACZ-34-FS}
\OznaczZdjecie[]{OZNACZ-34-FS}
\clearpage

\fg[\textwidth]{rys/04_Procedury_Instalacyjne/03_Klaster_iSCSI/FS/35.png}{35}{OZNACZ-35-FS}

\clearpage


Następnie, konfigurujemy nowe współudziały dla wszystkich grup organizacyjnych na serwerze. Proces ten przedstawiony jest na rysunkach od \OznaczZdjecie[Rys.]{OZNACZ-36-FS} do \OznaczZdjecie[Rys.]{OZNACZ-48-FS}.

\fg[\textwidth]{rys/04_Procedury_Instalacyjne/03_Klaster_iSCSI/FS/36.png}{36}{OZNACZ-36-FS}
\OznaczZdjecie[]{OZNACZ-36-FS}
\clearpage

\fg[\textwidth]{rys/04_Procedury_Instalacyjne/03_Klaster_iSCSI/FS/37.png}{37}{OZNACZ-37-FS}
\OznaczZdjecie[]{OZNACZ-37-FS}
\clearpage

\fg[\textwidth]{rys/04_Procedury_Instalacyjne/03_Klaster_iSCSI/FS/38.png}{38}{OZNACZ-38-FS}
\OznaczZdjecie[]{OZNACZ-38-FS}
\clearpage

\fg[\textwidth]{rys/04_Procedury_Instalacyjne/03_Klaster_iSCSI/FS/39.png}{39}{OZNACZ-39-FS}
\OznaczZdjecie[]{OZNACZ-39-FS}
\clearpage

\fg[\textwidth]{rys/04_Procedury_Instalacyjne/03_Klaster_iSCSI/FS/40.png}{40}{OZNACZ-40-FS}
\OznaczZdjecie[]{OZNACZ-40-FS}
\clearpage

\fg[\textwidth]{rys/04_Procedury_Instalacyjne/03_Klaster_iSCSI/FS/41.png}{41}{OZNACZ-41-FS}
\OznaczZdjecie[]{OZNACZ-41-FS}
\clearpage

\fg[\textwidth]{rys/04_Procedury_Instalacyjne/03_Klaster_iSCSI/FS/42.png}{42}{OZNACZ-42-FS}
\OznaczZdjecie[]{OZNACZ-42-FS}
\clearpage

\fg[\textwidth]{rys/04_Procedury_Instalacyjne/03_Klaster_iSCSI/FS/43.png}{43}{OZNACZ-43-FS}
\OznaczZdjecie[]{OZNACZ-43-FS}
\clearpage

\fg[\textwidth]{rys/04_Procedury_Instalacyjne/03_Klaster_iSCSI/FS/44.png}{44}{OZNACZ-44-FS}
\OznaczZdjecie[]{OZNACZ-44-FS}
\clearpage

\fg[\textwidth]{rys/04_Procedury_Instalacyjne/03_Klaster_iSCSI/FS/45.png}{45}{OZNACZ-45-FS}
\OznaczZdjecie[]{OZNACZ-45-FS}
\clearpage

\fg[\textwidth]{rys/04_Procedury_Instalacyjne/03_Klaster_iSCSI/FS/46.png}{46}{OZNACZ-46-FS}
\OznaczZdjecie[]{OZNACZ-46-FS}
\clearpage

\fg[\textwidth]{rys/04_Procedury_Instalacyjne/03_Klaster_iSCSI/FS/47.png}{47}{OZNACZ-47-FS}
\OznaczZdjecie[]{OZNACZ-47-FS}
\clearpage

\fg[\textwidth]{rys/04_Procedury_Instalacyjne/03_Klaster_iSCSI/FS/48.png}{48}{OZNACZ-48-FS}
\OznaczZdjecie[]{OZNACZ-48-FS} % TODO DODAĆ DO TESTÓW POZNIEJ
\clearpage

% *---------------------------------------------------------------------------------------------------------------------------%

\subsection{Zdjęcia dotyczące GPO DYSKI}

Aby rozpocząć konfigurację GPO DYSKI należy uruchomić program `Group Policy Management` na serwerze SDC96 oraz dodać ścieżki dla każdej grupy organizacyjnej. Proces ten przedstawiony jest na rysunkach od \OznaczZdjecie[Rys.]{OZNACZ-01-GPO-Dyski} do \OznaczZdjecie[Rys.]{OZNACZ-14-GPO-Dyski}.

\fg[\textwidth]{rys/04_Procedury_Instalacyjne/04_GPO_Mapowanie/GPO_DYSKI/01.png}{01}{OZNACZ-01-GPO-Dyski} 

\clearpage

\fg[\textwidth]{rys/04_Procedury_Instalacyjne/04_GPO_Mapowanie/GPO_DYSKI/02.png}{02}{OZNACZ-02-GPO-Dyski}

\clearpage

\fg[\textwidth]{rys/04_Procedury_Instalacyjne/04_GPO_Mapowanie/GPO_DYSKI/03.png}{03}{OZNACZ-03-GPO-Dyski}

\clearpage

\fg[\textwidth]{rys/04_Procedury_Instalacyjne/04_GPO_Mapowanie/GPO_DYSKI/04.png}{04}{OZNACZ-04-GPO-Dyski}

\clearpage

\fg[\textwidth]{rys/04_Procedury_Instalacyjne/04_GPO_Mapowanie/GPO_DYSKI/05.png}{05}{OZNACZ-05-GPO-Dyski}

\clearpage

\fg[\textwidth]{rys/04_Procedury_Instalacyjne/04_GPO_Mapowanie/GPO_DYSKI/06.png}{06}{OZNACZ-06-GPO-Dyski}
 % TODO DODAĆ DO TESTÓW POZNIEJ
\clearpage

\fg[\textwidth]{rys/04_Procedury_Instalacyjne/04_GPO_Mapowanie/GPO_DYSKI/07.png}{07}{OZNACZ-07-GPO-Dyski}

\clearpage

\fg[\textwidth]{rys/04_Procedury_Instalacyjne/04_GPO_Mapowanie/GPO_DYSKI/08.png}{08}{OZNACZ-08-GPO-Dyski}

\clearpage

\fg[\textwidth]{rys/04_Procedury_Instalacyjne/04_GPO_Mapowanie/GPO_DYSKI/09.png}{09}{OZNACZ-09-GPO-Dyski}

\clearpage

\fg[\textwidth]{rys/04_Procedury_Instalacyjne/04_GPO_Mapowanie/GPO_DYSKI/10.png}{10}{OZNACZ-10-GPO-Dyski}

\clearpage

\fg[\textwidth]{rys/04_Procedury_Instalacyjne/04_GPO_Mapowanie/GPO_DYSKI/11.png}{11}{OZNACZ-11-GPO-Dyski}

\clearpage

\fg[\textwidth]{rys/04_Procedury_Instalacyjne/04_GPO_Mapowanie/GPO_DYSKI/12.png}{12}{OZNACZ-12-GPO-Dyski}

\clearpage

\fg[\textwidth]{rys/04_Procedury_Instalacyjne/04_GPO_Mapowanie/GPO_DYSKI/13.png}{13}{OZNACZ-13-GPO-Dyski}

\clearpage

\fg[\textwidth]{rys/04_Procedury_Instalacyjne/04_GPO_Mapowanie/GPO_DYSKI/14.png}{14}{OZNACZ-14-GPO-Dyski}

\clearpage

Następnie konfigurujemy Item-level targeting dla każdej grupy organizacyjnej w celu przypisania odpowiednich dysków dla każdej grupy. Proces ten przedstawiony jest na rysunkach od \OznaczZdjecie[Rys.]{OZNACZ-15-GPO-Dyski} do \OznaczZdjecie[Rys.]{OZNACZ-31-GPO-Dyski}. 
\fg[\textwidth]{rys/04_Procedury_Instalacyjne/04_GPO_Mapowanie/GPO_DYSKI/15.png}{15}{OZNACZ-15-GPO-Dyski}
\OznaczZdjecie[]{OZNACZ-15-GPO-Dyski}
\clearpage

\fg[\textwidth]{rys/04_Procedury_Instalacyjne/04_GPO_Mapowanie/GPO_DYSKI/16.png}{16}{OZNACZ-16-GPO-Dyski}

\clearpage

\fg[\textwidth]{rys/04_Procedury_Instalacyjne/04_GPO_Mapowanie/GPO_DYSKI/17.png}{17}{OZNACZ-17-GPO-Dyski}

\clearpage

\fg[\textwidth]{rys/04_Procedury_Instalacyjne/04_GPO_Mapowanie/GPO_DYSKI/18.png}{18}{OZNACZ-18-GPO-Dyski}

\clearpage

\fg[\textwidth]{rys/04_Procedury_Instalacyjne/04_GPO_Mapowanie/GPO_DYSKI/19.png}{19}{OZNACZ-19-GPO-Dyski}

\clearpage

\fg[\textwidth]{rys/04_Procedury_Instalacyjne/04_GPO_Mapowanie/GPO_DYSKI/20.png}{20}{OZNACZ-20-GPO-Dyski}

\clearpage

\fg[\textwidth]{rys/04_Procedury_Instalacyjne/04_GPO_Mapowanie/GPO_DYSKI/21.png}{21}{OZNACZ-21-GPO-Dyski}

\clearpage

\fg[\textwidth]{rys/04_Procedury_Instalacyjne/04_GPO_Mapowanie/GPO_DYSKI/22.png}{22}{OZNACZ-22-GPO-Dyski}

\clearpage

\fg[\textwidth]{rys/04_Procedury_Instalacyjne/04_GPO_Mapowanie/GPO_DYSKI/23.png}{23}{OZNACZ-23-GPO-Dyski}

\clearpage

\fg[\textwidth]{rys/04_Procedury_Instalacyjne/04_GPO_Mapowanie/GPO_DYSKI/24.png}{24}{OZNACZ-24-GPO-Dyski}

\clearpage

\fg[\textwidth]{rys/04_Procedury_Instalacyjne/04_GPO_Mapowanie/GPO_DYSKI/25.png}{25}{OZNACZ-25-GPO-Dyski}

\clearpage

\fg[\textwidth]{rys/04_Procedury_Instalacyjne/04_GPO_Mapowanie/GPO_DYSKI/26.png}{26}{OZNACZ-26-GPO-Dyski}

\clearpage

\fg[\textwidth]{rys/04_Procedury_Instalacyjne/04_GPO_Mapowanie/GPO_DYSKI/27.png}{27}{OZNACZ-27-GPO-Dyski}

\clearpage

\fg[\textwidth]{rys/04_Procedury_Instalacyjne/04_GPO_Mapowanie/GPO_DYSKI/28.png}{28}{OZNACZ-28-GPO-Dyski}

\clearpage

\fg[\textwidth]{rys/04_Procedury_Instalacyjne/04_GPO_Mapowanie/GPO_DYSKI/29.png}{29}{OZNACZ-29-GPO-Dyski}

\clearpage

\fg[\textwidth]{rys/04_Procedury_Instalacyjne/04_GPO_Mapowanie/GPO_DYSKI/30.png}{30}{OZNACZ-30-GPO-Dyski}

\clearpage

\fg[\textwidth]{rys/04_Procedury_Instalacyjne/04_GPO_Mapowanie/GPO_DYSKI/31.png}{31}{OZNACZ-31-GPO-Dyski}

\clearpage

% *---------------------------------------------------------------------------------------------------------------------------%

\subsection{Zdjęcia dotyczące GPO RESZTA}

Po skonfigurowaniu udziałów GPO, należy skonfigurować pozostałe ustawienia GPO, czyli zmienną środowiskową i folder Projekty96. Proces ten przedstawiony jest na rysunkach od \OznaczZdjecie[Rys.]{oznacz-01-GPO-Reszta} do \OznaczZdjecie[Rys.]{oznacz-12-GPO-Reszta}.

\fg[\textwidth]{rys/04_Procedury_Instalacyjne/04_GPO_Mapowanie/GPO_RESZTA/01.png}{01}{oznacz-01-GPO-Reszta}

\clearpage

\fg[\textwidth]{rys/04_Procedury_Instalacyjne/04_GPO_Mapowanie/GPO_RESZTA/02.png}{02}{oznacz-02-GPO-Reszta}

\clearpage

\fg[\textwidth]{rys/04_Procedury_Instalacyjne/04_GPO_Mapowanie/GPO_RESZTA/03.png}{03}{oznacz-03-GPO-Reszta}

\clearpage

\fg[\textwidth]{rys/04_Procedury_Instalacyjne/04_GPO_Mapowanie/GPO_RESZTA/04.png}{04}{oznacz-04-GPO-Reszta}

\clearpage

\fg[\textwidth]{rys/04_Procedury_Instalacyjne/04_GPO_Mapowanie/GPO_RESZTA/05.png}{05}{oznacz-05-GPO-Reszta}

\clearpage

\fg[\textwidth]{rys/04_Procedury_Instalacyjne/04_GPO_Mapowanie/GPO_RESZTA/06.png}{06}{oznacz-06-GPO-Reszta}

\clearpage

\fg[\textwidth]{rys/04_Procedury_Instalacyjne/04_GPO_Mapowanie/GPO_RESZTA/07.png}{07}{oznacz-07-GPO-Reszta}

\clearpage

\fg[\textwidth]{rys/04_Procedury_Instalacyjne/04_GPO_Mapowanie/GPO_RESZTA/08.png}{08}{oznacz-08-GPO-Reszta}

\clearpage

\fg[\textwidth]{rys/04_Procedury_Instalacyjne/04_GPO_Mapowanie/GPO_RESZTA/09.png}{09}{oznacz-09-GPO-Reszta}

\clearpage

\fg[\textwidth]{rys/04_Procedury_Instalacyjne/04_GPO_Mapowanie/GPO_RESZTA/10.png}{10}{oznacz-10-GPO-Reszta}

\clearpage

\fg[\textwidth]{rys/04_Procedury_Instalacyjne/04_GPO_Mapowanie/GPO_RESZTA/11.png}{11}{oznacz-11-GPO-Reszta}

\clearpage

\fg[\textwidth]{rys/04_Procedury_Instalacyjne/04_GPO_Mapowanie/GPO_RESZTA/12.png}{12}{oznacz-12-GPO-Reszta}

\clearpage

% *---------------------------------------------------------------------------------------------------------------------------%

\subsection{Zdjęcia dotyczące DHCP}

Aby rozpocząć konfigurację DHCP należy uruchomić program `Server Manager` na serwerze SDC96 oraz dodać rolę DHCP do SN1-96 oraz SN2-96. Proces ten przedstawiony jest na rysunkach od \OznaczZdjecie[Rys.]{oznacz-01-DHCP} do \OznaczZdjecie[Rys.]{oznacz-04-DHCP}.

\clearpage

\fg[\textwidth]{rys/04_Procedury_Instalacyjne/05_DHCP/01.png}{01}{oznacz-01-DHCP}

\clearpage

\fg[\textwidth]{rys/04_Procedury_Instalacyjne/05_DHCP/02.png}{02}{oznacz-02-DHCP}

\clearpage

\fg[\textwidth]{rys/04_Procedury_Instalacyjne/05_DHCP/03.png}{03}{oznacz-03-DHCP}

\clearpage

\fg[\textwidth]{rys/04_Procedury_Instalacyjne/05_DHCP/04.png}{04}{oznacz-04-DHCP}

\clearpage

Po zakończeniu instalacji roli DHCP, należy dodać dysk do serwera SMP96. Proces ten przedstawiony jest na rysunkach od \OznaczZdjecie[Rys.]{oznacz-08-DHCP} do \OznaczZdjecie[Rys.]{oznacz-014-DHCP}.

\fg[\textwidth]{rys/04_Procedury_Instalacyjne/05_DHCP/08.png}{08}{oznacz-08-DHCP}

\clearpage

\fg[\textwidth]{rys/04_Procedury_Instalacyjne/05_DHCP/09.png}{09}{oznacz-09-DHCP}

\clearpage

\fg[\textwidth]{rys/04_Procedury_Instalacyjne/05_DHCP/10.png}{10}{oznacz-10-DHCP}

\clearpage

\fg[\textwidth]{rys/04_Procedury_Instalacyjne/05_DHCP/11.png}{11}{oznacz-11-DHCP}

\clearpage

\fg[\textwidth]{rys/04_Procedury_Instalacyjne/05_DHCP/12.png}{12}{oznacz-12-DHCP}

\clearpage

\fg[\textwidth]{rys/04_Procedury_Instalacyjne/05_DHCP/13.png}{13}{oznacz-13-DHCP}

\clearpage

\fg[\textwidth]{rys/04_Procedury_Instalacyjne/05_DHCP/14.png}{14}{oznacz-14-DHCP}

\clearpage

Nastepnie należy dodać ten dysk do klastra na serwerze SMP96. Proces ten przedstawiony jest na rysunkach od \OznaczZdjecie[Rys.]{oznacz-15-DHCP} do \OznaczZdjecie[Rys.]{oznacz-27-DHCP}.

\fg[\textwidth]{rys/04_Procedury_Instalacyjne/05_DHCP/15.png}{15}{oznacz-15-DHCP}

\clearpage


\fg[\textwidth]{rys/04_Procedury_Instalacyjne/05_DHCP/19.png}{19}{oznacz-19-DHCP}

\clearpage

\fg[\textwidth]{rys/04_Procedury_Instalacyjne/05_DHCP/20.png}{20}{oznacz-20-DHCP}

\clearpage

\fg[\textwidth]{rys/04_Procedury_Instalacyjne/05_DHCP/21.png}{21}{oznacz-21-DHCP}

\clearpage

\fg[\textwidth]{rys/04_Procedury_Instalacyjne/05_DHCP/22.png}{22}{oznacz-22-DHCP}

\clearpage

\fg[\textwidth]{rys/04_Procedury_Instalacyjne/05_DHCP/23.png}{23}{oznacz-23-DHCP}

\clearpage


\fg[\textwidth]{rys/04_Procedury_Instalacyjne/05_DHCP/25.png}{25}{oznacz-25-DHCP}

\clearpage

\fg[\textwidth]{rys/04_Procedury_Instalacyjne/05_DHCP/26.png}{26}{oznacz-26-DHCP}

\clearpage

\fg[\textwidth]{rys/04_Procedury_Instalacyjne/05_DHCP/27.png}{27}{oznacz-27-DHCP}

\clearpage

Teraz na serwerach SN1-96 oraz SN2-96 należy uruchomić narzędzie `iSCSI Initiator` i połączyć się z serwerem SMP96. Proces ten przedstawiony jest na rysunkach od \OznaczZdjecie[Rys.]{oznacz-28-DHCP} do \OznaczZdjecie[Rys.]{oznacz-31-DHCP}.

\fg[\textwidth]{rys/04_Procedury_Instalacyjne/05_DHCP/28.png}{28}{oznacz-28-DHCP}

\clearpage

\fg[\textwidth]{rys/04_Procedury_Instalacyjne/05_DHCP/29.png}{29}{oznacz-29-DHCP}

\clearpage

\fg[\textwidth]{rys/04_Procedury_Instalacyjne/05_DHCP/30.png}{30}{oznacz-30-DHCP}

\clearpage

\fg[\textwidth]{rys/04_Procedury_Instalacyjne/05_DHCP/31.png}{31}{oznacz-31-DHCP}

\clearpage

Potem na serwerze SN1-96 należy wejść w program `Disk Management`, powinniśmy tam zobaczyć nowy dysk. Proces jego konfiguracji przedstawiony jest na rysunkach od \OznaczZdjecie[Rys.]{oznacz-32-DHCP} do \OznaczZdjecie[Rys.]{oznacz-36-DHCP}.

\fg[\textwidth]{rys/04_Procedury_Instalacyjne/05_DHCP/32.png}{32}{oznacz-32-DHCP}

\clearpage

\fg[\textwidth]{rys/04_Procedury_Instalacyjne/05_DHCP/33.png}{33}{oznacz-33-DHCP}

\clearpage

\fg[\textwidth]{rys/04_Procedury_Instalacyjne/05_DHCP/34.png}{34}{oznacz-34-DHCP}

\clearpage

\fg[\textwidth]{rys/04_Procedury_Instalacyjne/05_DHCP/35.png}{35}{oznacz-35-DHCP}

\clearpage

\fg[\textwidth]{rys/04_Procedury_Instalacyjne/05_DHCP/36.png}{36}{oznacz-36-DHCP}

\clearpage

Następnie dodajemy nowy dysk do klastra na serwerze SN1-96. Proces ten przedstawiony jest na rysunkach od \OznaczZdjecie[Rys.]{oznacz-37-DHCP} do \OznaczZdjecie[Rys.]{oznacz-38-DHCP}.

\fg[\textwidth]{rys/04_Procedury_Instalacyjne/05_DHCP/37.png}{37}{oznacz-37-DHCP}

\clearpage

\fg[\textwidth]{rys/04_Procedury_Instalacyjne/05_DHCP/38.png}{38}{oznacz-38-DHCP}

\clearpage

Gdy ukończymy powyższe kroki, możemy zainstalować rolę DHCP na klastrze. Proces ten przedstawiony jest na rysunkach od \OznaczZdjecie[Rys.]{oznacz-39-DHCP} do \OznaczZdjecie[Rys.]{oznacz-43-DHCP}.

\fg[\textwidth]{rys/04_Procedury_Instalacyjne/05_DHCP/39.png}{39}{oznacz-39-DHCP}

\clearpage

\fg[\textwidth]{rys/04_Procedury_Instalacyjne/05_DHCP/40.png}{40}{oznacz-40-DHCP}

\clearpage

\fg[\textwidth]{rys/04_Procedury_Instalacyjne/05_DHCP/41.png}{41}{oznacz-41-DHCP}

\clearpage

\fg[\textwidth]{rys/04_Procedury_Instalacyjne/05_DHCP/42.png}{42}{oznacz-42-DHCP}

\clearpage

\fg[\textwidth]{rys/04_Procedury_Instalacyjne/05_DHCP/43.png}{43}{oznacz-43-DHCP}

\clearpage

Teraz, gdy już mamy zainstalowaną rolę DHCP na klastrze, możemy skonfigurować serwer DHCP i jego zakres(192.168.96.100-200). Proces ten przedstawiony jest na rysunkach od \OznaczZdjecie[Rys.]{oznacz-44-DHCP} do \OznaczZdjecie[Rys.]{oznacz-56-DHCP}.

\fg[\textwidth]{rys/04_Procedury_Instalacyjne/05_DHCP/44.png}{44}{oznacz-44-DHCP}

\clearpage

\fg[\textwidth]{rys/04_Procedury_Instalacyjne/05_DHCP/45.png}{45}{oznacz-45-DHCP}

\clearpage

\fg[\textwidth]{rys/04_Procedury_Instalacyjne/05_DHCP/46.png}{46}{oznacz-46-DHCP}

\clearpage

\fg[\textwidth]{rys/04_Procedury_Instalacyjne/05_DHCP/47.png}{47}{oznacz-47-DHCP}

\clearpage

\fg[\textwidth]{rys/04_Procedury_Instalacyjne/05_DHCP/48.png}{48}{oznacz-48-DHCP}

\clearpage

\fg[\textwidth]{rys/04_Procedury_Instalacyjne/05_DHCP/49.png}{49}{oznacz-49-DHCP}

\clearpage

\fg[\textwidth]{rys/04_Procedury_Instalacyjne/05_DHCP/50.png}{50}{oznacz-50-DHCP}

\clearpage

\fg[\textwidth]{rys/04_Procedury_Instalacyjne/05_DHCP/51.png}{51}{oznacz-51-DHCP}
Jako że nasza struktura nie posiada routera, adres ten można zignorować/ ustawić na jakikolwiek dostepny.
\clearpage

\fg[\textwidth]{rys/04_Procedury_Instalacyjne/05_DHCP/52.png}{52}{oznacz-52-DHCP}

\clearpage

\fg[\textwidth]{rys/04_Procedury_Instalacyjne/05_DHCP/53.png}{53}{oznacz-53-DHCP}

\clearpage

\fg[\textwidth]{rys/04_Procedury_Instalacyjne/05_DHCP/54.png}{54}{oznacz-54-DHCP}

\clearpage

%TODO DODAĆ DO TESTÓW POZNIEJ te 2 zdjecia

\fg[\textwidth]{rys/04_Procedury_Instalacyjne/05_DHCP/55.png}{55}{oznacz-55-DHCP}

\clearpage

\fg[\textwidth]{rys/04_Procedury_Instalacyjne/05_DHCP/56.png}{56}{oznacz-56-DHCP}

\clearpage

% *---------------------------------------------------------------------------------------------------------------------------%

\subsection{Zdjęcia dotyczące Serwera Wydruków}

Aby rozpocząć konfigurację serwera wydruków, należy dodać wirtualną drukarke do serwera SDC96 oraz skonfigurować ją. Proces ten przedstawiony jest na rysunkach od \OznaczZdjecie[Rys.]{oznacz-01-Print} do \OznaczZdjecie[Rys.]{oznacz-07-Print}.

\clearpage

\fg[\textwidth]{rys/04_Procedury_Instalacyjne/07_Serwer_Wydrukow/01.png}{01}{oznacz-01-Print}

\clearpage

\fg[\textwidth]{rys/04_Procedury_Instalacyjne/07_Serwer_Wydrukow/02.png}{02}{oznacz-02-Print}

\clearpage

\fg[\textwidth]{rys/04_Procedury_Instalacyjne/07_Serwer_Wydrukow/03.png}{03}{oznacz-03-Print}

\clearpage

\fg[\textwidth]{rys/04_Procedury_Instalacyjne/07_Serwer_Wydrukow/04.png}{04}{oznacz-04-Print}

\clearpage

\fg[\textwidth]{rys/04_Procedury_Instalacyjne/07_Serwer_Wydrukow/05.png}{05}{oznacz-05-Print}

\clearpage

\fg[\textwidth]{rys/04_Procedury_Instalacyjne/07_Serwer_Wydrukow/06.png}{06}{oznacz-06-Print}

\clearpage

\fg[\textwidth]{rys/04_Procedury_Instalacyjne/07_Serwer_Wydrukow/07.png}{07}{oznacz-07-Print}

\clearpage

Następnie należy dodać role `Print and Document Services' na serwerze SDC96. Proces ten przedstawiony jest na rysunkach od \OznaczZdjecie[Rys.]{oznacz-08-Print} do \OznaczZdjecie[Rys.]{oznacz-11-Print}. 

\fg[\textwidth]{rys/04_Procedury_Instalacyjne/07_Serwer_Wydrukow/08.png}{08}{oznacz-08-Print}

\clearpage

\fg[\textwidth]{rys/04_Procedury_Instalacyjne/07_Serwer_Wydrukow/09.png}{09}{oznacz-09-Print}

\clearpage

\fg[\textwidth]{rys/04_Procedury_Instalacyjne/07_Serwer_Wydrukow/10.png}{10}{oznacz-10-Print}

\clearpage

\fg[\textwidth]{rys/04_Procedury_Instalacyjne/07_Serwer_Wydrukow/11.png}{11}{oznacz-11-Print}

\clearpage

\fg[\textwidth]{rys/04_Procedury_Instalacyjne/07_Serwer_Wydrukow/12.png}{12}{oznacz-12-Print}

\clearpage

Następnie możemy się połączyć z drukarką na maszynie klienckiej co widać na rysunku \OznaczZdjecie[Rys.]{oznacz-13-Print}. 

\fg[\textwidth]{rys/04_Procedury_Instalacyjne/07_Serwer_Wydrukow/13.png}{13}{oznacz-13-Print}

\clearpage

%TODO DODAĆ DO TESTÓW POZNIEJ 

\fg[\textwidth]{rys/04_Procedury_Instalacyjne/07_Serwer_Wydrukow/14.png}{14}{oznacz-14-Print}
\OznaczZdjecie[]{oznacz-14-Print}
\clearpage

\fg[\textwidth]{rys/04_Procedury_Instalacyjne/07_Serwer_Wydrukow/15.png}{15}{oznacz-15-Print}
\OznaczZdjecie[]{oznacz-15-Print}
\clearpage

% *---------------------------------------------------------------------------------------------------------------------------%

\subsection{Zdjęcia dotyczące Instalacji Stacji Klienckich}

Proces instalacji stacji klienckich rozpoczynamy od uruchomienia komputera. Następnie należy dodać go do domeny i aktywować DHCP. Proces ten przedstawiony jest na rysunkach od \OznaczZdjecie[Rys.]{oznacz-01-client} do \OznaczZdjecie[Rys.]{oznacz-05-client}.


\fg[\textwidth]{rys/04_Procedury_Instalacyjne/08_Instalacja_Stacji_Klienckich/01.png}{01}{oznacz-01-client}
\OznaczZdjecie[]{oznacz-01-client}
\clearpage

\fg[\textwidth]{rys/04_Procedury_Instalacyjne/08_Instalacja_Stacji_Klienckich/02-DNS.png}{02-DNS}{oznacz-02-client}
\OznaczZdjecie[]{oznacz-02-client}
\clearpage

\fg[\textwidth]{rys/04_Procedury_Instalacyjne/08_Instalacja_Stacji_Klienckich/02.png}{02}{oznacz-02-client}
\OznaczZdjecie[]{oznacz-03-client}
\clearpage

\fg[\textwidth]{rys/04_Procedury_Instalacyjne/08_Instalacja_Stacji_Klienckich/03-DNS.png}{03-DNS}{oznacz-03-client}
\OznaczZdjecie[]{oznacz-04-client}
\clearpage

\fg[\textwidth]{rys/04_Procedury_Instalacyjne/08_Instalacja_Stacji_Klienckich/04-DNS.png}{04-DNS}{oznacz-04-client}
\OznaczZdjecie[]{oznacz-05-client}
\clearpage

% *---------------------------------------------------------------------------------------------------------------------------%

\subsection{Zdjęcia dotyczące WDS}

Konfiguracja WDS rozpoczyna się od dodania roli WDS na serwerze SDC96. Proces ten przedstawiony jest na rysunkach od \OznaczZdjecie[Rys.]{OZNACZ-WDS-01} do \OznaczZdjecie[Rys.]{OZNACZ-WDS-04}.

\fg[\textwidth]{rys/04_Procedury_Instalacyjne/08_Instalacja_Stacji_Klienckich/WDS/01.png}{WDS-01}{OZNACZ-WDS-01}

\clearpage

\fg[\textwidth]{rys/04_Procedury_Instalacyjne/08_Instalacja_Stacji_Klienckich/WDS/02.png}{WDS-02}{OZNACZ-WDS-02}

\clearpage

\fg[\textwidth]{rys/04_Procedury_Instalacyjne/08_Instalacja_Stacji_Klienckich/WDS/03.png}{WDS-03}{OZNACZ-WDS-03}

\clearpage

\fg[\textwidth]{rys/04_Procedury_Instalacyjne/08_Instalacja_Stacji_Klienckich/WDS/04.png}{WDS-04}{OZNACZ-WDS-04}

\clearpage

Nastepnie możemy skonfigurować WDS. Proces ten przedstawiony jest na rysunkach od \OznaczZdjecie[Rys.]{OZNACZ-WDS-05} do \OznaczZdjecie[Rys.]{OZNACZ-WDS-17}.

\fg[\textwidth]{rys/04_Procedury_Instalacyjne/08_Instalacja_Stacji_Klienckich/WDS/05.png}{WDS-05}{OZNACZ-WDS-05}

\clearpage

\fg[\textwidth]{rys/04_Procedury_Instalacyjne/08_Instalacja_Stacji_Klienckich/WDS/06.png}{WDS-06}{OZNACZ-WDS-06}

\clearpage

\fg[\textwidth]{rys/04_Procedury_Instalacyjne/08_Instalacja_Stacji_Klienckich/WDS/07.png}{WDS-07}{OZNACZ-WDS-07}

\clearpage

\fg[\textwidth]{rys/04_Procedury_Instalacyjne/08_Instalacja_Stacji_Klienckich/WDS/08.png}{WDS-08}{OZNACZ-WDS-08}

\clearpage

\fg[\textwidth]{rys/04_Procedury_Instalacyjne/08_Instalacja_Stacji_Klienckich/WDS/09.png}{WDS-09}{OZNACZ-WDS-09}

\clearpage

\fg[\textwidth]{rys/04_Procedury_Instalacyjne/08_Instalacja_Stacji_Klienckich/WDS/10.png}{WDS-10}{OZNACZ-WDS-10}

\clearpage



\fg[\textwidth]{rys/04_Procedury_Instalacyjne/08_Instalacja_Stacji_Klienckich/WDS/12.png}{WDS-12}{OZNACZ-WDS-12}

\clearpage

\fg[\textwidth]{rys/04_Procedury_Instalacyjne/08_Instalacja_Stacji_Klienckich/WDS/13.png}{WDS-13}{OZNACZ-WDS-13}
\OznaczZdjecie[]{OZNACZ-WDS-13}
\clearpage

\fg[\textwidth]{rys/04_Procedury_Instalacyjne/08_Instalacja_Stacji_Klienckich/WDS/14.png}{WDS-14}{OZNACZ-WDS-14}
\OznaczZdjecie[]{OZNACZ-WDS-14}
\clearpage

\fg[\textwidth]{rys/04_Procedury_Instalacyjne/08_Instalacja_Stacji_Klienckich/WDS/11.png}{WDS-11}{OZNACZ-WDS-11}
\OznaczZdjecie[]{OZNACZ-WDS-11}
\clearpage

\fg[\textwidth]{rys/04_Procedury_Instalacyjne/08_Instalacja_Stacji_Klienckich/WDS/15.png}{WDS-15}{OZNACZ-WDS-15}
\OznaczZdjecie[]{OZNACZ-WDS-15}
\clearpage

\fg[\textwidth]{rys/04_Procedury_Instalacyjne/08_Instalacja_Stacji_Klienckich/WDS/16.png}{WDS-16}{OZNACZ-WDS-16}
\OznaczZdjecie[]{OZNACZ-WDS-16}
\clearpage

\fg[\textwidth]{rys/04_Procedury_Instalacyjne/08_Instalacja_Stacji_Klienckich/WDS/17.png}{WDS-17}{OZNACZ-WDS-17}
\OznaczZdjecie[]{OZNACZ-WDS-17}
\clearpage

\fg[\textwidth]{rys/04_Procedury_Instalacyjne/08_Instalacja_Stacji_Klienckich/WDS/18.png}{WDS-18}{OZNACZ-WDS-18}
\OznaczZdjecie[]{OZNACZ-WDS-18}
\clearpage

\fg[\textwidth]{rys/04_Procedury_Instalacyjne/08_Instalacja_Stacji_Klienckich/WDS/19.png}{WDS-19}{OZNACZ-WDS-19}
\OznaczZdjecie[]{OZNACZ-WDS-19}
\clearpage

\fg[\textwidth]{rys/04_Procedury_Instalacyjne/08_Instalacja_Stacji_Klienckich/WDS/20.png}{WDS-20}{OZNACZ-WDS-20}
\OznaczZdjecie[]{OZNACZ-WDS-20}
\clearpage

\fg[\textwidth]{rys/04_Procedury_Instalacyjne/08_Instalacja_Stacji_Klienckich/WDS/21.png}{WDS-21}{OZNACZ-WDS-21}
\OznaczZdjecie[]{OZNACZ-WDS-21}
\clearpage

\fg[\textwidth]{rys/04_Procedury_Instalacyjne/08_Instalacja_Stacji_Klienckich/WDS/22.png}{WDS-22}{OZNACZ-WDS-22}
\OznaczZdjecie[]{OZNACZ-WDS-22}
\clearpage

\fg[\textwidth]{rys/04_Procedury_Instalacyjne/08_Instalacja_Stacji_Klienckich/WDS/23.png}{WDS-23}{OZNACZ-WDS-23}
\OznaczZdjecie[]{OZNACZ-WDS-23}
\clearpage

\fg[\textwidth]{rys/04_Procedury_Instalacyjne/08_Instalacja_Stacji_Klienckich/WDS/24.png}{WDS-24}{OZNACZ-WDS-24}
\OznaczZdjecie[]{OZNACZ-WDS-24}
\clearpage

\fg[\textwidth]{rys/04_Procedury_Instalacyjne/08_Instalacja_Stacji_Klienckich/WDS/25.png}{WDS-25}{OZNACZ-WDS-25}
\OznaczZdjecie[]{OZNACZ-WDS-25}
\clearpage

\fg[\textwidth]{rys/04_Procedury_Instalacyjne/08_Instalacja_Stacji_Klienckich/WDS/26.png}{WDS-26}{OZNACZ-WDS-26}
\OznaczZdjecie[]{OZNACZ-WDS-26}
\clearpage

\fg[\textwidth]{rys/04_Procedury_Instalacyjne/08_Instalacja_Stacji_Klienckich/WDS/27.png}{WDS-27}{OZNACZ-WDS-27}
\OznaczZdjecie[]{OZNACZ-WDS-27}
\clearpage

\fg[\textwidth]{rys/04_Procedury_Instalacyjne/08_Instalacja_Stacji_Klienckich/WDS/28.png}{WDS-28}{OZNACZ-WDS-28}
\OznaczZdjecie[]{OZNACZ-WDS-28}
\clearpage

\fg[\textwidth]{rys/04_Procedury_Instalacyjne/08_Instalacja_Stacji_Klienckich/WDS/29.png}{WDS-29}{OZNACZ-WDS-29}
\OznaczZdjecie[]{OZNACZ-WDS-29}
\clearpage

% TODO DODAĆ DO TESTÓW POZNIEJ

\fg[\textwidth]{rys/04_Procedury_Instalacyjne/08_Instalacja_Stacji_Klienckich/WDS/30.png}{WDS-30}{OZNACZ-WDS-30}
\OznaczZdjecie[]{OZNACZ-WDS-30}
\clearpage

% *---------------------------------------------------------------------------------------------------------------------------%



\subsection{Zdjęcia dotyczące IIS}

\clearpage

\fg[\textwidth]{rys/04_Procedury_Instalacyjne/06_WWW_WordPress/IIS/01.png}{01}{OZNACZ-01}
\OznaczZdjecie[]{OZNACZ-01}
\clearpage

\fg[\textwidth]{rys/04_Procedury_Instalacyjne/06_WWW_WordPress/IIS/02.png}{02}{OZNACZ-02}
\OznaczZdjecie[]{OZNACZ-02}
\clearpage

\fg[\textwidth]{rys/04_Procedury_Instalacyjne/06_WWW_WordPress/IIS/03.png}{03}{OZNACZ-03}
\OznaczZdjecie[]{OZNACZ-03}
\clearpage

\fg[\textwidth]{rys/04_Procedury_Instalacyjne/06_WWW_WordPress/IIS/04.png}{04}{OZNACZ-04}
\OznaczZdjecie[]{OZNACZ-04}
\clearpage

\fg[\textwidth]{rys/04_Procedury_Instalacyjne/06_WWW_WordPress/IIS/05.png}{05}{OZNACZ-05}
\OznaczZdjecie[]{OZNACZ-05}
\clearpage

\fg[\textwidth]{rys/04_Procedury_Instalacyjne/06_WWW_WordPress/IIS/06.png}{06}{OZNACZ-06}
\OznaczZdjecie[]{OZNACZ-06}
\clearpage

\fg[\textwidth]{rys/04_Procedury_Instalacyjne/06_WWW_WordPress/IIS/07.png}{07}{OZNACZ-07}
\OznaczZdjecie[]{OZNACZ-07}
\clearpage

\fg[\textwidth]{rys/04_Procedury_Instalacyjne/06_WWW_WordPress/IIS/08.png}{08}{OZNACZ-08}
\OznaczZdjecie[]{OZNACZ-08}
\clearpage

\fg[\textwidth]{rys/04_Procedury_Instalacyjne/06_WWW_WordPress/IIS/09.png}{09}{OZNACZ-09}
\OznaczZdjecie[]{OZNACZ-09}
\clearpage

\fg[\textwidth]{rys/04_Procedury_Instalacyjne/06_WWW_WordPress/IIS/10.png}{10}{OZNACZ-10}
\OznaczZdjecie[]{OZNACZ-10}
\clearpage

\fg[\textwidth]{rys/04_Procedury_Instalacyjne/06_WWW_WordPress/IIS/11.png}{11}{OZNACZ-11}
\OznaczZdjecie[]{OZNACZ-11}
\clearpage

% *---------------------------------------------------------------------------------------------------------------------------%

\subsection{Zdjęcia dotyczące WordPress}
\label{sec:WP}

\clearpage
\fg[\textwidth]{rys/04_Procedury_Instalacyjne/06_WWW_WordPress/WordPress/01.png}{01}{oznacz-01-WP}
\OznaczZdjecie[]{oznacz-01-WP}
\clearpage

\fg[\textwidth]{rys/04_Procedury_Instalacyjne/06_WWW_WordPress/WordPress/02.png}{02}{oznacz-02-WP}
\OznaczZdjecie[]{oznacz-02-WP}
\clearpage

\fg[\textwidth]{rys/04_Procedury_Instalacyjne/06_WWW_WordPress/WordPress/03.png}{03}{oznacz-03-WP}
\OznaczZdjecie[]{oznacz-03-WP}
\clearpage

\fg[\textwidth]{rys/04_Procedury_Instalacyjne/06_WWW_WordPress/WordPress/04.png}{04}{oznacz-04-WP}
\OznaczZdjecie[]{oznacz-04-WP}
\clearpage

\fg[\textwidth]{rys/04_Procedury_Instalacyjne/06_WWW_WordPress/WordPress/05.png}{05}{oznacz-05-WP}
\OznaczZdjecie[]{oznacz-05-WP}
\clearpage

\fg[\textwidth]{rys/04_Procedury_Instalacyjne/06_WWW_WordPress/WordPress/06.png}{06}{oznacz-06-WP}
\OznaczZdjecie[]{oznacz-06-WP}
\clearpage

\fg[\textwidth]{rys/04_Procedury_Instalacyjne/06_WWW_WordPress/WordPress/07.png}{07}{oznacz-07-WP}
\OznaczZdjecie[]{oznacz-07-WP}
\clearpage

\fg[\textwidth]{rys/04_Procedury_Instalacyjne/06_WWW_WordPress/WordPress/08.png}{08}{oznacz-08-WP}
\OznaczZdjecie[]{oznacz-08-WP}
\clearpage

\fg[\textwidth]{rys/04_Procedury_Instalacyjne/06_WWW_WordPress/WordPress/09.png}{09}{oznacz-09-WP}
\OznaczZdjecie[]{oznacz-09-WP}
\clearpage

\fg[\textwidth]{rys/04_Procedury_Instalacyjne/06_WWW_WordPress/WordPress/10.png}{10}{oznacz-10-WP}
\OznaczZdjecie[]{oznacz-10-WP}
\clearpage

\fg[\textwidth]{rys/04_Procedury_Instalacyjne/06_WWW_WordPress/WordPress/11.png}{11}{oznacz-11-WP}
\OznaczZdjecie[]{oznacz-11-WP}
\clearpage

\fg[\textwidth]{rys/04_Procedury_Instalacyjne/06_WWW_WordPress/WordPress/12.png}{12}{oznacz-12-WP}
\OznaczZdjecie[]{oznacz-12-WP}
\clearpage

\fg[\textwidth]{rys/04_Procedury_Instalacyjne/06_WWW_WordPress/WordPress/13.png}{13}{oznacz-13-WP}
\OznaczZdjecie[]{oznacz-13-WP}
\clearpage

\fg[\textwidth]{rys/04_Procedury_Instalacyjne/06_WWW_WordPress/WordPress/14.png}{14}{oznacz-14-WP}
\OznaczZdjecie[]{oznacz-14-WP}
\clearpage

\fg[\textwidth]{rys/04_Procedury_Instalacyjne/06_WWW_WordPress/WordPress/15.png}{15}{oznacz-15-WP}
\OznaczZdjecie[]{oznacz-15-WP}
\clearpage

\fg[\textwidth]{rys/04_Procedury_Instalacyjne/06_WWW_WordPress/WordPress/16.png}{16}{oznacz-16-WP}
\OznaczZdjecie[]{oznacz-16-WP}
\clearpage

\fg[\textwidth]{rys/04_Procedury_Instalacyjne/06_WWW_WordPress/WordPress/17.png}{17}{oznacz-17-WP}
\OznaczZdjecie[]{oznacz-17-WP}
\clearpage

\fg[\textwidth]{rys/04_Procedury_Instalacyjne/06_WWW_WordPress/WordPress/18.png}{18}{oznacz-18-WP}
\OznaczZdjecie[]{oznacz-18-WP}
\clearpage

\fg[\textwidth]{rys/04_Procedury_Instalacyjne/06_WWW_WordPress/WordPress/19.png}{19}{oznacz-19-WP}
\OznaczZdjecie[]{oznacz-19-WP}
\clearpage

\fg[\textwidth]{rys/04_Procedury_Instalacyjne/06_WWW_WordPress/WordPress/20.png}{20}{oznacz-20-WP}
\OznaczZdjecie[]{oznacz-20-WP}
\clearpage

\fg[\textwidth]{rys/04_Procedury_Instalacyjne/06_WWW_WordPress/WordPress/21.png}{21}{oznacz-21-WP}
\OznaczZdjecie[]{oznacz-21-WP}
\clearpage

\fg[\textwidth]{rys/04_Procedury_Instalacyjne/06_WWW_WordPress/WordPress/22.png}{22}{oznacz-22-WP}
\OznaczZdjecie[]{oznacz-22-WP}
\clearpage

\fg[\textwidth]{rys/04_Procedury_Instalacyjne/06_WWW_WordPress/WordPress/23.png}{23}{oznacz-23-WP}
\OznaczZdjecie[]{oznacz-23-WP}
\clearpage

\fg[\textwidth]{rys/04_Procedury_Instalacyjne/06_WWW_WordPress/WordPress/24.png}{24}{oznacz-24-WP}
\OznaczZdjecie[]{oznacz-24-WP}
\clearpage

\fg[\textwidth]{rys/04_Procedury_Instalacyjne/06_WWW_WordPress/WordPress/25.png}{25}{oznacz-25-WP}
\OznaczZdjecie[]{oznacz-25-WP}
\clearpage

\fg[\textwidth]{rys/04_Procedury_Instalacyjne/06_WWW_WordPress/WordPress/26.png}{26}{oznacz-26-WP}
\OznaczZdjecie[]{oznacz-26-WP}
\clearpage

\fg[\textwidth]{rys/04_Procedury_Instalacyjne/06_WWW_WordPress/WordPress/27.png}{27}{oznacz-27-WP}
\OznaczZdjecie[]{oznacz-27-WP}
\clearpage

\fg[\textwidth]{rys/04_Procedury_Instalacyjne/06_WWW_WordPress/WordPress/28.png}{28}{oznacz-28-WP}
\OznaczZdjecie[]{oznacz-28-WP}
\clearpage

\fg[\textwidth]{rys/04_Procedury_Instalacyjne/06_WWW_WordPress/WordPress/29.png}{29}{oznacz-29-WP}
\OznaczZdjecie[]{oznacz-29-WP}
\clearpage

\fg[\textwidth]{rys/04_Procedury_Instalacyjne/06_WWW_WordPress/WordPress/30.png}{30}{oznacz-30-WP}
\OznaczZdjecie[]{oznacz-30-WP}
\clearpage
   	\newpage
\section{Testy działania wdrożonych usług}	%5
% (W podpunktach zamieścić zrzuty ekranów pokazujące działanie wdrożonych usług)


   	\newpage
\section{Wnioski}	%5
%Npisać wnioski końcowe z przeprowadzonego projektu, 



   
       
%%%%%%%%%%%%%%%%%%% koniec treść główna dokumentu %%%%%%%%%%%%%%%%%%%%%
	\newpage
    % \addcontentsline{toc}{section}{Literatura}
    % Modified by: Maciej Wójs  
    \printbibliography[heading=bibnumbered, label=Literatura, title=Literatura]

    \newpage
    \hypersetup{linkcolor=black}
    \renewcommand{\cftparskip}{3pt}
    \clearpage
    \renewcommand{\cftloftitlefont}{\Large\bfseries\sffamily}
    \listoffigures
    \addcontentsline{toc}{section}{Spis rysunków}
	\thispagestyle{fancy}
	
    \newpage
    \renewcommand{\cftlottitlefont}{\Large\bfseries\sffamily}
    \def\listtablename{Spis tabel}
    \addcontentsline{toc}{section}{Spis tabel}\listoftables 
	\thispagestyle{fancy}
	
	\newpage
	\renewcommand{\cftlottitlefont}{\Large\bfseries\sffamily}
	\renewcommand\lstlistlistingname{Spis listingów}
	\addcontentsline{toc}{section}{Spis listingów}\lstlistoflistings 
	\thispagestyle{fancy}
    \label{LastPage}
	


    %lista rzeczy do zrobienia: wypisuje na koñcu dokumentu, patrz: pakiet todo.sty
    \todos
    %koniec listy rzeczy do zrobienia
\end{document}
